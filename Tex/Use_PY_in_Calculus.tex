
% Default to the notebook output style

    


% Inherit from the specified cell style.




    
\documentclass[11pt]{article}

    
    
    \usepackage[T1]{fontenc}
    % Nicer default font (+ math font) than Computer Modern for most use cases
    \usepackage{mathpazo}

    % Basic figure setup, for now with no caption control since it's done
    % automatically by Pandoc (which extracts ![](path) syntax from Markdown).
    \usepackage{graphicx}
    % We will generate all images so they have a width \maxwidth. This means
    % that they will get their normal width if they fit onto the page, but
    % are scaled down if they would overflow the margins.
    \makeatletter
    \def\maxwidth{\ifdim\Gin@nat@width>\linewidth\linewidth
    \else\Gin@nat@width\fi}
    \makeatother
    \let\Oldincludegraphics\includegraphics
    % Set max figure width to be 80% of text width, for now hardcoded.
    \renewcommand{\includegraphics}[1]{\Oldincludegraphics[width=.8\maxwidth]{#1}}
    % Ensure that by default, figures have no caption (until we provide a
    % proper Figure object with a Caption API and a way to capture that
    % in the conversion process - todo).
    \usepackage{caption}
    \DeclareCaptionLabelFormat{nolabel}{}
    \captionsetup{labelformat=nolabel}

    \usepackage{adjustbox} % Used to constrain images to a maximum size 
    \usepackage{xcolor} % Allow colors to be defined
    \usepackage{enumerate} % Needed for markdown enumerations to work
    \usepackage{geometry} % Used to adjust the document margins
    \usepackage{amsmath} % Equations
    \usepackage{amssymb} % Equations
    \usepackage{textcomp} % defines textquotesingle
    % Hack from http://tex.stackexchange.com/a/47451/13684:
    \AtBeginDocument{%
        \def\PYZsq{\textquotesingle}% Upright quotes in Pygmentized code
    }
    \usepackage{upquote} % Upright quotes for verbatim code
    \usepackage{eurosym} % defines \euro
    \usepackage[mathletters]{ucs} % Extended unicode (utf-8) support
    \usepackage[utf8x]{inputenc} % Allow utf-8 characters in the tex document
    \usepackage{fancyvrb} % verbatim replacement that allows latex
    \usepackage{grffile} % extends the file name processing of package graphics 
                         % to support a larger range 
    % The hyperref package gives us a pdf with properly built
    % internal navigation ('pdf bookmarks' for the table of contents,
    % internal cross-reference links, web links for URLs, etc.)
    \usepackage{hyperref}
    \usepackage{longtable} % longtable support required by pandoc >1.10
    \usepackage{booktabs}  % table support for pandoc > 1.12.2
    \usepackage[inline]{enumitem} % IRkernel/repr support (it uses the enumerate* environment)
    \usepackage[normalem]{ulem} % ulem is needed to support strikethroughs (\sout)
                                % normalem makes italics be italics, not underlines
    

    
    
    % Colors for the hyperref package
    \definecolor{urlcolor}{rgb}{0,.145,.698}
    \definecolor{linkcolor}{rgb}{.71,0.21,0.01}
    \definecolor{citecolor}{rgb}{.12,.54,.11}

    % ANSI colors
    \definecolor{ansi-black}{HTML}{3E424D}
    \definecolor{ansi-black-intense}{HTML}{282C36}
    \definecolor{ansi-red}{HTML}{E75C58}
    \definecolor{ansi-red-intense}{HTML}{B22B31}
    \definecolor{ansi-green}{HTML}{00A250}
    \definecolor{ansi-green-intense}{HTML}{007427}
    \definecolor{ansi-yellow}{HTML}{DDB62B}
    \definecolor{ansi-yellow-intense}{HTML}{B27D12}
    \definecolor{ansi-blue}{HTML}{208FFB}
    \definecolor{ansi-blue-intense}{HTML}{0065CA}
    \definecolor{ansi-magenta}{HTML}{D160C4}
    \definecolor{ansi-magenta-intense}{HTML}{A03196}
    \definecolor{ansi-cyan}{HTML}{60C6C8}
    \definecolor{ansi-cyan-intense}{HTML}{258F8F}
    \definecolor{ansi-white}{HTML}{C5C1B4}
    \definecolor{ansi-white-intense}{HTML}{A1A6B2}

    % commands and environments needed by pandoc snippets
    % extracted from the output of `pandoc -s`
    \providecommand{\tightlist}{%
      \setlength{\itemsep}{0pt}\setlength{\parskip}{0pt}}
    \DefineVerbatimEnvironment{Highlighting}{Verbatim}{commandchars=\\\{\}}
    % Add ',fontsize=\small' for more characters per line
    \newenvironment{Shaded}{}{}
    \newcommand{\KeywordTok}[1]{\textcolor[rgb]{0.00,0.44,0.13}{\textbf{{#1}}}}
    \newcommand{\DataTypeTok}[1]{\textcolor[rgb]{0.56,0.13,0.00}{{#1}}}
    \newcommand{\DecValTok}[1]{\textcolor[rgb]{0.25,0.63,0.44}{{#1}}}
    \newcommand{\BaseNTok}[1]{\textcolor[rgb]{0.25,0.63,0.44}{{#1}}}
    \newcommand{\FloatTok}[1]{\textcolor[rgb]{0.25,0.63,0.44}{{#1}}}
    \newcommand{\CharTok}[1]{\textcolor[rgb]{0.25,0.44,0.63}{{#1}}}
    \newcommand{\StringTok}[1]{\textcolor[rgb]{0.25,0.44,0.63}{{#1}}}
    \newcommand{\CommentTok}[1]{\textcolor[rgb]{0.38,0.63,0.69}{\textit{{#1}}}}
    \newcommand{\OtherTok}[1]{\textcolor[rgb]{0.00,0.44,0.13}{{#1}}}
    \newcommand{\AlertTok}[1]{\textcolor[rgb]{1.00,0.00,0.00}{\textbf{{#1}}}}
    \newcommand{\FunctionTok}[1]{\textcolor[rgb]{0.02,0.16,0.49}{{#1}}}
    \newcommand{\RegionMarkerTok}[1]{{#1}}
    \newcommand{\ErrorTok}[1]{\textcolor[rgb]{1.00,0.00,0.00}{\textbf{{#1}}}}
    \newcommand{\NormalTok}[1]{{#1}}
    
    % Additional commands for more recent versions of Pandoc
    \newcommand{\ConstantTok}[1]{\textcolor[rgb]{0.53,0.00,0.00}{{#1}}}
    \newcommand{\SpecialCharTok}[1]{\textcolor[rgb]{0.25,0.44,0.63}{{#1}}}
    \newcommand{\VerbatimStringTok}[1]{\textcolor[rgb]{0.25,0.44,0.63}{{#1}}}
    \newcommand{\SpecialStringTok}[1]{\textcolor[rgb]{0.73,0.40,0.53}{{#1}}}
    \newcommand{\ImportTok}[1]{{#1}}
    \newcommand{\DocumentationTok}[1]{\textcolor[rgb]{0.73,0.13,0.13}{\textit{{#1}}}}
    \newcommand{\AnnotationTok}[1]{\textcolor[rgb]{0.38,0.63,0.69}{\textbf{\textit{{#1}}}}}
    \newcommand{\CommentVarTok}[1]{\textcolor[rgb]{0.38,0.63,0.69}{\textbf{\textit{{#1}}}}}
    \newcommand{\VariableTok}[1]{\textcolor[rgb]{0.10,0.09,0.49}{{#1}}}
    \newcommand{\ControlFlowTok}[1]{\textcolor[rgb]{0.00,0.44,0.13}{\textbf{{#1}}}}
    \newcommand{\OperatorTok}[1]{\textcolor[rgb]{0.40,0.40,0.40}{{#1}}}
    \newcommand{\BuiltInTok}[1]{{#1}}
    \newcommand{\ExtensionTok}[1]{{#1}}
    \newcommand{\PreprocessorTok}[1]{\textcolor[rgb]{0.74,0.48,0.00}{{#1}}}
    \newcommand{\AttributeTok}[1]{\textcolor[rgb]{0.49,0.56,0.16}{{#1}}}
    \newcommand{\InformationTok}[1]{\textcolor[rgb]{0.38,0.63,0.69}{\textbf{\textit{{#1}}}}}
    \newcommand{\WarningTok}[1]{\textcolor[rgb]{0.38,0.63,0.69}{\textbf{\textit{{#1}}}}}
    
    
    % Define a nice break command that doesn't care if a line doesn't already
    % exist.
    \def\br{\hspace*{\fill} \\* }
    % Math Jax compatability definitions
    \def\gt{>}
    \def\lt{<}
    % Document parameters
    \title{Use\_PY\_in\_Calculus}
    
    
    

    % Pygments definitions
    
\makeatletter
\def\PY@reset{\let\PY@it=\relax \let\PY@bf=\relax%
    \let\PY@ul=\relax \let\PY@tc=\relax%
    \let\PY@bc=\relax \let\PY@ff=\relax}
\def\PY@tok#1{\csname PY@tok@#1\endcsname}
\def\PY@toks#1+{\ifx\relax#1\empty\else%
    \PY@tok{#1}\expandafter\PY@toks\fi}
\def\PY@do#1{\PY@bc{\PY@tc{\PY@ul{%
    \PY@it{\PY@bf{\PY@ff{#1}}}}}}}
\def\PY#1#2{\PY@reset\PY@toks#1+\relax+\PY@do{#2}}

\expandafter\def\csname PY@tok@w\endcsname{\def\PY@tc##1{\textcolor[rgb]{0.73,0.73,0.73}{##1}}}
\expandafter\def\csname PY@tok@c\endcsname{\let\PY@it=\textit\def\PY@tc##1{\textcolor[rgb]{0.25,0.50,0.50}{##1}}}
\expandafter\def\csname PY@tok@cp\endcsname{\def\PY@tc##1{\textcolor[rgb]{0.74,0.48,0.00}{##1}}}
\expandafter\def\csname PY@tok@k\endcsname{\let\PY@bf=\textbf\def\PY@tc##1{\textcolor[rgb]{0.00,0.50,0.00}{##1}}}
\expandafter\def\csname PY@tok@kp\endcsname{\def\PY@tc##1{\textcolor[rgb]{0.00,0.50,0.00}{##1}}}
\expandafter\def\csname PY@tok@kt\endcsname{\def\PY@tc##1{\textcolor[rgb]{0.69,0.00,0.25}{##1}}}
\expandafter\def\csname PY@tok@o\endcsname{\def\PY@tc##1{\textcolor[rgb]{0.40,0.40,0.40}{##1}}}
\expandafter\def\csname PY@tok@ow\endcsname{\let\PY@bf=\textbf\def\PY@tc##1{\textcolor[rgb]{0.67,0.13,1.00}{##1}}}
\expandafter\def\csname PY@tok@nb\endcsname{\def\PY@tc##1{\textcolor[rgb]{0.00,0.50,0.00}{##1}}}
\expandafter\def\csname PY@tok@nf\endcsname{\def\PY@tc##1{\textcolor[rgb]{0.00,0.00,1.00}{##1}}}
\expandafter\def\csname PY@tok@nc\endcsname{\let\PY@bf=\textbf\def\PY@tc##1{\textcolor[rgb]{0.00,0.00,1.00}{##1}}}
\expandafter\def\csname PY@tok@nn\endcsname{\let\PY@bf=\textbf\def\PY@tc##1{\textcolor[rgb]{0.00,0.00,1.00}{##1}}}
\expandafter\def\csname PY@tok@ne\endcsname{\let\PY@bf=\textbf\def\PY@tc##1{\textcolor[rgb]{0.82,0.25,0.23}{##1}}}
\expandafter\def\csname PY@tok@nv\endcsname{\def\PY@tc##1{\textcolor[rgb]{0.10,0.09,0.49}{##1}}}
\expandafter\def\csname PY@tok@no\endcsname{\def\PY@tc##1{\textcolor[rgb]{0.53,0.00,0.00}{##1}}}
\expandafter\def\csname PY@tok@nl\endcsname{\def\PY@tc##1{\textcolor[rgb]{0.63,0.63,0.00}{##1}}}
\expandafter\def\csname PY@tok@ni\endcsname{\let\PY@bf=\textbf\def\PY@tc##1{\textcolor[rgb]{0.60,0.60,0.60}{##1}}}
\expandafter\def\csname PY@tok@na\endcsname{\def\PY@tc##1{\textcolor[rgb]{0.49,0.56,0.16}{##1}}}
\expandafter\def\csname PY@tok@nt\endcsname{\let\PY@bf=\textbf\def\PY@tc##1{\textcolor[rgb]{0.00,0.50,0.00}{##1}}}
\expandafter\def\csname PY@tok@nd\endcsname{\def\PY@tc##1{\textcolor[rgb]{0.67,0.13,1.00}{##1}}}
\expandafter\def\csname PY@tok@s\endcsname{\def\PY@tc##1{\textcolor[rgb]{0.73,0.13,0.13}{##1}}}
\expandafter\def\csname PY@tok@sd\endcsname{\let\PY@it=\textit\def\PY@tc##1{\textcolor[rgb]{0.73,0.13,0.13}{##1}}}
\expandafter\def\csname PY@tok@si\endcsname{\let\PY@bf=\textbf\def\PY@tc##1{\textcolor[rgb]{0.73,0.40,0.53}{##1}}}
\expandafter\def\csname PY@tok@se\endcsname{\let\PY@bf=\textbf\def\PY@tc##1{\textcolor[rgb]{0.73,0.40,0.13}{##1}}}
\expandafter\def\csname PY@tok@sr\endcsname{\def\PY@tc##1{\textcolor[rgb]{0.73,0.40,0.53}{##1}}}
\expandafter\def\csname PY@tok@ss\endcsname{\def\PY@tc##1{\textcolor[rgb]{0.10,0.09,0.49}{##1}}}
\expandafter\def\csname PY@tok@sx\endcsname{\def\PY@tc##1{\textcolor[rgb]{0.00,0.50,0.00}{##1}}}
\expandafter\def\csname PY@tok@m\endcsname{\def\PY@tc##1{\textcolor[rgb]{0.40,0.40,0.40}{##1}}}
\expandafter\def\csname PY@tok@gh\endcsname{\let\PY@bf=\textbf\def\PY@tc##1{\textcolor[rgb]{0.00,0.00,0.50}{##1}}}
\expandafter\def\csname PY@tok@gu\endcsname{\let\PY@bf=\textbf\def\PY@tc##1{\textcolor[rgb]{0.50,0.00,0.50}{##1}}}
\expandafter\def\csname PY@tok@gd\endcsname{\def\PY@tc##1{\textcolor[rgb]{0.63,0.00,0.00}{##1}}}
\expandafter\def\csname PY@tok@gi\endcsname{\def\PY@tc##1{\textcolor[rgb]{0.00,0.63,0.00}{##1}}}
\expandafter\def\csname PY@tok@gr\endcsname{\def\PY@tc##1{\textcolor[rgb]{1.00,0.00,0.00}{##1}}}
\expandafter\def\csname PY@tok@ge\endcsname{\let\PY@it=\textit}
\expandafter\def\csname PY@tok@gs\endcsname{\let\PY@bf=\textbf}
\expandafter\def\csname PY@tok@gp\endcsname{\let\PY@bf=\textbf\def\PY@tc##1{\textcolor[rgb]{0.00,0.00,0.50}{##1}}}
\expandafter\def\csname PY@tok@go\endcsname{\def\PY@tc##1{\textcolor[rgb]{0.53,0.53,0.53}{##1}}}
\expandafter\def\csname PY@tok@gt\endcsname{\def\PY@tc##1{\textcolor[rgb]{0.00,0.27,0.87}{##1}}}
\expandafter\def\csname PY@tok@err\endcsname{\def\PY@bc##1{\setlength{\fboxsep}{0pt}\fcolorbox[rgb]{1.00,0.00,0.00}{1,1,1}{\strut ##1}}}
\expandafter\def\csname PY@tok@kc\endcsname{\let\PY@bf=\textbf\def\PY@tc##1{\textcolor[rgb]{0.00,0.50,0.00}{##1}}}
\expandafter\def\csname PY@tok@kd\endcsname{\let\PY@bf=\textbf\def\PY@tc##1{\textcolor[rgb]{0.00,0.50,0.00}{##1}}}
\expandafter\def\csname PY@tok@kn\endcsname{\let\PY@bf=\textbf\def\PY@tc##1{\textcolor[rgb]{0.00,0.50,0.00}{##1}}}
\expandafter\def\csname PY@tok@kr\endcsname{\let\PY@bf=\textbf\def\PY@tc##1{\textcolor[rgb]{0.00,0.50,0.00}{##1}}}
\expandafter\def\csname PY@tok@bp\endcsname{\def\PY@tc##1{\textcolor[rgb]{0.00,0.50,0.00}{##1}}}
\expandafter\def\csname PY@tok@fm\endcsname{\def\PY@tc##1{\textcolor[rgb]{0.00,0.00,1.00}{##1}}}
\expandafter\def\csname PY@tok@vc\endcsname{\def\PY@tc##1{\textcolor[rgb]{0.10,0.09,0.49}{##1}}}
\expandafter\def\csname PY@tok@vg\endcsname{\def\PY@tc##1{\textcolor[rgb]{0.10,0.09,0.49}{##1}}}
\expandafter\def\csname PY@tok@vi\endcsname{\def\PY@tc##1{\textcolor[rgb]{0.10,0.09,0.49}{##1}}}
\expandafter\def\csname PY@tok@vm\endcsname{\def\PY@tc##1{\textcolor[rgb]{0.10,0.09,0.49}{##1}}}
\expandafter\def\csname PY@tok@sa\endcsname{\def\PY@tc##1{\textcolor[rgb]{0.73,0.13,0.13}{##1}}}
\expandafter\def\csname PY@tok@sb\endcsname{\def\PY@tc##1{\textcolor[rgb]{0.73,0.13,0.13}{##1}}}
\expandafter\def\csname PY@tok@sc\endcsname{\def\PY@tc##1{\textcolor[rgb]{0.73,0.13,0.13}{##1}}}
\expandafter\def\csname PY@tok@dl\endcsname{\def\PY@tc##1{\textcolor[rgb]{0.73,0.13,0.13}{##1}}}
\expandafter\def\csname PY@tok@s2\endcsname{\def\PY@tc##1{\textcolor[rgb]{0.73,0.13,0.13}{##1}}}
\expandafter\def\csname PY@tok@sh\endcsname{\def\PY@tc##1{\textcolor[rgb]{0.73,0.13,0.13}{##1}}}
\expandafter\def\csname PY@tok@s1\endcsname{\def\PY@tc##1{\textcolor[rgb]{0.73,0.13,0.13}{##1}}}
\expandafter\def\csname PY@tok@mb\endcsname{\def\PY@tc##1{\textcolor[rgb]{0.40,0.40,0.40}{##1}}}
\expandafter\def\csname PY@tok@mf\endcsname{\def\PY@tc##1{\textcolor[rgb]{0.40,0.40,0.40}{##1}}}
\expandafter\def\csname PY@tok@mh\endcsname{\def\PY@tc##1{\textcolor[rgb]{0.40,0.40,0.40}{##1}}}
\expandafter\def\csname PY@tok@mi\endcsname{\def\PY@tc##1{\textcolor[rgb]{0.40,0.40,0.40}{##1}}}
\expandafter\def\csname PY@tok@il\endcsname{\def\PY@tc##1{\textcolor[rgb]{0.40,0.40,0.40}{##1}}}
\expandafter\def\csname PY@tok@mo\endcsname{\def\PY@tc##1{\textcolor[rgb]{0.40,0.40,0.40}{##1}}}
\expandafter\def\csname PY@tok@ch\endcsname{\let\PY@it=\textit\def\PY@tc##1{\textcolor[rgb]{0.25,0.50,0.50}{##1}}}
\expandafter\def\csname PY@tok@cm\endcsname{\let\PY@it=\textit\def\PY@tc##1{\textcolor[rgb]{0.25,0.50,0.50}{##1}}}
\expandafter\def\csname PY@tok@cpf\endcsname{\let\PY@it=\textit\def\PY@tc##1{\textcolor[rgb]{0.25,0.50,0.50}{##1}}}
\expandafter\def\csname PY@tok@c1\endcsname{\let\PY@it=\textit\def\PY@tc##1{\textcolor[rgb]{0.25,0.50,0.50}{##1}}}
\expandafter\def\csname PY@tok@cs\endcsname{\let\PY@it=\textit\def\PY@tc##1{\textcolor[rgb]{0.25,0.50,0.50}{##1}}}

\def\PYZbs{\char`\\}
\def\PYZus{\char`\_}
\def\PYZob{\char`\{}
\def\PYZcb{\char`\}}
\def\PYZca{\char`\^}
\def\PYZam{\char`\&}
\def\PYZlt{\char`\<}
\def\PYZgt{\char`\>}
\def\PYZsh{\char`\#}
\def\PYZpc{\char`\%}
\def\PYZdl{\char`\$}
\def\PYZhy{\char`\-}
\def\PYZsq{\char`\'}
\def\PYZdq{\char`\"}
\def\PYZti{\char`\~}
% for compatibility with earlier versions
\def\PYZat{@}
\def\PYZlb{[}
\def\PYZrb{]}
\makeatother


    % Exact colors from NB
    \definecolor{incolor}{rgb}{0.0, 0.0, 0.5}
    \definecolor{outcolor}{rgb}{0.545, 0.0, 0.0}



    
    % Prevent overflowing lines due to hard-to-break entities
    \sloppy 
    % Setup hyperref package
    \hypersetup{
      breaklinks=true,  % so long urls are correctly broken across lines
      colorlinks=true,
      urlcolor=urlcolor,
      linkcolor=linkcolor,
      citecolor=citecolor,
      }
    % Slightly bigger margins than the latex defaults
    
    \geometry{verbose,tmargin=1in,bmargin=1in,lmargin=1in,rmargin=1in}
    
    

    \begin{document}
    
    
    \maketitle
    
    

    
    \hypertarget{use-py-in-calculus}{%
\section{Use PY in Calculus}\label{use-py-in-calculus}}

\hypertarget{what-is-function}{%
\subsection{What is Function}\label{what-is-function}}

我們可以將函數(functions)看作一台機器,當我們向這台機器輸入「x」時,它將輸出「f(x)」

這台機器所能接受的所有輸入的集合被稱為定義域(domian),其所有可能的輸出的集合被稱為值域(range)。函數的定義域和值域都十分重要,當我們知道一個函數的定義域,就不會將不合適的\texttt{x}扔給這個函數;知道了定義域就可以判斷一個值是否可能是這個函數所輸出的。

\hypertarget{ux591aux9805ux5f0fpolynomials}{%
\subsubsection{多項式(polynomials):}\label{ux591aux9805ux5f0fpolynomials}}

\(f(x) = x^3 - 5^2 +9\) 因為這是個三次函數,當 \(x\rightarrow \infty\)
時,\(f(x) \rightarrow -\infty\),當 \(x\rightarrow \infty\)
時,\(f(x) \rightarrow \infty\)
因此,這個函數的定義域和值域都屬於實數集\(R\)。

    \begin{Verbatim}[commandchars=\\\{\}]
{\color{incolor}In [{\color{incolor}1}]:} \PY{k}{def} \PY{n+nf}{f}\PY{p}{(}\PY{n}{x}\PY{p}{)}\PY{p}{:}
            \PY{k}{return} \PY{n}{x}\PY{o}{*}\PY{o}{*}\PY{l+m+mi}{3} \PY{o}{\PYZhy{}} \PY{l+m+mi}{5}\PY{o}{*}\PY{n}{x}\PY{o}{*}\PY{o}{*}\PY{l+m+mi}{2} \PY{o}{+} \PY{l+m+mi}{9}
        
        \PY{n+nb}{print}\PY{p}{(}\PY{n}{f}\PY{p}{(}\PY{l+m+mi}{1}\PY{p}{)}\PY{p}{,} \PY{n}{f}\PY{p}{(}\PY{l+m+mi}{2}\PY{p}{)}\PY{p}{)}
\end{Verbatim}


    \begin{Verbatim}[commandchars=\\\{\}]
5 -3

    \end{Verbatim}

    通常,我們會繪製函數圖像來幫助我們來理解函數的變化

    \begin{Verbatim}[commandchars=\\\{\}]
{\color{incolor}In [{\color{incolor}3}]:} \PY{k+kn}{import} \PY{n+nn}{numpy} \PY{k}{as} \PY{n+nn}{np}
        \PY{k+kn}{import} \PY{n+nn}{matplotlib}\PY{n+nn}{.}\PY{n+nn}{pyplot} \PY{k}{as} \PY{n+nn}{plt}
        \PY{n}{x} \PY{o}{=} \PY{n}{np}\PY{o}{.}\PY{n}{linspace}\PY{p}{(}\PY{o}{\PYZhy{}}\PY{l+m+mi}{10}\PY{p}{,}\PY{l+m+mi}{10}\PY{p}{,}\PY{n}{num} \PY{o}{=} \PY{l+m+mi}{1000}\PY{p}{)}
        \PY{n}{y} \PY{o}{=} \PY{n}{f}\PY{p}{(}\PY{n}{x}\PY{p}{)} 
        \PY{n}{plt}\PY{o}{.}\PY{n}{plot}\PY{p}{(}\PY{n}{x}\PY{p}{,}\PY{n}{y}\PY{p}{)}
\end{Verbatim}


\begin{Verbatim}[commandchars=\\\{\}]
{\color{outcolor}Out[{\color{outcolor}3}]:} [<matplotlib.lines.Line2D at 0x6f6a6270>]
\end{Verbatim}
            
    \begin{center}
    \adjustimage{max size={0.9\linewidth}{0.9\paperheight}}{Use_PY_in_Calculus_files/Use_PY_in_Calculus_3_1.png}
    \end{center}
    { \hspace*{\fill} \\}
    
    \hypertarget{ux6307ux6578ux51fdux6578exponential-functions}{%
\subsubsection{指數函數(Exponential
Functions)}\label{ux6307ux6578ux51fdux6578exponential-functions}}

\(exp(x) = e^x\) domain is \((-\infty,\infty)\),range is
\((0,\infty)\)。在 py 中,我們可以利用歐拉常數 \(e\) 定義指數函數:

    \begin{Verbatim}[commandchars=\\\{\}]
{\color{incolor}In [{\color{incolor}4}]:} \PY{k}{def} \PY{n+nf}{exp}\PY{p}{(}\PY{n}{x}\PY{p}{)}\PY{p}{:}
            \PY{k}{return} \PY{n}{np}\PY{o}{.}\PY{n}{e}\PY{o}{*}\PY{o}{*}\PY{n}{x}
        
        \PY{n+nb}{print}\PY{p}{(}\PY{l+s+s2}{\PYZdq{}}\PY{l+s+s2}{exp(2) = e\PYZca{}2 = }\PY{l+s+s2}{\PYZdq{}}\PY{p}{,}\PY{n}{exp}\PY{p}{(}\PY{l+m+mi}{2}\PY{p}{)}\PY{p}{)}
\end{Verbatim}


    \begin{Verbatim}[commandchars=\\\{\}]
exp(2) = e\^{}2 =  7.3890560989306495

    \end{Verbatim}

    或者可以使用 \texttt{numpy} 自帶的指數函數:\texttt{np.e**x}

    \begin{Verbatim}[commandchars=\\\{\}]
{\color{incolor}In [{\color{incolor}5}]:} \PY{k}{def} \PY{n+nf}{eexp}\PY{p}{(}\PY{n}{x}\PY{p}{)}\PY{p}{:}
            \PY{k}{return} \PY{n}{np}\PY{o}{.}\PY{n}{e}\PY{o}{*}\PY{o}{*}\PY{p}{(}\PY{n}{x}\PY{p}{)}
        
        \PY{n+nb}{print}\PY{p}{(}\PY{l+s+s2}{\PYZdq{}}\PY{l+s+s2}{exp(2) = e\PYZca{}2 = }\PY{l+s+s2}{\PYZdq{}}\PY{p}{,}\PY{n}{eexp}\PY{p}{(}\PY{l+m+mi}{2}\PY{p}{)}\PY{p}{)}
\end{Verbatim}


    \begin{Verbatim}[commandchars=\\\{\}]
exp(2) = e\^{}2 =  7.3890560989306495

    \end{Verbatim}

    \begin{Verbatim}[commandchars=\\\{\}]
{\color{incolor}In [{\color{incolor}57}]:} \PY{n}{plt}\PY{o}{.}\PY{n}{plot}\PY{p}{(}\PY{n}{x}\PY{p}{,}\PY{n}{exp}\PY{p}{(}\PY{n}{x}\PY{p}{)}\PY{p}{)}
\end{Verbatim}


\begin{Verbatim}[commandchars=\\\{\}]
{\color{outcolor}Out[{\color{outcolor}57}]:} [<matplotlib.lines.Line2D at 0x1137944a8>]
\end{Verbatim}
            
    \begin{center}
    \adjustimage{max size={0.9\linewidth}{0.9\paperheight}}{Use_PY_in_Calculus_files/Use_PY_in_Calculus_8_1.png}
    \end{center}
    { \hspace*{\fill} \\}
    
    當然,數學課就會講的更加深入\(e^x\)的定義式應該長成這樣:\(\begin{align*}\sum_{k=0}^{\infty}\frac{x^k}{k!}\end{align*}\)
至於為什麼他會長成這樣,會在後面提及。
這個式子應該怎麼在\texttt{python}中實現呢?

    \begin{Verbatim}[commandchars=\\\{\}]
{\color{incolor}In [{\color{incolor}58}]:} \PY{k}{def} \PY{n+nf}{eeexp}\PY{p}{(}\PY{n}{x}\PY{p}{)}\PY{p}{:}
             \PY{n+nb}{sum} \PY{o}{=} \PY{l+m+mi}{0}
             \PY{k}{for} \PY{n}{k} \PY{o+ow}{in} \PY{n+nb}{range}\PY{p}{(}\PY{l+m+mi}{100}\PY{p}{)}\PY{p}{:}
                 \PY{n+nb}{sum} \PY{o}{+}\PY{o}{=} \PY{n+nb}{float}\PY{p}{(}\PY{n}{x}\PY{o}{*}\PY{o}{*}\PY{n}{k}\PY{p}{)}\PY{o}{/}\PY{n}{np}\PY{o}{.}\PY{n}{math}\PY{o}{.}\PY{n}{factorial}\PY{p}{(}\PY{n}{k}\PY{p}{)}
             \PY{k}{return} \PY{n+nb}{sum}
         
         \PY{n+nb}{print}\PY{p}{(}\PY{l+s+s2}{\PYZdq{}}\PY{l+s+s2}{exp(2) = e\PYZca{}2 = }\PY{l+s+s2}{\PYZdq{}}\PY{p}{,}\PY{n}{eeexp}\PY{p}{(}\PY{l+m+mi}{2}\PY{p}{)}\PY{p}{)}
\end{Verbatim}


    \begin{Verbatim}[commandchars=\\\{\}]
exp(2) = e\^{}2 =  7.389056098930649

    \end{Verbatim}

    \hypertarget{ux5c0dux6578ux51fdux6578logarithmic-function}{%
\subsubsection{對數函數(Logarithmic
Function)}\label{ux5c0dux6578ux51fdux6578logarithmic-function}}

\(log_e(x) = ln(x)\) \emph{高中教的 \(ln(x)\)
在大學和以後的生活中經常會被寫成 \(log(x)\)}
對數函數其實就是指數函數的反函數,即,定義域為\((0,\infty)\),值域為\((-\infty,\infty)\)。
\texttt{numpy} 為我們提供了以\(2,e,10\) 為底數的對數函數:

    \begin{Verbatim}[commandchars=\\\{\}]
{\color{incolor}In [{\color{incolor}59}]:} \PY{n}{x} \PY{o}{=} \PY{n}{np}\PY{o}{.}\PY{n}{linspace}\PY{p}{(}\PY{l+m+mi}{1}\PY{p}{,}\PY{l+m+mi}{10}\PY{p}{,}\PY{l+m+mi}{1000}\PY{p}{,}\PY{n}{endpoint} \PY{o}{=} \PY{k+kc}{False}\PY{p}{)}
         \PY{n}{y1} \PY{o}{=} \PY{n}{np}\PY{o}{.}\PY{n}{log2}\PY{p}{(}\PY{n}{x}\PY{p}{)}
         \PY{n}{y2} \PY{o}{=} \PY{n}{np}\PY{o}{.}\PY{n}{log}\PY{p}{(}\PY{n}{x}\PY{p}{)}
         \PY{n}{y3} \PY{o}{=} \PY{n}{np}\PY{o}{.}\PY{n}{log10}\PY{p}{(}\PY{n}{x}\PY{p}{)}
         \PY{n}{plt}\PY{o}{.}\PY{n}{plot}\PY{p}{(}\PY{n}{x}\PY{p}{,}\PY{n}{y1}\PY{p}{,}\PY{l+s+s1}{\PYZsq{}}\PY{l+s+s1}{red}\PY{l+s+s1}{\PYZsq{}}\PY{p}{,}\PY{n}{x}\PY{p}{,}\PY{n}{y2}\PY{p}{,}\PY{l+s+s1}{\PYZsq{}}\PY{l+s+s1}{yellow}\PY{l+s+s1}{\PYZsq{}}\PY{p}{,}\PY{n}{x}\PY{p}{,}\PY{n}{y3}\PY{p}{,}\PY{l+s+s1}{\PYZsq{}}\PY{l+s+s1}{blue}\PY{l+s+s1}{\PYZsq{}}\PY{p}{)}
\end{Verbatim}


\begin{Verbatim}[commandchars=\\\{\}]
{\color{outcolor}Out[{\color{outcolor}59}]:} [<matplotlib.lines.Line2D at 0x1138561d0>,
          <matplotlib.lines.Line2D at 0x1138567f0>,
          <matplotlib.lines.Line2D at 0x113856ba8>]
\end{Verbatim}
            
    \begin{center}
    \adjustimage{max size={0.9\linewidth}{0.9\paperheight}}{Use_PY_in_Calculus_files/Use_PY_in_Calculus_12_1.png}
    \end{center}
    { \hspace*{\fill} \\}
    
    \hypertarget{ux4e09ux89d2ux51fdux6578trigonometric-functions}{%
\subsubsection{三角函數(Trigonometric
functions)}\label{ux4e09ux89d2ux51fdux6578trigonometric-functions}}

三角函數是常見的關於角的函數,三角函數在研究三角形和園等集合形狀的性質時,有很重要的作用,也是研究週期性現象的基礎工具;常見的三角函數有:正弦(sin),餘弦(cos)和正切(tan),當然,以後還會用到如餘切,正割,餘割等。

    \begin{Verbatim}[commandchars=\\\{\}]
{\color{incolor}In [{\color{incolor}60}]:} \PY{n}{x} \PY{o}{=} \PY{n}{np}\PY{o}{.}\PY{n}{linspace}\PY{p}{(}\PY{o}{\PYZhy{}}\PY{l+m+mi}{10}\PY{p}{,} \PY{l+m+mi}{10}\PY{p}{,} \PY{l+m+mi}{10000}\PY{p}{)}  
         \PY{n}{a} \PY{o}{=} \PY{n}{np}\PY{o}{.}\PY{n}{sin}\PY{p}{(}\PY{n}{x}\PY{p}{)}  
         \PY{n}{b} \PY{o}{=} \PY{n}{np}\PY{o}{.}\PY{n}{cos}\PY{p}{(}\PY{n}{x}\PY{p}{)}  
         \PY{n}{c} \PY{o}{=} \PY{n}{np}\PY{o}{.}\PY{n}{tan}\PY{p}{(}\PY{n}{x}\PY{p}{)}  
         \PY{c+c1}{\PYZsh{} d = np.log(x)  }
           
         \PY{n}{plt}\PY{o}{.}\PY{n}{figure}\PY{p}{(}\PY{n}{figsize}\PY{o}{=}\PY{p}{(}\PY{l+m+mi}{8}\PY{p}{,}\PY{l+m+mi}{4}\PY{p}{)}\PY{p}{)}  
         \PY{n}{plt}\PY{o}{.}\PY{n}{plot}\PY{p}{(}\PY{n}{x}\PY{p}{,}\PY{n}{a}\PY{p}{,}\PY{n}{label}\PY{o}{=}\PY{l+s+s1}{\PYZsq{}}\PY{l+s+s1}{\PYZdl{}sin(x)\PYZdl{}}\PY{l+s+s1}{\PYZsq{}}\PY{p}{,}\PY{n}{color}\PY{o}{=}\PY{l+s+s1}{\PYZsq{}}\PY{l+s+s1}{green}\PY{l+s+s1}{\PYZsq{}}\PY{p}{,}\PY{n}{linewidth}\PY{o}{=}\PY{l+m+mf}{0.5}\PY{p}{)}  
         \PY{n}{plt}\PY{o}{.}\PY{n}{plot}\PY{p}{(}\PY{n}{x}\PY{p}{,}\PY{n}{b}\PY{p}{,}\PY{n}{label}\PY{o}{=}\PY{l+s+s1}{\PYZsq{}}\PY{l+s+s1}{\PYZdl{}cos(x)\PYZdl{}}\PY{l+s+s1}{\PYZsq{}}\PY{p}{,}\PY{n}{color}\PY{o}{=}\PY{l+s+s1}{\PYZsq{}}\PY{l+s+s1}{red}\PY{l+s+s1}{\PYZsq{}}\PY{p}{,}\PY{n}{linewidth}\PY{o}{=}\PY{l+m+mf}{0.5}\PY{p}{)}  
         \PY{n}{plt}\PY{o}{.}\PY{n}{plot}\PY{p}{(}\PY{n}{x}\PY{p}{,}\PY{n}{c}\PY{p}{,}\PY{n}{label}\PY{o}{=}\PY{l+s+s1}{\PYZsq{}}\PY{l+s+s1}{\PYZdl{}tan(x)\PYZdl{}}\PY{l+s+s1}{\PYZsq{}}\PY{p}{,}\PY{n}{color}\PY{o}{=}\PY{l+s+s1}{\PYZsq{}}\PY{l+s+s1}{blue}\PY{l+s+s1}{\PYZsq{}}\PY{p}{,}\PY{n}{linewidth}\PY{o}{=}\PY{l+m+mf}{0.5}\PY{p}{)}  
         \PY{c+c1}{\PYZsh{} plt.plot(x,d,label=\PYZsq{}\PYZdl{}log(x)\PYZdl{}\PYZsq{},color=\PYZsq{}grey\PYZsq{},linewidth=0.5)  }
           
         \PY{n}{plt}\PY{o}{.}\PY{n}{xlabel}\PY{p}{(}\PY{l+s+s1}{\PYZsq{}}\PY{l+s+s1}{Time(s)}\PY{l+s+s1}{\PYZsq{}}\PY{p}{)}  
         \PY{n}{plt}\PY{o}{.}\PY{n}{ylabel}\PY{p}{(}\PY{l+s+s1}{\PYZsq{}}\PY{l+s+s1}{Volt}\PY{l+s+s1}{\PYZsq{}}\PY{p}{)}  
         \PY{n}{plt}\PY{o}{.}\PY{n}{title}\PY{p}{(}\PY{l+s+s1}{\PYZsq{}}\PY{l+s+s1}{PyPlot}\PY{l+s+s1}{\PYZsq{}}\PY{p}{)}  
         \PY{n}{plt}\PY{o}{.}\PY{n}{xlim}\PY{p}{(}\PY{l+m+mi}{0}\PY{p}{,}\PY{l+m+mi}{10}\PY{p}{)}  
         \PY{n}{plt}\PY{o}{.}\PY{n}{ylim}\PY{p}{(}\PY{o}{\PYZhy{}}\PY{l+m+mi}{5}\PY{p}{,}\PY{l+m+mi}{5}\PY{p}{)}  
         \PY{n}{plt}\PY{o}{.}\PY{n}{legend}\PY{p}{(}\PY{p}{)}  
         \PY{n}{plt}\PY{o}{.}\PY{n}{show}\PY{p}{(}\PY{p}{)}  
\end{Verbatim}


    \begin{center}
    \adjustimage{max size={0.9\linewidth}{0.9\paperheight}}{Use_PY_in_Calculus_files/Use_PY_in_Calculus_14_0.png}
    \end{center}
    { \hspace*{\fill} \\}
    
    \hypertarget{ux8907ux5408ux51fdux6578composition}{%
\subsection{複合函數(composition)}\label{ux8907ux5408ux51fdux6578composition}}

函數 \(f\) 和 \(g\) 複合,\(f \circ g = f(g(x))\),可以理解為先把\(x\)
輸入給 \(g\) 函數,獲得 \(g(x)\) 後在輸入函數 \(f\)
中,最後得出:\(f(g(x))\) * 幾個函數符合後仍然為一個函數 *
任何函數都可以看成若干個函數的複合形式 * \(f\circ g(x)\) 的定義域與
\(g(x)\) 相同,但是值域不一定與 \(f(x)\) 相同

例:\(f(x) = x^2, g(x) = x^2 + x, h(x) = x^4 +2x^2\cdot x + x^2\)

    \begin{Verbatim}[commandchars=\\\{\}]
{\color{incolor}In [{\color{incolor}61}]:} \PY{k}{def} \PY{n+nf}{f}\PY{p}{(}\PY{n}{x}\PY{p}{)}\PY{p}{:}
             \PY{k}{return} \PY{n}{x}\PY{o}{*}\PY{o}{*}\PY{l+m+mi}{2}
         \PY{k}{def} \PY{n+nf}{g}\PY{p}{(}\PY{n}{x}\PY{p}{)}\PY{p}{:}
             \PY{k}{return} \PY{n}{x}\PY{o}{*}\PY{o}{*}\PY{l+m+mi}{2}\PY{o}{+}\PY{n}{x}
         \PY{k}{def} \PY{n+nf}{h}\PY{p}{(}\PY{n}{x}\PY{p}{)}\PY{p}{:}
             \PY{k}{return} \PY{n}{f}\PY{p}{(}\PY{n}{g}\PY{p}{(}\PY{n}{x}\PY{p}{)}\PY{p}{)}
         
         \PY{n+nb}{print}\PY{p}{(}\PY{l+s+s2}{\PYZdq{}}\PY{l+s+s2}{f(1) equals}\PY{l+s+s2}{\PYZdq{}}\PY{p}{,}\PY{n}{f}\PY{p}{(}\PY{l+m+mi}{1}\PY{p}{)}\PY{p}{,}\PY{l+s+s2}{\PYZdq{}}\PY{l+s+s2}{g(1) equals}\PY{l+s+s2}{\PYZdq{}}\PY{p}{,}\PY{n}{g}\PY{p}{(}\PY{l+m+mi}{1}\PY{p}{)}\PY{p}{,}\PY{l+s+s2}{\PYZdq{}}\PY{l+s+s2}{h(1) equals}\PY{l+s+s2}{\PYZdq{}}\PY{p}{,}\PY{n}{h}\PY{p}{(}\PY{l+m+mi}{1}\PY{p}{)}\PY{p}{)}
         
         \PY{n}{x} \PY{o}{=} \PY{n}{np}\PY{o}{.}\PY{n}{array}\PY{p}{(}\PY{n+nb}{range}\PY{p}{(}\PY{o}{\PYZhy{}}\PY{l+m+mi}{10}\PY{p}{,}\PY{l+m+mi}{10}\PY{p}{)}\PY{p}{)}
         \PY{n}{y} \PY{o}{=} \PY{n}{np}\PY{o}{.}\PY{n}{array}\PY{p}{(}\PY{p}{[}\PY{n}{h}\PY{p}{(}\PY{n}{i}\PY{p}{)} \PY{k}{for} \PY{n}{i} \PY{o+ow}{in} \PY{n}{x}\PY{p}{]}\PY{p}{)}
         \PY{n}{plt}\PY{o}{.}\PY{n}{scatter}\PY{p}{(}\PY{n}{x}\PY{p}{,}\PY{n}{y}\PY{p}{,}\PY{p}{)}
\end{Verbatim}


    \begin{Verbatim}[commandchars=\\\{\}]
f(1) equals 1 g(1) equals 2 h(1) equals 4

    \end{Verbatim}

\begin{Verbatim}[commandchars=\\\{\}]
{\color{outcolor}Out[{\color{outcolor}61}]:} <matplotlib.collections.PathCollection at 0x114391208>
\end{Verbatim}
            
    \begin{center}
    \adjustimage{max size={0.9\linewidth}{0.9\paperheight}}{Use_PY_in_Calculus_files/Use_PY_in_Calculus_16_2.png}
    \end{center}
    { \hspace*{\fill} \\}
    
    \hypertarget{ux9006ux51fdux6578inverse-function}{%
\subsubsection{逆函數(Inverse
Function)}\label{ux9006ux51fdux6578inverse-function}}

給定一個函數\(f\),其逆函數 \(f^{-1}\) 是一個與 \(f\) 進行複合後
\(f\circ f^{-1}(x) = x\) 的特殊函數 函數與其反函數圖像一定是關於
\(y = x\) 對稱的

    \begin{Verbatim}[commandchars=\\\{\}]
{\color{incolor}In [{\color{incolor}62}]:} \PY{k}{def} \PY{n+nf}{w}\PY{p}{(}\PY{n}{x}\PY{p}{)}\PY{p}{:}
             \PY{k}{return} \PY{n}{x}\PY{o}{*}\PY{o}{*}\PY{l+m+mi}{2}
         \PY{k}{def} \PY{n+nf}{inv}\PY{p}{(}\PY{n}{x}\PY{p}{)}\PY{p}{:}
             \PY{k}{return} \PY{n}{np}\PY{o}{.}\PY{n}{sqrt}\PY{p}{(}\PY{n}{x}\PY{p}{)}
         \PY{n}{x} \PY{o}{=} \PY{n}{np}\PY{o}{.}\PY{n}{linspace}\PY{p}{(}\PY{l+m+mi}{0}\PY{p}{,}\PY{l+m+mi}{2}\PY{p}{,}\PY{l+m+mi}{100}\PY{p}{)}
         \PY{n}{plt}\PY{o}{.}\PY{n}{plot}\PY{p}{(}\PY{n}{x}\PY{p}{,}\PY{n}{w}\PY{p}{(}\PY{n}{x}\PY{p}{)}\PY{p}{,}\PY{l+s+s1}{\PYZsq{}}\PY{l+s+s1}{r}\PY{l+s+s1}{\PYZsq{}}\PY{p}{,}\PY{n}{x}\PY{p}{,}\PY{n}{inv}\PY{p}{(}\PY{n}{x}\PY{p}{)}\PY{p}{,}\PY{l+s+s1}{\PYZsq{}}\PY{l+s+s1}{b}\PY{l+s+s1}{\PYZsq{}}\PY{p}{,}\PY{n}{x}\PY{p}{,}\PY{n}{x}\PY{p}{,}\PY{l+s+s1}{\PYZsq{}}\PY{l+s+s1}{g\PYZhy{}.}\PY{l+s+s1}{\PYZsq{}}\PY{p}{)}
\end{Verbatim}


\begin{Verbatim}[commandchars=\\\{\}]
{\color{outcolor}Out[{\color{outcolor}62}]:} [<matplotlib.lines.Line2D at 0x1138ce630>,
          <matplotlib.lines.Line2D at 0x1138cea90>,
          <matplotlib.lines.Line2D at 0x1138eb7f0>]
\end{Verbatim}
            
    \begin{center}
    \adjustimage{max size={0.9\linewidth}{0.9\paperheight}}{Use_PY_in_Calculus_files/Use_PY_in_Calculus_18_1.png}
    \end{center}
    { \hspace*{\fill} \\}
    
    \hypertarget{ux9ad8ux968eux51fdux6578higher-order-function}{%
\subsubsection{高階函數(Higher Order
Function)}\label{ux9ad8ux968eux51fdux6578higher-order-function}}

我们可以不局限于将数值作为函数的输入和输出,函数本身也可以作为输入和输出,

在給出例子之前,插一段話: 這裡介紹一下在
\texttt{python}中十分重要的一個表達式:\texttt{lambda},\texttt{lambda}本身就是一行函數,他們在其他語言中被稱為匿名函數,如果你不想在程序中對一個函數使用兩次,你也許會想到用
\texttt{lambda} 表達式,他們和普通函數完全一樣。 原型: \texttt{lambda}
參數:操作(參數)

    \begin{Verbatim}[commandchars=\\\{\}]
{\color{incolor}In [{\color{incolor}63}]:} \PY{n}{add} \PY{o}{=} \PY{k}{lambda} \PY{n}{x}\PY{p}{,}\PY{n}{y}\PY{p}{:} \PY{n}{x}\PY{o}{+}\PY{n}{y}
         
         \PY{n+nb}{print}\PY{p}{(}\PY{n}{add}\PY{p}{(}\PY{l+m+mi}{3}\PY{p}{,}\PY{l+m+mi}{5}\PY{p}{)}\PY{p}{)}
\end{Verbatim}


    \begin{Verbatim}[commandchars=\\\{\}]
8

    \end{Verbatim}

    這裡,我們給出 高階函數 的例子:

    \begin{Verbatim}[commandchars=\\\{\}]
{\color{incolor}In [{\color{incolor}64}]:} \PY{k}{def} \PY{n+nf}{horizontal\PYZus{}shift}\PY{p}{(}\PY{n}{f}\PY{p}{,}\PY{n}{H}\PY{p}{)}\PY{p}{:}
             \PY{k}{return} \PY{k}{lambda} \PY{n}{x}\PY{p}{:} \PY{n}{f}\PY{p}{(}\PY{n}{x}\PY{o}{\PYZhy{}}\PY{n}{H}\PY{p}{)}
\end{Verbatim}


    上面定義的函數 \texttt{horizontal\_shift(f,H)}。接受的輸入是一個函數
\(f\) 和一個實數 \(H\),然後輸出一個新的函數,新函數是將 \(f\)
沿著水平方向平移了距離 \(H\) 以後得到的。

    \begin{Verbatim}[commandchars=\\\{\}]
{\color{incolor}In [{\color{incolor}65}]:} \PY{n}{x} \PY{o}{=} \PY{n}{np}\PY{o}{.}\PY{n}{linspace}\PY{p}{(}\PY{o}{\PYZhy{}}\PY{l+m+mi}{10}\PY{p}{,}\PY{l+m+mi}{10}\PY{p}{,}\PY{l+m+mi}{1000}\PY{p}{)}
         \PY{n}{shifted\PYZus{}g} \PY{o}{=} \PY{n}{horizontal\PYZus{}shift}\PY{p}{(}\PY{n}{g}\PY{p}{,}\PY{l+m+mi}{2}\PY{p}{)}
         \PY{n}{plt}\PY{o}{.}\PY{n}{plot}\PY{p}{(}\PY{n}{x}\PY{p}{,}\PY{n}{g}\PY{p}{(}\PY{n}{x}\PY{p}{)}\PY{p}{,}\PY{l+s+s1}{\PYZsq{}}\PY{l+s+s1}{b}\PY{l+s+s1}{\PYZsq{}}\PY{p}{,}\PY{n}{x}\PY{p}{,}\PY{n}{shifted\PYZus{}g}\PY{p}{(}\PY{n}{x}\PY{p}{)}\PY{p}{,}\PY{l+s+s1}{\PYZsq{}}\PY{l+s+s1}{r}\PY{l+s+s1}{\PYZsq{}}\PY{p}{)}
\end{Verbatim}


\begin{Verbatim}[commandchars=\\\{\}]
{\color{outcolor}Out[{\color{outcolor}65}]:} [<matplotlib.lines.Line2D at 0x113737278>,
          <matplotlib.lines.Line2D at 0x113737da0>]
\end{Verbatim}
            
    \begin{center}
    \adjustimage{max size={0.9\linewidth}{0.9\paperheight}}{Use_PY_in_Calculus_files/Use_PY_in_Calculus_24_1.png}
    \end{center}
    { \hspace*{\fill} \\}
    
    以高階函數的觀點去看,函數的複合就等於將兩個函數作為輸入給複合函數,然後由其產生一個新的函數作為輸出。所以複合函數又有了新的定義:

    \begin{Verbatim}[commandchars=\\\{\}]
{\color{incolor}In [{\color{incolor}66}]:} \PY{k}{def}  \PY{n+nf}{composite}\PY{p}{(}\PY{n}{f}\PY{p}{,}\PY{n}{g}\PY{p}{)}\PY{p}{:}
             \PY{k}{return} \PY{k}{lambda} \PY{n}{x}\PY{p}{:} \PY{n}{f}\PY{p}{(}\PY{n}{g}\PY{p}{(}\PY{n}{x}\PY{p}{)}\PY{p}{)}
         \PY{n}{h3} \PY{o}{=} \PY{n}{composite}\PY{p}{(}\PY{n}{f}\PY{p}{,}\PY{n}{g}\PY{p}{)}
         \PY{n+nb}{print} \PY{p}{(}\PY{n+nb}{sum} \PY{p}{(}\PY{n}{h}\PY{p}{(}\PY{n}{x}\PY{p}{)} \PY{o}{==} \PY{n}{h3}\PY{p}{(}\PY{n}{x}\PY{p}{)}\PY{p}{)} \PY{o}{==} \PY{n+nb}{len}\PY{p}{(}\PY{n}{x}\PY{p}{)}\PY{p}{)}
             
\end{Verbatim}


    \begin{Verbatim}[commandchars=\\\{\}]
True

    \end{Verbatim}

    \hypertarget{ux6b50ux62c9ux516cux5f0feulers-formula}{%
\subsection{歐拉公式(Euler's
Formula)}\label{ux6b50ux62c9ux516cux5f0feulers-formula}}

在前面給出了指數函數的多項式形式:\(e^x = 1 + \frac{x}{1!} + \frac{x^2}{2!} + \dots = \sum_{k = 0}^{\infty}\frac{x^k}{k!}\)
接下來,我們不僅不去解釋上面的式子是怎麼來的,而且還要喪心病狂地扔給讀者:
三角函數:
\(\begin{align*} &sin(x) = \frac{x}{1!}-\frac{x^3}{3!}+\frac{x^5}{5!}-\frac{x^7}{7!}\dots = \sum_{k=0}^{\infty}(-1)^k\frac{x^{(2k+1)}}{(2k+1)!} \\ &cos(x) = \frac{x^0}{0!}-\frac{x^2}{2!}+\frac{x^4}{4!}-\dots =\sum_{k=0}^{\infty}(-1)^k\frac{x^{2k}}{2k!}\end{align*}\)
在中學,我們曾經學過虛數 \texttt{i} (Imaginary
Number)的概念,這裡我們對其來源和意義暫不討論,只是簡單回顧一下其基本的運算規則:
\(i^0 = 1, i^1 = i, i^2 = -1 \dots\) 將 \(ix\)
帶入指數函數的公式中,得:
\(\begin{align*}e^{ix} &= \frac{(ix)^0}{0!} + \frac{(ix)^1}{1!} + \frac{(ix)^2}{2!} + \dots \\ &= \frac{i^0 x^0}{0!} + \frac{i^1 x^1}{1!} + \frac{i^2 x^2}{2!} + \dots \\ &= 1\frac{x^0}{0!} + i\frac{x^i}{1!} -1\frac{x^2}{2!} -i\frac{x^3}{3!} \dots \\ &=(\frac{x^0}{0!}-\frac{x^2}{2!} + \frac{x^4}{4!} - \frac{x^6}{6!} + \dots ) + i(\frac{x^1}{1!} -\frac{x^3}{3!} + \frac{x^5}{5!}-\frac{x^7}{7!} + \dots \\&cos(x) + isin(x)\end{align*}\)
此時,我們便可以獲得著名的歐拉公式:\(e^{ix} = cos(x) + isin(x)\)
令,\(x = \pi\)時,\(\Rightarrow e^{i\pi} + 1 = 0\)
歐拉公式在三角函數、圓周率、虛數以及自然指數之間建立的橋樑,在很多領域都扮演著重要的角色。
如果你對偶啦公式的正確性感到疑惑,不妨在\texttt{Python}中驗證一下:

    \begin{Verbatim}[commandchars=\\\{\}]
{\color{incolor}In [{\color{incolor}70}]:} \PY{k+kn}{import} \PY{n+nn}{math} 
         \PY{k+kn}{import} \PY{n+nn}{numpy} \PY{k}{as} \PY{n+nn}{np}
         \PY{n}{a} \PY{o}{=} \PY{n}{np}\PY{o}{.}\PY{n}{sin}\PY{p}{(}\PY{n}{x}\PY{p}{)}  
         \PY{n}{b} \PY{o}{=} \PY{n}{np}\PY{o}{.}\PY{n}{cos}\PY{p}{(}\PY{n}{x}\PY{p}{)}
         \PY{n}{x} \PY{o}{=} \PY{n}{np}\PY{o}{.}\PY{n}{pi} 
         \PY{c+c1}{\PYZsh{} the imaginary number in Numpy is \PYZsq{}j\PYZsq{};}
         \PY{n}{lhs} \PY{o}{=} \PY{n}{math}\PY{o}{.}\PY{n}{e}\PY{o}{*}\PY{o}{*}\PY{p}{(}\PY{l+m+mi}{1}\PY{n}{j}\PY{o}{*}\PY{n}{x}\PY{p}{)}
         \PY{n}{rhs} \PY{o}{=} \PY{n}{b} \PY{o}{+} \PY{p}{(}\PY{l+m+mi}{0}\PY{o}{+}\PY{l+m+mi}{1}\PY{n}{j}\PY{p}{)}\PY{o}{*}\PY{n}{a}
         \PY{k}{if}\PY{p}{(}\PY{n}{lhs} \PY{o}{==} \PY{n}{rhs}\PY{p}{)}\PY{p}{:}
             \PY{n+nb}{print}\PY{p}{(}\PY{n+nb}{bool}\PY{p}{(}\PY{l+m+mi}{1}\PY{p}{)}\PY{p}{)}
         \PY{k}{else}\PY{p}{:}
             \PY{n+nb}{print}\PY{p}{(}\PY{n+nb}{bool}\PY{p}{(}\PY{l+m+mi}{0}\PY{p}{)}\PY{p}{)}
\end{Verbatim}


    \begin{Verbatim}[commandchars=\\\{\}]
True

    \end{Verbatim}

    這裡給大家介紹一個很好的 \texttt{Python}
庫:\texttt{sympy},如名所示,它是符號數學的 \texttt{Python}
庫,它的目標是稱為一個全功能的計算機代數系統,同時保證代碼簡潔、易於理解和拓展;
所以,我們也可以通過 \texttt{sympy} 來展開 \(e^x\) 來看看它的結果是什麼🙂

    \begin{Verbatim}[commandchars=\\\{\}]
{\color{incolor}In [{\color{incolor}71}]:} \PY{k+kn}{import} \PY{n+nn}{sympy}
         \PY{n}{z} \PY{o}{=}\PY{n}{sympy}\PY{o}{.}\PY{n}{Symbol}\PY{p}{(}\PY{l+s+s1}{\PYZsq{}}\PY{l+s+s1}{z}\PY{l+s+s1}{\PYZsq{}}\PY{p}{,}\PY{n}{real} \PY{o}{=} \PY{k+kc}{True}\PY{p}{)}
         \PY{n}{sympy}\PY{o}{.}\PY{n}{expand}\PY{p}{(}\PY{n}{sympy}\PY{o}{.}\PY{n}{E}\PY{o}{*}\PY{o}{*}\PY{p}{(}\PY{n}{sympy}\PY{o}{.}\PY{n}{I}\PY{o}{*}\PY{n}{z}\PY{p}{)}\PY{p}{,}\PY{n+nb}{complex} \PY{o}{=} \PY{k+kc}{True}\PY{p}{)}
\end{Verbatim}


\begin{Verbatim}[commandchars=\\\{\}]
{\color{outcolor}Out[{\color{outcolor}71}]:} I*sin(z) + cos(z)
\end{Verbatim}
            
    將函數寫成多項式形式有很多的好處,多項式的微分和積分都相對容易。這是就很容易證明這個公式了:
\(\frac{d}{dx}e^x = e^x \frac{d}{dx}sin(x) = cos(x)\frac{d}{dx}cos(x) = -sin(x)\)

    喔,對了,這一章怎麼能沒有圖呢?收尾之前來一發吧: 我也不知道这是啥 🤨

    \begin{Verbatim}[commandchars=\\\{\}]
{\color{incolor}In [{\color{incolor}72}]:} \PY{k+kn}{import} \PY{n+nn}{numpy} \PY{k}{as} \PY{n+nn}{np}  
         \PY{k+kn}{import} \PY{n+nn}{matplotlib}\PY{n+nn}{.}\PY{n+nn}{pyplot} \PY{k}{as} \PY{n+nn}{plt}  
         \PY{k+kn}{import} \PY{n+nn}{mpl\PYZus{}toolkits}\PY{n+nn}{.}\PY{n+nn}{mplot3d}  
           
         \PY{n}{x}\PY{p}{,}\PY{n}{y}\PY{o}{=}\PY{n}{np}\PY{o}{.}\PY{n}{mgrid}\PY{p}{[}\PY{o}{\PYZhy{}}\PY{l+m+mi}{2}\PY{p}{:}\PY{l+m+mi}{2}\PY{p}{:}\PY{l+m+mi}{20}\PY{n}{j}\PY{p}{,}\PY{o}{\PYZhy{}}\PY{l+m+mi}{2}\PY{p}{:}\PY{l+m+mi}{2}\PY{p}{:}\PY{l+m+mi}{20}\PY{n}{j}\PY{p}{]}  
         \PY{n}{z}\PY{o}{=}\PY{n}{x}\PY{o}{*}\PY{n}{np}\PY{o}{.}\PY{n}{exp}\PY{p}{(}\PY{o}{\PYZhy{}}\PY{n}{x}\PY{o}{*}\PY{o}{*}\PY{l+m+mi}{2}\PY{o}{\PYZhy{}}\PY{n}{y}\PY{o}{*}\PY{o}{*}\PY{l+m+mi}{2}\PY{p}{)}  
           
         \PY{n}{ax}\PY{o}{=}\PY{n}{plt}\PY{o}{.}\PY{n}{subplot}\PY{p}{(}\PY{l+m+mi}{111}\PY{p}{,}\PY{n}{projection}\PY{o}{=}\PY{l+s+s1}{\PYZsq{}}\PY{l+s+s1}{3d}\PY{l+s+s1}{\PYZsq{}}\PY{p}{)}  
         \PY{n}{ax}\PY{o}{.}\PY{n}{plot\PYZus{}surface}\PY{p}{(}\PY{n}{x}\PY{p}{,}\PY{n}{y}\PY{p}{,}\PY{n}{z}\PY{p}{,}\PY{n}{rstride}\PY{o}{=}\PY{l+m+mi}{2}\PY{p}{,}\PY{n}{cstride}\PY{o}{=}\PY{l+m+mi}{1}\PY{p}{,}\PY{n}{cmap}\PY{o}{=}\PY{n}{plt}\PY{o}{.}\PY{n}{cm}\PY{o}{.}\PY{n}{coolwarm}\PY{p}{,}\PY{n}{alpha}\PY{o}{=}\PY{l+m+mf}{0.8}\PY{p}{)}  
         \PY{n}{ax}\PY{o}{.}\PY{n}{set\PYZus{}xlabel}\PY{p}{(}\PY{l+s+s1}{\PYZsq{}}\PY{l+s+s1}{x}\PY{l+s+s1}{\PYZsq{}}\PY{p}{)}  
         \PY{n}{ax}\PY{o}{.}\PY{n}{set\PYZus{}ylabel}\PY{p}{(}\PY{l+s+s1}{\PYZsq{}}\PY{l+s+s1}{y}\PY{l+s+s1}{\PYZsq{}}\PY{p}{)}  
         \PY{n}{ax}\PY{o}{.}\PY{n}{set\PYZus{}zlabel}\PY{p}{(}\PY{l+s+s1}{\PYZsq{}}\PY{l+s+s1}{z}\PY{l+s+s1}{\PYZsq{}}\PY{p}{)}    
         \PY{n}{plt}\PY{o}{.}\PY{n}{show}\PY{p}{(}\PY{p}{)}
\end{Verbatim}


    \begin{center}
    \adjustimage{max size={0.9\linewidth}{0.9\paperheight}}{Use_PY_in_Calculus_files/Use_PY_in_Calculus_33_0.png}
    \end{center}
    { \hspace*{\fill} \\}
    
    \hypertarget{ux6cf0ux52d2ux7d1aux6578}{%
\subsubsection{泰勒級數}\label{ux6cf0ux52d2ux7d1aux6578}}

\hypertarget{ux6cf0ux52d2ux7d1aux6578taylor-series}{%
\paragraph{泰勒級數(Taylor
Series)}\label{ux6cf0ux52d2ux7d1aux6578taylor-series}}

在前幾章的預熱之後,讀者可能有這樣的疑問,是否任何函數都可以寫成友善的多項式形式呢?
到目前為止,我們介紹的\(e^x\), \(sin(x)\), \(cos(x)\)
都可以用多項式進行表達。其實,這些多項式實際上就是這些函數在 \(x=0\)
處展開的泰勒級數。 下面我們給出函數 \(f(x)\) 在\(x=0\)
處展開的泰勒級數的定義:
\(\begin{align*}f(x) = f(0) + \frac{f'(0)}{1!}x + \frac{f''(0)}{2!}x^2 + \frac{f'''(0)}{3!}x^3 + \dots = \sum^{\infty}{k = 0} \frac{f^{(k)}(0)}{k!}x^k \end{align*}\)
其中:\(f^{(k)}(0)\) 表示函數 \(f\) 在 \(k\) 次導函數在 \(x=0\) 的取值。
我們知道 \(e^x\) 無論計算多少次導數結果出來都是 \(e^x\)
即,\(exp(x) = exp'(x)=exp''(x)=exp'''(x)=exp'''(x) = \dots\)
因而,根據上面的定義展開:
\(\begin{align*}exp(x) &= exp(0) + \frac{exp'(0)}{1!}+\frac{exp''(0)}{2!}x^2 +\frac{exp'''(0)}{3!}x^3 + \dots \\ &=1 + \frac{x}{1!} + \frac{x^2}{2!} + \frac{x^3}{3!} + \dots \\&=\sum_{k=0}^{\infty}\frac{x^k}{k!}\end{align*}\)
\#\#\#\# 多項式近似(Polynomial Approximation)
泰勒級數,可以把非常複雜的函數變成無限項的和的形式。通常,我們可以只計算泰勒級數的前幾項和,就可以獲得原函數的局部近似了。在做這樣的多項式近似時,我們所計算的項越多,則近似的結果越精確。
下面,開始使用 \texttt{python} 做演示

    \begin{Verbatim}[commandchars=\\\{\}]
{\color{incolor}In [{\color{incolor}73}]:} \PY{k+kn}{import} \PY{n+nn}{sympy} \PY{k}{as} \PY{n+nn}{sy}
         \PY{k+kn}{import} \PY{n+nn}{numpy} \PY{k}{as} \PY{n+nn}{np}
         \PY{k+kn}{from} \PY{n+nn}{sympy}\PY{n+nn}{.}\PY{n+nn}{functions} \PY{k}{import} \PY{n}{sin}\PY{p}{,}\PY{n}{cos}
         \PY{k+kn}{import} \PY{n+nn}{matplotlib}\PY{n+nn}{.}\PY{n+nn}{pyplot} \PY{k}{as} \PY{n+nn}{plt}
         
         \PY{n}{plt}\PY{o}{.}\PY{n}{style}\PY{o}{.}\PY{n}{use}\PY{p}{(}\PY{l+s+s2}{\PYZdq{}}\PY{l+s+s2}{ggplot}\PY{l+s+s2}{\PYZdq{}}\PY{p}{)}
         
         \PY{c+c1}{\PYZsh{} Define the variable and the function to approximate}
         \PY{n}{x} \PY{o}{=} \PY{n}{sy}\PY{o}{.}\PY{n}{Symbol}\PY{p}{(}\PY{l+s+s1}{\PYZsq{}}\PY{l+s+s1}{x}\PY{l+s+s1}{\PYZsq{}}\PY{p}{)}
         \PY{n}{f} \PY{o}{=} \PY{n}{sin}\PY{p}{(}\PY{n}{x}\PY{p}{)}
         
         \PY{c+c1}{\PYZsh{} Factorial function}
         \PY{k}{def} \PY{n+nf}{factorial}\PY{p}{(}\PY{n}{n}\PY{p}{)}\PY{p}{:}
             \PY{k}{if} \PY{n}{n} \PY{o}{\PYZlt{}}\PY{o}{=} \PY{l+m+mi}{0}\PY{p}{:}
                 \PY{k}{return} \PY{l+m+mi}{1}
             \PY{k}{else}\PY{p}{:}
                 \PY{k}{return} \PY{n}{n}\PY{o}{*}\PY{n}{factorial}\PY{p}{(}\PY{n}{n}\PY{o}{\PYZhy{}}\PY{l+m+mi}{1}\PY{p}{)}
         
         \PY{c+c1}{\PYZsh{} Taylor approximation at x0 of the function \PYZsq{}function\PYZsq{}}
         \PY{k}{def} \PY{n+nf}{taylor}\PY{p}{(}\PY{n}{function}\PY{p}{,}\PY{n}{x0}\PY{p}{,}\PY{n}{n}\PY{p}{)}\PY{p}{:}
             \PY{n}{i} \PY{o}{=} \PY{l+m+mi}{0}
             \PY{n}{p} \PY{o}{=} \PY{l+m+mi}{0}
             \PY{k}{while} \PY{n}{i} \PY{o}{\PYZlt{}}\PY{o}{=} \PY{n}{n}\PY{p}{:}
                 \PY{n}{p} \PY{o}{=} \PY{n}{p} \PY{o}{+} \PY{p}{(}\PY{n}{function}\PY{o}{.}\PY{n}{diff}\PY{p}{(}\PY{n}{x}\PY{p}{,}\PY{n}{i}\PY{p}{)}\PY{o}{.}\PY{n}{subs}\PY{p}{(}\PY{n}{x}\PY{p}{,}\PY{n}{x0}\PY{p}{)}\PY{p}{)}\PY{o}{/}\PY{p}{(}\PY{n}{factorial}\PY{p}{(}\PY{n}{i}\PY{p}{)}\PY{p}{)}\PY{o}{*}\PY{p}{(}\PY{n}{x}\PY{o}{\PYZhy{}}\PY{n}{x0}\PY{p}{)}\PY{o}{*}\PY{o}{*}\PY{n}{i}
                 \PY{n}{i} \PY{o}{+}\PY{o}{=} \PY{l+m+mi}{1}
             \PY{k}{return} \PY{n}{p}
         \PY{c+c1}{\PYZsh{} Plot results}
         \PY{k}{def} \PY{n+nf}{plot}\PY{p}{(}\PY{p}{)}\PY{p}{:}
             \PY{n}{x\PYZus{}lims} \PY{o}{=} \PY{p}{[}\PY{o}{\PYZhy{}}\PY{l+m+mi}{5}\PY{p}{,}\PY{l+m+mi}{5}\PY{p}{]}
             \PY{n}{x1} \PY{o}{=} \PY{n}{np}\PY{o}{.}\PY{n}{linspace}\PY{p}{(}\PY{n}{x\PYZus{}lims}\PY{p}{[}\PY{l+m+mi}{0}\PY{p}{]}\PY{p}{,}\PY{n}{x\PYZus{}lims}\PY{p}{[}\PY{l+m+mi}{1}\PY{p}{]}\PY{p}{,}\PY{l+m+mi}{800}\PY{p}{)}
             \PY{n}{y1} \PY{o}{=} \PY{p}{[}\PY{p}{]}
             \PY{c+c1}{\PYZsh{} Approximate up until 10 starting from 1 and using steps of 2}
             \PY{k}{for} \PY{n}{j} \PY{o+ow}{in} \PY{n+nb}{range}\PY{p}{(}\PY{l+m+mi}{1}\PY{p}{,}\PY{l+m+mi}{10}\PY{p}{,}\PY{l+m+mi}{2}\PY{p}{)}\PY{p}{:}
                 \PY{n}{func} \PY{o}{=} \PY{n}{taylor}\PY{p}{(}\PY{n}{f}\PY{p}{,}\PY{l+m+mi}{0}\PY{p}{,}\PY{n}{j}\PY{p}{)}
                 \PY{n+nb}{print}\PY{p}{(}\PY{l+s+s1}{\PYZsq{}}\PY{l+s+s1}{Taylor expansion at n=}\PY{l+s+s1}{\PYZsq{}}\PY{o}{+}\PY{n+nb}{str}\PY{p}{(}\PY{n}{j}\PY{p}{)}\PY{p}{,}\PY{n}{func}\PY{p}{)}
                 \PY{k}{for} \PY{n}{k} \PY{o+ow}{in} \PY{n}{x1}\PY{p}{:}
                     \PY{n}{y1}\PY{o}{.}\PY{n}{append}\PY{p}{(}\PY{n}{func}\PY{o}{.}\PY{n}{subs}\PY{p}{(}\PY{n}{x}\PY{p}{,}\PY{n}{k}\PY{p}{)}\PY{p}{)}
                 \PY{n}{plt}\PY{o}{.}\PY{n}{plot}\PY{p}{(}\PY{n}{x1}\PY{p}{,}\PY{n}{y1}\PY{p}{,}\PY{n}{label}\PY{o}{=}\PY{l+s+s1}{\PYZsq{}}\PY{l+s+s1}{order }\PY{l+s+s1}{\PYZsq{}}\PY{o}{+}\PY{n+nb}{str}\PY{p}{(}\PY{n}{j}\PY{p}{)}\PY{p}{)}
                 \PY{n}{y1} \PY{o}{=} \PY{p}{[}\PY{p}{]}
             \PY{c+c1}{\PYZsh{} Plot the function to approximate (sine, in this case)}
             \PY{n}{plt}\PY{o}{.}\PY{n}{plot}\PY{p}{(}\PY{n}{x1}\PY{p}{,}\PY{n}{np}\PY{o}{.}\PY{n}{sin}\PY{p}{(}\PY{n}{x1}\PY{p}{)}\PY{p}{,}\PY{n}{label}\PY{o}{=}\PY{l+s+s1}{\PYZsq{}}\PY{l+s+s1}{sin of x}\PY{l+s+s1}{\PYZsq{}}\PY{p}{)}
             \PY{n}{plt}\PY{o}{.}\PY{n}{xlim}\PY{p}{(}\PY{n}{x\PYZus{}lims}\PY{p}{)}
             \PY{n}{plt}\PY{o}{.}\PY{n}{ylim}\PY{p}{(}\PY{p}{[}\PY{o}{\PYZhy{}}\PY{l+m+mi}{5}\PY{p}{,}\PY{l+m+mi}{5}\PY{p}{]}\PY{p}{)}
             \PY{n}{plt}\PY{o}{.}\PY{n}{xlabel}\PY{p}{(}\PY{l+s+s1}{\PYZsq{}}\PY{l+s+s1}{x}\PY{l+s+s1}{\PYZsq{}}\PY{p}{)}
             \PY{n}{plt}\PY{o}{.}\PY{n}{ylabel}\PY{p}{(}\PY{l+s+s1}{\PYZsq{}}\PY{l+s+s1}{y}\PY{l+s+s1}{\PYZsq{}}\PY{p}{)}
             \PY{n}{plt}\PY{o}{.}\PY{n}{legend}\PY{p}{(}\PY{p}{)}
             \PY{n}{plt}\PY{o}{.}\PY{n}{grid}\PY{p}{(}\PY{k+kc}{True}\PY{p}{)}
             \PY{n}{plt}\PY{o}{.}\PY{n}{title}\PY{p}{(}\PY{l+s+s1}{\PYZsq{}}\PY{l+s+s1}{Taylor series approximation}\PY{l+s+s1}{\PYZsq{}}\PY{p}{)}
             \PY{n}{plt}\PY{o}{.}\PY{n}{show}\PY{p}{(}\PY{p}{)}
         
         \PY{n}{plot}\PY{p}{(}\PY{p}{)}
\end{Verbatim}


    \begin{Verbatim}[commandchars=\\\{\}]
Taylor expansion at n=1 x
Taylor expansion at n=3 -x**3/6 + x
Taylor expansion at n=5 x**5/120 - x**3/6 + x
Taylor expansion at n=7 -x**7/5040 + x**5/120 - x**3/6 + x
Taylor expansion at n=9 x**9/362880 - x**7/5040 + x**5/120 - x**3/6 + x

    \end{Verbatim}

    \begin{center}
    \adjustimage{max size={0.9\linewidth}{0.9\paperheight}}{Use_PY_in_Calculus_files/Use_PY_in_Calculus_35_1.png}
    \end{center}
    { \hspace*{\fill} \\}
    
    \hypertarget{ux5c55ux958bux9edeexpansion-point}{%
\subparagraph{展開點(Expansion
Point)}\label{ux5c55ux958bux9edeexpansion-point}}

上述的式子,都是在 \(x=0\) 進行的,我們會發現多項式近似只在 \(x=0\)
處較為準確。但,這不代表,我們可以在別的點進行多項式近似,如\(x=a\) :
\$f(x) = f(a) + \frac{f'(a)}{1!}(x-a) + \frac{f''(a)}{2!}(x-a)\^{}2 +
\dots \$

    \hypertarget{ux6975ux9650}{%
\subsection{極限}\label{ux6975ux9650}}

\hypertarget{ux6975ux9650limits}{%
\subsubsection{極限(Limits)}\label{ux6975ux9650limits}}

函數的極限,描述的是輸入值在接近一個特定值時函數的表現。 定義:
我們如果要稱函數 \(f(x)\) 在 \(x = a\) 處的極限為
\(L\),即:\(lim_{x\rightarrow a} f(x) = L\),則需要: 對任意一個
\(\epsilon > 0\),我們要能找到一個 \(\delta > 0\) 使的當 \(x\)
的取值滿足:\(0<|x-a|<\delta\)時,\(|f(x)-L|<\epsilon\)

    \begin{Verbatim}[commandchars=\\\{\}]
{\color{incolor}In [{\color{incolor}74}]:} \PY{k+kn}{import} \PY{n+nn}{sympy}
         \PY{n}{x} \PY{o}{=} \PY{n}{sympy}\PY{o}{.}\PY{n}{Symbol}\PY{p}{(}\PY{l+s+s1}{\PYZsq{}}\PY{l+s+s1}{x}\PY{l+s+s1}{\PYZsq{}}\PY{p}{,}\PY{n}{real} \PY{o}{=} \PY{k+kc}{True}\PY{p}{)}
         \PY{n}{f} \PY{o}{=} \PY{k}{lambda} \PY{n}{x}\PY{p}{:} \PY{n}{x}\PY{o}{*}\PY{o}{*}\PY{n}{x}\PY{o}{\PYZhy{}}\PY{l+m+mi}{2}\PY{o}{*}\PY{n}{x}\PY{o}{\PYZhy{}}\PY{l+m+mi}{6}
         \PY{n}{y} \PY{o}{=} \PY{n}{f}\PY{p}{(}\PY{n}{x}\PY{p}{)}
         \PY{n+nb}{print}\PY{p}{(}\PY{n}{y}\PY{o}{.}\PY{n}{limit}\PY{p}{(}\PY{n}{x}\PY{p}{,}\PY{l+m+mi}{2}\PY{p}{)}\PY{p}{)}
\end{Verbatim}


    \begin{Verbatim}[commandchars=\\\{\}]
-6

    \end{Verbatim}

    \hypertarget{ux51fdux6578ux7684ux9023ux7e8cux6027}{%
\paragraph{函數的連續性}\label{ux51fdux6578ux7684ux9023ux7e8cux6027}}

極限可以用來判斷一個函數是否為連續函數。
當極限\(\begin{align*}\lim_{x\rightarrow a} f(x)= f(a)\end{align*}\)時,稱函數\(f(x)\)在點\$
x = a\$
處為連續的。當一個函數在其定義域中任意一點均為連續,則稱該函數是連續函數。

\hypertarget{ux6cf0ux52d2ux7d1aux6578ux7528ux65bcux6975ux9650ux8a08ux7b97}{%
\paragraph{泰勒級數用於極限計算}\label{ux6cf0ux52d2ux7d1aux6578ux7528ux65bcux6975ux9650ux8a08ux7b97}}

我們在中學的時候,學習過關於部分極限的計算,這裡不再贅述。泰勒級數也可以用於計算一些形式比較複雜的函數的極限。這裡,僅舉一個例子:
\(\begin{align*} \lim_{x\rightarrow 0}\frac{sin(X)}{x} &= lim_{x\rightarrow 0} \frac{\frac{x}{1!}-\frac{x^3}{3!}\dots }{x} \\ &= \lim_{x\rightarrow 0} \frac{x(1-\frac{x^2}{3!}+\frac{x^4}{5!}-\frac{x^6}{7!}+\dots}{x} \\ &= \lim_{x\rightarrow 0} 1 -\frac{x^2}{3!} + \frac{x^4}{5!}-\frac{x^6}{7!}+\dots \\& = 1 \end{align*}\)

\hypertarget{ux6d1bux5fc5ux9054ux6cd5ux5247lhopitals-rule}{%
\paragraph{洛必達法則(l'Hopital's
rule)}\label{ux6d1bux5fc5ux9054ux6cd5ux5247lhopitals-rule}}

在高中,老師就教過的一個神奇的法則:如果我們在求極限的時候,所求極限是無窮的,那我們可以試一下使用洛必達法則,哪些形式呢:\(\frac{0}{0}, \frac{\infty}{\infty}, \frac{\infty}{0}\)等等。\textbf{這裡,我們要注意一個前提條件:上下兩個函數都是連續函數才可以使用洛必達法則}這裡我們用
\(\frac{0}{0}\) 作為一個例子:
\(\begin{align*}\lim_{x \rightarrow a}\frac{f'(x)}{g'(x)} \\ = \lim_{x \rightarrow a}\frac{f'(x)}{g'(x)} \end{align*}\)
若此時,分子分母還都是\(0\)的話,再次重複:\(\begin{align*}\lim_{x \rightarrow a}\frac{f''(x)}{g''(x)}\end{align*}\)

    \hypertarget{ux5927oux8a18ux6cd5big-o-notation}{%
\paragraph{\texorpdfstring{大\(O\)記法(Big-O
Notation)}{大O記法(Big-O Notation)}}\label{ux5927oux8a18ux6cd5big-o-notation}}

\emph{這個我在網上能找到的資料很少,大多是算法的時間複雜度相關的資料}
算法複雜度的定義: \textgreater{} We denote an algorithm has a
complexity of O(g(n))if there exists a constants \textgreater{}
\(c \in R^+\), suchthat \(t(n)\leq c\cdot g(n), \forall n\geq 0\).
\textgreater{} \textgreater{} 這裡的\(n\)是算法的輸入大小(input
size),可以看作變量的個數等等。 \textgreater{} \textgreater{}
方程\(t\)在這裡指算法的``時間'',也可以看作執行基本算法需要的步驟等等。
\textgreater{} \textgreater{} 方程\(g\)在這裡值得是任意函數。

\emph{我們也可以將這個概念用在函數上:}
我們已經見過了很多函數,在比較這兩個函數時,我們可能會知道,隨著輸入值\(x\)的增加或者減少,兩個函數的輸出值,兩個函數的輸出值增長或者減少的速度究竟是誰快誰慢,哪一個函數最終會遠遠甩開另一個。
通過繪製函數圖像,我們可以得到一些之直觀的感受:

    \begin{Verbatim}[commandchars=\\\{\}]
{\color{incolor}In [{\color{incolor}4}]:} \PY{k+kn}{import} \PY{n+nn}{numpy} \PY{k}{as} \PY{n+nn}{np}
        \PY{k+kn}{import} \PY{n+nn}{matplotlib}\PY{n+nn}{.}\PY{n+nn}{pyplot} \PY{k}{as} \PY{n+nn}{plt}
        \PY{n}{m}\PY{o}{=} \PY{n+nb}{range}\PY{p}{(}\PY{l+m+mi}{1}\PY{p}{,}\PY{l+m+mi}{7}\PY{p}{)}
        \PY{n}{fac} \PY{o}{=} \PY{p}{[}\PY{n}{np}\PY{o}{.}\PY{n}{math}\PY{o}{.}\PY{n}{factorial}\PY{p}{(}\PY{n}{i}\PY{p}{)} \PY{k}{for} \PY{n}{i} \PY{o+ow}{in} \PY{n}{m}\PY{p}{]} \PY{c+c1}{\PYZsh{}fac means factorial\PYZsh{}}
        \PY{n}{exponential} \PY{o}{=} \PY{p}{[}\PY{n}{np}\PY{o}{.}\PY{n}{e}\PY{o}{*}\PY{o}{*}\PY{n}{i} \PY{k}{for} \PY{n}{i} \PY{o+ow}{in} \PY{n}{m}\PY{p}{]}
        \PY{n}{polynomial} \PY{o}{=} \PY{p}{[}\PY{n}{i}\PY{o}{*}\PY{o}{*}\PY{l+m+mi}{3} \PY{k}{for} \PY{n}{i} \PY{o+ow}{in} \PY{n}{m}\PY{p}{]}
        \PY{n}{logarithimic} \PY{o}{=} \PY{p}{[}\PY{n}{np}\PY{o}{.}\PY{n}{log}\PY{p}{(}\PY{n}{i}\PY{p}{)} \PY{k}{for} \PY{n}{i} \PY{o+ow}{in} \PY{n}{m}\PY{p}{]}
        
        \PY{n}{plt}\PY{o}{.}\PY{n}{plot}\PY{p}{(}\PY{n}{m}\PY{p}{,}\PY{n}{fac}\PY{p}{,}\PY{l+s+s1}{\PYZsq{}}\PY{l+s+s1}{black}\PY{l+s+s1}{\PYZsq{}}\PY{p}{,}\PY{n}{m}\PY{p}{,}\PY{n}{exponential}\PY{p}{,}\PY{l+s+s1}{\PYZsq{}}\PY{l+s+s1}{blue}\PY{l+s+s1}{\PYZsq{}}\PY{p}{,}\PY{n}{m}\PY{p}{,}\PY{n}{polynomial}\PY{p}{,}\PY{l+s+s1}{\PYZsq{}}\PY{l+s+s1}{green}\PY{l+s+s1}{\PYZsq{}}\PY{p}{,}\PY{n}{m}\PY{p}{,}\PY{n}{logarithimic}\PY{p}{,}\PY{l+s+s1}{\PYZsq{}}\PY{l+s+s1}{red}\PY{l+s+s1}{\PYZsq{}}\PY{p}{)}
        \PY{n}{plt}\PY{o}{.}\PY{n}{show}\PY{p}{(}\PY{p}{)}
\end{Verbatim}


    \begin{center}
    \adjustimage{max size={0.9\linewidth}{0.9\paperheight}}{Use_PY_in_Calculus_files/Use_PY_in_Calculus_41_0.png}
    \end{center}
    { \hspace*{\fill} \\}
    
    根據上面的圖,我們可以看出\(x \rightarrow \infty\)
時,\(x! > e^x > x^3 > ln(x)\)
,想要證明的話,我們需要去極限去算(用洛必達法則)。
\(\begin{align*}\lim_{x\rightarrow \infty}\frac{e^x}{x^3} = \infty \end{align*}\)
可以看出,趨於無窮時,分子遠大於分母,反之同理。 我們可以用
\texttt{sympy} 來算一下這個例子:

    \begin{Verbatim}[commandchars=\\\{\}]
{\color{incolor}In [{\color{incolor}14}]:} \PY{k+kn}{import} \PY{n+nn}{sympy}
         \PY{k+kn}{import} \PY{n+nn}{numpy} \PY{k}{as} \PY{n+nn}{np}
         \PY{n}{x} \PY{o}{=} \PY{n}{sympy}\PY{o}{.}\PY{n}{Symbol}\PY{p}{(}\PY{l+s+s1}{\PYZsq{}}\PY{l+s+s1}{x}\PY{l+s+s1}{\PYZsq{}}\PY{p}{,}\PY{n}{real} \PY{o}{=} \PY{k+kc}{True}\PY{p}{)}
         \PY{n}{f} \PY{o}{=} \PY{k}{lambda} \PY{n}{x}\PY{p}{:} \PY{n}{np}\PY{o}{.}\PY{n}{e}\PY{o}{*}\PY{o}{*}\PY{n}{x}\PY{o}{/}\PY{n}{x}\PY{o}{*}\PY{o}{*}\PY{l+m+mi}{3}
         \PY{n}{y} \PY{o}{=} \PY{n}{f}\PY{p}{(}\PY{n}{x}\PY{p}{)}
         \PY{n+nb}{print}\PY{p}{(}\PY{n}{y}\PY{o}{.}\PY{n}{limit}\PY{p}{(}\PY{n}{x}\PY{p}{,}\PY{n}{oo}\PY{p}{)}\PY{p}{)}
\end{Verbatim}


    \begin{Verbatim}[commandchars=\\\{\}]
oo

    \end{Verbatim}

    為了描述這種隨著輸入\(x\rightarrow \infty\)或\(x \rightarrow 0\)時,函數的表現,我們如下定義大\(O\)記法:
若我們稱函數\(f(x)\)在\(x\rightarrow 0\)時,時\(O(g(x))\),則需要找到一個常數\(C\),對於所有足夠小的\(x\)均有\(|f(x)|<C|g(x)|\)
若我們稱函數\(f(x)\)在\(x\rightarrow 0\)時是\(O(g(x))\),則需要找一個常數\(C\),對於所有足夠大的\(x\)均有\(|f(x)|<C|g(x)|\)
大\(O\)記法之所以得此名稱,是因為函數的增長速率很多時候被稱為函數的階(\textbf{Order})
下面舉一個例子:當\(x\rightarrow \infty\)時,\(x\sqrt{1+x^2}\)是\(O(x^2)\)

    \begin{Verbatim}[commandchars=\\\{\}]
{\color{incolor}In [{\color{incolor}12}]:} \PY{k+kn}{import} \PY{n+nn}{sympy}
         \PY{k+kn}{import} \PY{n+nn}{numpy} \PY{k}{as} \PY{n+nn}{np}
         \PY{k+kn}{import} \PY{n+nn}{matplotlib}\PY{n+nn}{.}\PY{n+nn}{pyplot} \PY{k}{as} \PY{n+nn}{plt}
         \PY{n}{x} \PY{o}{=} \PY{n}{sympy}\PY{o}{.}\PY{n}{Symbol}\PY{p}{(}\PY{l+s+s1}{\PYZsq{}}\PY{l+s+s1}{x}\PY{l+s+s1}{\PYZsq{}}\PY{p}{,}\PY{n}{real} \PY{o}{=} \PY{k+kc}{True}\PY{p}{)}
         \PY{n}{xvals} \PY{o}{=} \PY{n}{np}\PY{o}{.}\PY{n}{linspace}\PY{p}{(}\PY{l+m+mi}{0}\PY{p}{,}\PY{l+m+mi}{100}\PY{p}{,}\PY{l+m+mi}{1000}\PY{p}{)}
         \PY{n}{f} \PY{o}{=} \PY{n}{x}\PY{o}{*}\PY{n}{sympy}\PY{o}{.}\PY{n}{sqrt}\PY{p}{(}\PY{l+m+mi}{1}\PY{o}{+}\PY{n}{x}\PY{o}{*}\PY{o}{*}\PY{l+m+mi}{2}\PY{p}{)}
         \PY{n}{g} \PY{o}{=} \PY{l+m+mi}{2}\PY{o}{*}\PY{n}{x}\PY{o}{*}\PY{o}{*}\PY{l+m+mi}{2}
         \PY{n}{y1} \PY{o}{=} \PY{p}{[}\PY{n}{f}\PY{o}{.}\PY{n}{evalf}\PY{p}{(}\PY{n}{subs} \PY{o}{=} \PY{p}{\PYZob{}}\PY{n}{x}\PY{p}{:}\PY{n}{xval}\PY{p}{\PYZcb{}}\PY{p}{)} \PY{k}{for} \PY{n}{xval} \PY{o+ow}{in} \PY{n}{xvals}\PY{p}{]}
         \PY{n}{y2} \PY{o}{=} \PY{p}{[}\PY{n}{g}\PY{o}{.}\PY{n}{evalf}\PY{p}{(}\PY{n}{subs} \PY{o}{=} \PY{p}{\PYZob{}}\PY{n}{x}\PY{p}{:}\PY{n}{xval}\PY{p}{\PYZcb{}}\PY{p}{)} \PY{k}{for} \PY{n}{xval} \PY{o+ow}{in} \PY{n}{xvals}\PY{p}{]}
         \PY{n}{plt}\PY{o}{.}\PY{n}{plot}\PY{p}{(}\PY{n}{xvals}\PY{p}{[}\PY{p}{:}\PY{l+m+mi}{10}\PY{p}{]}\PY{p}{,}\PY{n}{y1}\PY{p}{[}\PY{p}{:}\PY{l+m+mi}{10}\PY{p}{]}\PY{p}{,}\PY{l+s+s1}{\PYZsq{}}\PY{l+s+s1}{r}\PY{l+s+s1}{\PYZsq{}}\PY{p}{,}\PY{n}{xvals}\PY{p}{[}\PY{p}{:}\PY{l+m+mi}{10}\PY{p}{]}\PY{p}{,}\PY{n}{y2}\PY{p}{[}\PY{p}{:}\PY{l+m+mi}{10}\PY{p}{]}\PY{p}{,}\PY{l+s+s1}{\PYZsq{}}\PY{l+s+s1}{b}\PY{l+s+s1}{\PYZsq{}}\PY{p}{)}
         \PY{n}{plt}\PY{o}{.}\PY{n}{show}\PY{p}{(}\PY{p}{)}
         \PY{n}{plt}\PY{o}{.}\PY{n}{plot}\PY{p}{(}\PY{n}{xvals}\PY{p}{,}\PY{n}{y1}\PY{p}{,}\PY{l+s+s1}{\PYZsq{}}\PY{l+s+s1}{r}\PY{l+s+s1}{\PYZsq{}}\PY{p}{,}\PY{n}{xvals}\PY{p}{,}\PY{n}{y2}\PY{p}{,}\PY{l+s+s1}{\PYZsq{}}\PY{l+s+s1}{b}\PY{l+s+s1}{\PYZsq{}}\PY{p}{)}
         \PY{n}{plt}\PY{o}{.}\PY{n}{show}\PY{p}{(}\PY{p}{)}
\end{Verbatim}


    \begin{center}
    \adjustimage{max size={0.9\linewidth}{0.9\paperheight}}{Use_PY_in_Calculus_files/Use_PY_in_Calculus_45_0.png}
    \end{center}
    { \hspace*{\fill} \\}
    
    \begin{center}
    \adjustimage{max size={0.9\linewidth}{0.9\paperheight}}{Use_PY_in_Calculus_files/Use_PY_in_Calculus_45_1.png}
    \end{center}
    { \hspace*{\fill} \\}
    
    \hypertarget{ux5c0eux6578}{%
\subsection{導數}\label{ux5c0eux6578}}

\hypertarget{ux5272ux7ddasecent-line}{%
\subsubsection{割線(Secent Line)}\label{ux5272ux7ddasecent-line}}

曲線的格線是指與弧線由兩個公共點的直線。

    \begin{Verbatim}[commandchars=\\\{\}]
{\color{incolor}In [{\color{incolor}26}]:} \PY{k+kn}{import} \PY{n+nn}{numpy} \PY{k}{as} \PY{n+nn}{np}
         \PY{k+kn}{from} \PY{n+nn}{sympy}\PY{n+nn}{.}\PY{n+nn}{abc} \PY{k}{import} \PY{n}{x}
         \PY{k+kn}{import} \PY{n+nn}{matplotlib}\PY{n+nn}{.}\PY{n+nn}{pyplot} \PY{k}{as} \PY{n+nn}{plt}
         
         \PY{c+c1}{\PYZsh{} function}
         \PY{n}{f} \PY{o}{=} \PY{n}{x}\PY{o}{*}\PY{o}{*}\PY{l+m+mi}{3}\PY{o}{\PYZhy{}}\PY{l+m+mi}{3}\PY{o}{*}\PY{n}{x}\PY{o}{\PYZhy{}}\PY{l+m+mi}{6}
         \PY{c+c1}{\PYZsh{} the tengent line at x=6}
         \PY{n}{line} \PY{o}{=} \PY{l+m+mi}{106}\PY{o}{*}\PY{n}{x}\PY{o}{\PYZhy{}}\PY{l+m+mi}{428}
         
         
         \PY{n}{d4} \PY{o}{=} \PY{n}{np}\PY{o}{.}\PY{n}{linspace}\PY{p}{(}\PY{l+m+mf}{5.9}\PY{p}{,}\PY{l+m+mf}{6.1}\PY{p}{,}\PY{l+m+mi}{100}\PY{p}{)}
         \PY{n}{domains} \PY{o}{=} \PY{p}{[}\PY{n}{d3}\PY{p}{]}
         
         \PY{c+c1}{\PYZsh{} define the plot funtion}
         \PY{k}{def} \PY{n+nf}{makeplot}\PY{p}{(}\PY{n}{f}\PY{p}{,}\PY{n}{l}\PY{p}{,}\PY{n}{d}\PY{p}{)}\PY{p}{:}
             \PY{n}{plt}\PY{o}{.}\PY{n}{plot}\PY{p}{(}\PY{n}{d}\PY{p}{,}\PY{p}{[}\PY{n}{f}\PY{o}{.}\PY{n}{evalf}\PY{p}{(}\PY{n}{subs}\PY{o}{=}\PY{p}{\PYZob{}}\PY{n}{x}\PY{p}{:}\PY{n}{xval}\PY{p}{\PYZcb{}}\PY{p}{)} \PY{k}{for} \PY{n}{xval} \PY{o+ow}{in} \PY{n}{d}\PY{p}{]}\PY{p}{,}\PY{l+s+s1}{\PYZsq{}}\PY{l+s+s1}{b}\PY{l+s+s1}{\PYZsq{}}\PY{p}{,}\PYZbs{}
                      \PY{n}{d}\PY{p}{,}\PY{p}{[}\PY{n}{l}\PY{o}{.}\PY{n}{evalf}\PY{p}{(}\PY{n}{subs}\PY{o}{=}\PY{p}{\PYZob{}}\PY{n}{x}\PY{p}{:}\PY{n}{xval}\PY{p}{\PYZcb{}}\PY{p}{)} \PY{k}{for} \PY{n}{xval} \PY{o+ow}{in} \PY{n}{d}\PY{p}{]}\PY{p}{,}\PY{l+s+s1}{\PYZsq{}}\PY{l+s+s1}{r}\PY{l+s+s1}{\PYZsq{}}\PY{p}{)}
         
         \PY{k}{for} \PY{n}{i} \PY{o+ow}{in} \PY{n+nb}{range}\PY{p}{(}\PY{n+nb}{len}\PY{p}{(}\PY{n}{domains}\PY{p}{)}\PY{p}{)}\PY{p}{:}
             \PY{c+c1}{\PYZsh{} draw the plot and the subplot}
             \PY{n}{plt}\PY{o}{.}\PY{n}{subplot}\PY{p}{(}\PY{l+m+mi}{2}\PY{p}{,} \PY{l+m+mi}{2}\PY{p}{,} \PY{n}{i}\PY{o}{+}\PY{l+m+mi}{1}\PY{p}{)}
             \PY{n}{makeplot}\PY{p}{(}\PY{n}{f}\PY{p}{,}\PY{n}{line}\PY{p}{,}\PY{n}{domains}\PY{p}{[}\PY{n}{i}\PY{p}{]}\PY{p}{)}
         
         \PY{n}{plt}\PY{o}{.}\PY{n}{show}\PY{p}{(}\PY{p}{)}
\end{Verbatim}


    \begin{center}
    \adjustimage{max size={0.9\linewidth}{0.9\paperheight}}{Use_PY_in_Calculus_files/Use_PY_in_Calculus_47_0.png}
    \end{center}
    { \hspace*{\fill} \\}
    
    \hypertarget{ux5207ux7ddatangent-line}{%
\subsubsection{切線(Tangent Line)}\label{ux5207ux7ddatangent-line}}

中學介紹導數的時候,通常會舉兩個例子,其中一個是幾何意義上的例子:對於函數關於某一點進行球道,得到的是函數在該點處切線的斜率。
選中函數圖像中的某一點,然後不斷地將函數圖放大,當我們將鏡頭拉至足夠近後便會發現函數圖看起來像一條直線,這條直線就是切線。

    \begin{Verbatim}[commandchars=\\\{\}]
{\color{incolor}In [{\color{incolor}14}]:} \PY{k+kn}{import} \PY{n+nn}{numpy} \PY{k}{as} \PY{n+nn}{np}
         \PY{k+kn}{from} \PY{n+nn}{sympy}\PY{n+nn}{.}\PY{n+nn}{abc} \PY{k}{import} \PY{n}{x}
         \PY{k+kn}{import} \PY{n+nn}{matplotlib}\PY{n+nn}{.}\PY{n+nn}{pyplot} \PY{k}{as} \PY{n+nn}{plt}
         
         \PY{c+c1}{\PYZsh{} function}
         \PY{n}{f} \PY{o}{=} \PY{n}{x}\PY{o}{*}\PY{o}{*}\PY{l+m+mi}{3}\PY{o}{\PYZhy{}}\PY{l+m+mi}{2}\PY{o}{*}\PY{n}{x}\PY{o}{\PYZhy{}}\PY{l+m+mi}{6}
         \PY{c+c1}{\PYZsh{} the tengent line at x=6}
         \PY{n}{line} \PY{o}{=} \PY{l+m+mi}{106}\PY{o}{*}\PY{n}{x}\PY{o}{\PYZhy{}}\PY{l+m+mi}{438}
         
         \PY{n}{d1} \PY{o}{=} \PY{n}{np}\PY{o}{.}\PY{n}{linspace}\PY{p}{(}\PY{l+m+mi}{2}\PY{p}{,}\PY{l+m+mi}{10}\PY{p}{,}\PY{l+m+mi}{1000}\PY{p}{)}
         \PY{n}{d2} \PY{o}{=} \PY{n}{np}\PY{o}{.}\PY{n}{linspace}\PY{p}{(}\PY{l+m+mi}{4}\PY{p}{,}\PY{l+m+mi}{8}\PY{p}{,}\PY{l+m+mi}{1000}\PY{p}{)}
         \PY{n}{d3} \PY{o}{=} \PY{n}{np}\PY{o}{.}\PY{n}{linspace}\PY{p}{(}\PY{l+m+mi}{5}\PY{p}{,}\PY{l+m+mi}{7}\PY{p}{,}\PY{l+m+mi}{1000}\PY{p}{)}
         \PY{n}{d4} \PY{o}{=} \PY{n}{np}\PY{o}{.}\PY{n}{linspace}\PY{p}{(}\PY{l+m+mf}{5.9}\PY{p}{,}\PY{l+m+mf}{6.1}\PY{p}{,}\PY{l+m+mi}{100}\PY{p}{)}
         \PY{n}{domains} \PY{o}{=} \PY{p}{[}\PY{n}{d1}\PY{p}{,}\PY{n}{d2}\PY{p}{,}\PY{n}{d3}\PY{p}{,}\PY{n}{d4}\PY{p}{]}
         
         \PY{c+c1}{\PYZsh{} define the plot funtion}
         \PY{k}{def} \PY{n+nf}{makeplot}\PY{p}{(}\PY{n}{f}\PY{p}{,}\PY{n}{l}\PY{p}{,}\PY{n}{d}\PY{p}{)}\PY{p}{:}
             \PY{n}{plt}\PY{o}{.}\PY{n}{plot}\PY{p}{(}\PY{n}{d}\PY{p}{,}\PY{p}{[}\PY{n}{f}\PY{o}{.}\PY{n}{evalf}\PY{p}{(}\PY{n}{subs}\PY{o}{=}\PY{p}{\PYZob{}}\PY{n}{x}\PY{p}{:}\PY{n}{xval}\PY{p}{\PYZcb{}}\PY{p}{)} \PY{k}{for} \PY{n}{xval} \PY{o+ow}{in} \PY{n}{d}\PY{p}{]}\PY{p}{,}\PY{l+s+s1}{\PYZsq{}}\PY{l+s+s1}{b}\PY{l+s+s1}{\PYZsq{}}\PY{p}{,}\PYZbs{}
                      \PY{n}{d}\PY{p}{,}\PY{p}{[}\PY{n}{l}\PY{o}{.}\PY{n}{evalf}\PY{p}{(}\PY{n}{subs}\PY{o}{=}\PY{p}{\PYZob{}}\PY{n}{x}\PY{p}{:}\PY{n}{xval}\PY{p}{\PYZcb{}}\PY{p}{)} \PY{k}{for} \PY{n}{xval} \PY{o+ow}{in} \PY{n}{d}\PY{p}{]}\PY{p}{,}\PY{l+s+s1}{\PYZsq{}}\PY{l+s+s1}{r}\PY{l+s+s1}{\PYZsq{}}\PY{p}{)}
         
         \PY{k}{for} \PY{n}{i} \PY{o+ow}{in} \PY{n+nb}{range}\PY{p}{(}\PY{n+nb}{len}\PY{p}{(}\PY{n}{domains}\PY{p}{)}\PY{p}{)}\PY{p}{:}
             \PY{c+c1}{\PYZsh{} draw the plot and the subplot}
             \PY{n}{plt}\PY{o}{.}\PY{n}{subplot}\PY{p}{(}\PY{l+m+mi}{2}\PY{p}{,} \PY{l+m+mi}{2}\PY{p}{,} \PY{n}{i}\PY{o}{+}\PY{l+m+mi}{1}\PY{p}{)}
             \PY{n}{makeplot}\PY{p}{(}\PY{n}{f}\PY{p}{,}\PY{n}{line}\PY{p}{,}\PY{n}{domains}\PY{p}{[}\PY{n}{i}\PY{p}{]}\PY{p}{)}
         
         \PY{n}{plt}\PY{o}{.}\PY{n}{show}\PY{p}{(}\PY{p}{)}
\end{Verbatim}


    \begin{center}
    \adjustimage{max size={0.9\linewidth}{0.9\paperheight}}{Use_PY_in_Calculus_files/Use_PY_in_Calculus_49_0.png}
    \end{center}
    { \hspace*{\fill} \\}
    
    另一個例子就是:對路程的時間函數 \(s(t)\) 求導可以得到速度的時間函數
\(v(t)\),再進一步求導可以得到加速度的時間函數
\(a(t)\)。這個比較好理解,因為函數真正關心的是:當我們稍稍改變一點函數的輸入值時,函數的輸出值有怎樣的變化。

    \hypertarget{ux5c0eux6578derivative}{%
\subsubsection{導數(Derivative)}\label{ux5c0eux6578derivative}}

導數的定義如下: 定義一:
\(\begin{align*}f'(a) = \frac{df}{dx}\mid_{x=a} = \lim_{x\rightarrow 0} \frac{f(x)-f(a)}{x-a}\end{align*}\)
若該極限不存在,則函數在 \(x=a\) 處的導數也不存在。 定義二:
\(\begin{align*}f'(a) = \frac{df}{dx}\mid_{x=a} = \lim_{h\rightarrow 0} \frac{f(a+h)-f(a)}{h}\end{align*}\)
以上两个定义都是耳熟能详的定义了,这里不多加赘述。 \textbf{定義三}:
函數\(f(x)\)在\(x=a\)處的導數\(f'(a)\)是滿足如下條件的常數\(C\):
對於在\(a\)附近輸入值的微笑變化\(h\)有,\(f(a+h)=f(a) + Ch + O(h^2)\)
始終成立,也就是說導數\(C\)是輸出值變化中一階項的係數。
\(\begin{align*} \lim_{h\rightarrow 0} \frac{f(a+h)-f(a)}{h} = \lim_{h\rightarrow 0} C + O(h) = C \end{align*}\)
下面具一個例子,求\(cos(x)\)在\(x=a\)處的導數:
\(\begin{align*} cos(a+h) &= cos(a)cos(h) - sin(a)sin(h)\\&=cos(a)(a+O(h^2)) - sin(a)(h+O(h^3))\\&=cos(a)-sin(a)h+O(h^2)\end{align*}\)
因此,\(\frac{d}{dx}cos(x)\mid_{x=a} = -sin(a)\)

    \begin{Verbatim}[commandchars=\\\{\}]
{\color{incolor}In [{\color{incolor}26}]:} \PY{k+kn}{import} \PY{n+nn}{numpy} \PY{k}{as} \PY{n+nn}{np}
         \PY{k+kn}{from} \PY{n+nn}{sympy}\PY{n+nn}{.}\PY{n+nn}{abc} \PY{k}{import} \PY{n}{x}
         
         \PY{n}{f} \PY{o}{=} \PY{k}{lambda} \PY{n}{x}\PY{p}{:} \PY{n}{x}\PY{o}{*}\PY{o}{*}\PY{l+m+mi}{3}\PY{o}{\PYZhy{}}\PY{l+m+mi}{2}\PY{o}{*}\PY{n}{x}\PY{o}{\PYZhy{}}\PY{l+m+mi}{6}
         
         \PY{k}{def} \PY{n+nf}{derivative}\PY{p}{(}\PY{n}{f}\PY{p}{,}\PY{n}{h}\PY{o}{=}\PY{l+m+mf}{0.00001}\PY{p}{)}\PY{p}{:}\PY{c+c1}{\PYZsh{}define the \PYZsq{}derivative\PYZsq{} function}
             \PY{k}{return} \PY{k}{lambda} \PY{n}{x}\PY{p}{:} \PY{n+nb}{float}\PY{p}{(}\PY{n}{f}\PY{p}{(}\PY{n}{x}\PY{o}{+}\PY{n}{h}\PY{p}{)}\PY{o}{\PYZhy{}}\PY{n}{f}\PY{p}{(}\PY{n}{x}\PY{p}{)}\PY{p}{)}\PY{o}{/}\PY{n}{h}
         
         \PY{n}{fprime} \PY{o}{=} \PY{n}{derivative}\PY{p}{(}\PY{n}{f}\PY{p}{)}
         
         \PY{n+nb}{print} \PY{p}{(}\PY{n}{fprime}\PY{p}{(}\PY{l+m+mi}{6}\PY{p}{)}\PY{p}{)}
\end{Verbatim}


    \begin{Verbatim}[commandchars=\\\{\}]
106.0001799942256

    \end{Verbatim}

    \begin{Verbatim}[commandchars=\\\{\}]
{\color{incolor}In [{\color{incolor}29}]:} \PY{c+c1}{\PYZsh{}use sympy\PYZsq{}s defult derivative function}
         \PY{k+kn}{from} \PY{n+nn}{sympy}\PY{n+nn}{.}\PY{n+nn}{abc} \PY{k}{import} \PY{n}{x}
         \PY{n}{f} \PY{o}{=} \PY{n}{x}\PY{o}{*}\PY{o}{*}\PY{l+m+mi}{3}\PY{o}{\PYZhy{}}\PY{l+m+mi}{2}\PY{o}{*}\PY{n}{x}\PY{o}{\PYZhy{}}\PY{l+m+mi}{6}
         \PY{n+nb}{print}\PY{p}{(}\PY{n}{f}\PY{o}{.}\PY{n}{diff}\PY{p}{(}\PY{p}{)}\PY{p}{)}
         \PY{n+nb}{print}\PY{p}{(}\PY{n}{f}\PY{o}{.}\PY{n}{diff}\PY{p}{(}\PY{p}{)}\PY{o}{.}\PY{n}{evalf}\PY{p}{(}\PY{n}{subs}\PY{o}{=}\PY{p}{\PYZob{}}\PY{n}{x}\PY{p}{:}\PY{l+m+mi}{6}\PY{p}{\PYZcb{}}\PY{p}{)}\PY{p}{)}
\end{Verbatim}


    \begin{Verbatim}[commandchars=\\\{\}]
3*x**2 - 2
106.000000000000

    \end{Verbatim}

    \hypertarget{ux7ddaux6027ux8fd1ux4f3clinear-approximation}{%
\subsubsection{線性近似(Linear
approximation)}\label{ux7ddaux6027ux8fd1ux4f3clinear-approximation}}

定義:就是用線性函數去對普通函數進行近似。依據導數的定義三,我們有:\(f(a+h) = f(a) + f'(a)h + O(h^2)\)
如果,我們將高階項去掉,就獲得了\(f(a+h)\)的線性近似式了:\(f(a+h) = \approx f(a) + f'(a)h\)
舉個例子,用線性逼近去估算:\(\begin{align*} \sqrt{255} &= \sqrt {256-1} \approx \sqrt{256} + \frac{1}{2\sqrt{256}(-1)} \\ &=16-\frac{1}{32} \\ &=15 \frac{31}{32} \end{align*}\)

    \hypertarget{ux725bux9813ux8fedux4ee3ux6cd5newtons-method}{%
\subsection{牛頓迭代法(Newton's
Method)}\label{ux725bux9813ux8fedux4ee3ux6cd5newtons-method}}

\textbf{它是一種用於在實數域和複數域上近似求解方程的方法:使用函數\(f(x)\)的泰勒級數的前面幾項來尋找\(f(X)=0\)的根。}
首先,選擇一個接近函數\(f(x)\)零點的\(x_0\),計算對應的函數值\(f(x_0)\)和切線的斜率\(f'(x_0)\);
然後計算切線和\(x\)軸的交點\(x_1\)的\(x\)座標:\$ 0 = (x\_1 -
x\_0)\cdot f'(x\_0) + f(x\_0)\(;<br> 通常來說,\)x\_1\$ 會比 \(x_0\)
更接近方程\(f(X)=0\)的解。因此,
我們現在會利用\(x_1\)去開始新一輪的迭代。公式如下:
\(x_{n+1} = x_n - \frac{f(x_n)}{f'(x_n)}\)

    \begin{Verbatim}[commandchars=\\\{\}]
{\color{incolor}In [{\color{incolor}34}]:} \PY{k+kn}{from} \PY{n+nn}{sympy}\PY{n+nn}{.}\PY{n+nn}{abc} \PY{k}{import} \PY{n}{x}
         
         \PY{k}{def} \PY{n+nf}{mysqrt}\PY{p}{(}\PY{n}{c}\PY{p}{,} \PY{n}{x} \PY{o}{=} \PY{l+m+mi}{1}\PY{p}{,} \PY{n}{maxiter} \PY{o}{=} \PY{l+m+mi}{10}\PY{p}{,} \PY{n}{prt\PYZus{}step} \PY{o}{=} \PY{k+kc}{False}\PY{p}{)}\PY{p}{:}
             \PY{k}{for} \PY{n}{i} \PY{o+ow}{in} \PY{n+nb}{range}\PY{p}{(}\PY{n}{maxiter}\PY{p}{)}\PY{p}{:}
                 \PY{n}{x} \PY{o}{=} \PY{l+m+mf}{0.5}\PY{o}{*}\PY{p}{(}\PY{n}{x}\PY{o}{+} \PY{n}{c}\PY{o}{/}\PY{n}{x}\PY{p}{)}
                 \PY{k}{if} \PY{n}{prt\PYZus{}step} \PY{o}{==} \PY{k+kc}{True}\PY{p}{:}
                     \PY{c+c1}{\PYZsh{} 在输出时,\PYZob{}0\PYZcb{}和\PYZob{}1\PYZcb{}将被i+1和x所替代}
                     \PY{n+nb}{print} \PY{p}{(}\PY{l+s+s2}{\PYZdq{}}\PY{l+s+s2}{After }\PY{l+s+si}{\PYZob{}0\PYZcb{}}\PY{l+s+s2}{ iteration, the root value is updated to }\PY{l+s+si}{\PYZob{}1\PYZcb{}}\PY{l+s+s2}{\PYZdq{}}\PY{o}{.}\PY{n}{format}\PY{p}{(}\PY{n}{i}\PY{o}{+}\PY{l+m+mi}{1}\PY{p}{,}\PY{n}{x}\PY{p}{)}\PY{p}{)}
             \PY{k}{return} \PY{n}{x}
         
         \PY{n+nb}{print} \PY{p}{(}\PY{n}{mysqrt}\PY{p}{(}\PY{l+m+mi}{2}\PY{p}{,}\PY{n}{maxiter} \PY{o}{=}\PY{l+m+mi}{4}\PY{p}{,}\PY{n}{prt\PYZus{}step} \PY{o}{=} \PY{k+kc}{True}\PY{p}{)}\PY{p}{)}
\end{Verbatim}


    \begin{Verbatim}[commandchars=\\\{\}]
After 1 iteration, the root value is updated to 1.5
After 2 iteration, the root value is updated to 1.4166666666666665
After 3 iteration, the root value is updated to 1.4142156862745097
After 4 iteration, the root value is updated to 1.4142135623746899
1.4142135623746899

    \end{Verbatim}

    我們可以通過畫圖,更加了解牛頓法

    \begin{Verbatim}[commandchars=\\\{\}]
{\color{incolor}In [{\color{incolor}35}]:} \PY{k+kn}{import} \PY{n+nn}{numpy} \PY{k}{as} \PY{n+nn}{np}
         \PY{k+kn}{import} \PY{n+nn}{matplotlib}\PY{n+nn}{.}\PY{n+nn}{pyplot} \PY{k}{as} \PY{n+nn}{plt}
         
         \PY{n}{f} \PY{o}{=} \PY{k}{lambda} \PY{n}{x}\PY{p}{:} \PY{n}{x}\PY{o}{*}\PY{o}{*}\PY{l+m+mi}{2}\PY{o}{\PYZhy{}}\PY{l+m+mi}{2}\PY{o}{*}\PY{n}{x}\PY{o}{\PYZhy{}}\PY{l+m+mi}{4}
         \PY{n}{l1} \PY{o}{=} \PY{k}{lambda} \PY{n}{x}\PY{p}{:} \PY{l+m+mi}{2}\PY{o}{*}\PY{n}{x}\PY{o}{\PYZhy{}}\PY{l+m+mi}{8}
         \PY{n}{l2} \PY{o}{=} \PY{k}{lambda} \PY{n}{x}\PY{p}{:} \PY{l+m+mi}{6}\PY{o}{*}\PY{n}{x}\PY{o}{\PYZhy{}}\PY{l+m+mi}{20}
         
         \PY{n}{x} \PY{o}{=} \PY{n}{np}\PY{o}{.}\PY{n}{linspace}\PY{p}{(}\PY{l+m+mi}{0}\PY{p}{,}\PY{l+m+mi}{5}\PY{p}{,}\PY{l+m+mi}{100}\PY{p}{)}
         
         \PY{n}{plt}\PY{o}{.}\PY{n}{plot}\PY{p}{(}\PY{n}{x}\PY{p}{,}\PY{n}{f}\PY{p}{(}\PY{n}{x}\PY{p}{)}\PY{p}{,}\PY{l+s+s1}{\PYZsq{}}\PY{l+s+s1}{black}\PY{l+s+s1}{\PYZsq{}}\PY{p}{)}
         \PY{n}{plt}\PY{o}{.}\PY{n}{plot}\PY{p}{(}\PY{n}{x}\PY{p}{[}\PY{l+m+mi}{30}\PY{p}{:}\PY{l+m+mi}{80}\PY{p}{]}\PY{p}{,}\PY{n}{l1}\PY{p}{(}\PY{n}{x}\PY{p}{[}\PY{l+m+mi}{30}\PY{p}{:}\PY{l+m+mi}{80}\PY{p}{]}\PY{p}{)}\PY{p}{,}\PY{l+s+s1}{\PYZsq{}}\PY{l+s+s1}{blue}\PY{l+s+s1}{\PYZsq{}}\PY{p}{,} \PY{n}{linestyle} \PY{o}{=} \PY{l+s+s1}{\PYZsq{}}\PY{l+s+s1}{\PYZhy{}\PYZhy{}}\PY{l+s+s1}{\PYZsq{}}\PY{p}{)}
         \PY{n}{plt}\PY{o}{.}\PY{n}{plot}\PY{p}{(}\PY{n}{x}\PY{p}{[}\PY{l+m+mi}{66}\PY{p}{:}\PY{p}{]}\PY{p}{,}\PY{n}{l2}\PY{p}{(}\PY{n}{x}\PY{p}{[}\PY{l+m+mi}{66}\PY{p}{:}\PY{p}{]}\PY{p}{)}\PY{p}{,}\PY{l+s+s1}{\PYZsq{}}\PY{l+s+s1}{blue}\PY{l+s+s1}{\PYZsq{}}\PY{p}{,} \PY{n}{linestyle} \PY{o}{=} \PY{l+s+s1}{\PYZsq{}}\PY{l+s+s1}{\PYZhy{}\PYZhy{}}\PY{l+s+s1}{\PYZsq{}}\PY{p}{)}
         
         \PY{n}{l} \PY{o}{=} \PY{n}{plt}\PY{o}{.}\PY{n}{axhline}\PY{p}{(}\PY{n}{y}\PY{o}{=}\PY{l+m+mi}{0}\PY{p}{,}\PY{n}{xmin}\PY{o}{=}\PY{l+m+mi}{0}\PY{p}{,}\PY{n}{xmax}\PY{o}{=}\PY{l+m+mi}{1}\PY{p}{,}\PY{n}{color} \PY{o}{=} \PY{l+s+s1}{\PYZsq{}}\PY{l+s+s1}{black}\PY{l+s+s1}{\PYZsq{}}\PY{p}{)}
         \PY{n}{l} \PY{o}{=} \PY{n}{plt}\PY{o}{.}\PY{n}{axvline}\PY{p}{(}\PY{n}{x}\PY{o}{=}\PY{l+m+mi}{2}\PY{p}{,}\PY{n}{ymin}\PY{o}{=}\PY{l+m+mf}{2.0}\PY{o}{/}\PY{l+m+mi}{18}\PY{p}{,}\PY{n}{ymax}\PY{o}{=}\PY{l+m+mf}{6.0}\PY{o}{/}\PY{l+m+mi}{18}\PY{p}{,} \PY{n}{linestyle} \PY{o}{=} \PY{l+s+s1}{\PYZsq{}}\PY{l+s+s1}{\PYZhy{}\PYZhy{}}\PY{l+s+s1}{\PYZsq{}}\PY{p}{)}
         \PY{n}{l} \PY{o}{=} \PY{n}{plt}\PY{o}{.}\PY{n}{axvline}\PY{p}{(}\PY{n}{x}\PY{o}{=}\PY{l+m+mi}{4}\PY{p}{,}\PY{n}{ymin}\PY{o}{=}\PY{l+m+mf}{6.0}\PY{o}{/}\PY{l+m+mi}{18}\PY{p}{,}\PY{n}{ymax}\PY{o}{=}\PY{l+m+mf}{10.0}\PY{o}{/}\PY{l+m+mi}{18}\PY{p}{,} \PY{n}{linestyle} \PY{o}{=} \PY{l+s+s1}{\PYZsq{}}\PY{l+s+s1}{\PYZhy{}\PYZhy{}}\PY{l+s+s1}{\PYZsq{}}\PY{p}{)}
         
         \PY{n}{plt}\PY{o}{.}\PY{n}{text}\PY{p}{(}\PY{l+m+mf}{1.9}\PY{p}{,}\PY{l+m+mf}{0.5}\PY{p}{,}\PY{l+s+sa}{r}\PY{l+s+s2}{\PYZdq{}}\PY{l+s+s2}{\PYZdl{}x\PYZus{}0\PYZdl{}}\PY{l+s+s2}{\PYZdq{}}\PY{p}{,} \PY{n}{fontsize} \PY{o}{=} \PY{l+m+mi}{18}\PY{p}{)}
         \PY{n}{plt}\PY{o}{.}\PY{n}{text}\PY{p}{(}\PY{l+m+mf}{3.9}\PY{p}{,}\PY{o}{\PYZhy{}}\PY{l+m+mf}{1.5}\PY{p}{,}\PY{l+s+sa}{r}\PY{l+s+s2}{\PYZdq{}}\PY{l+s+s2}{\PYZdl{}x\PYZus{}1\PYZdl{}}\PY{l+s+s2}{\PYZdq{}}\PY{p}{,} \PY{n}{fontsize} \PY{o}{=} \PY{l+m+mi}{18}\PY{p}{)}
         \PY{n}{plt}\PY{o}{.}\PY{n}{text}\PY{p}{(}\PY{l+m+mf}{3.1}\PY{p}{,}\PY{l+m+mf}{1.3}\PY{p}{,}\PY{l+s+sa}{r}\PY{l+s+s2}{\PYZdq{}}\PY{l+s+s2}{\PYZdl{}x\PYZus{}2\PYZdl{}}\PY{l+s+s2}{\PYZdq{}}\PY{p}{,} \PY{n}{fontsize} \PY{o}{=} \PY{l+m+mi}{18}\PY{p}{)}
         
         
         \PY{n}{plt}\PY{o}{.}\PY{n}{plot}\PY{p}{(}\PY{l+m+mi}{2}\PY{p}{,}\PY{l+m+mi}{0}\PY{p}{,}\PY{n}{marker} \PY{o}{=} \PY{l+s+s1}{\PYZsq{}}\PY{l+s+s1}{o}\PY{l+s+s1}{\PYZsq{}}\PY{p}{,} \PY{n}{color} \PY{o}{=} \PY{l+s+s1}{\PYZsq{}}\PY{l+s+s1}{r}\PY{l+s+s1}{\PYZsq{}} \PY{p}{)}
         \PY{n}{plt}\PY{o}{.}\PY{n}{plot}\PY{p}{(}\PY{l+m+mi}{2}\PY{p}{,}\PY{o}{\PYZhy{}}\PY{l+m+mi}{4}\PY{p}{,}\PY{n}{marker} \PY{o}{=} \PY{l+s+s1}{\PYZsq{}}\PY{l+s+s1}{o}\PY{l+s+s1}{\PYZsq{}}\PY{p}{,} \PY{n}{color} \PY{o}{=} \PY{l+s+s1}{\PYZsq{}}\PY{l+s+s1}{r}\PY{l+s+s1}{\PYZsq{}} \PY{p}{)}
         \PY{n}{plt}\PY{o}{.}\PY{n}{plot}\PY{p}{(}\PY{l+m+mi}{4}\PY{p}{,}\PY{l+m+mi}{0}\PY{p}{,}\PY{n}{marker} \PY{o}{=} \PY{l+s+s1}{\PYZsq{}}\PY{l+s+s1}{o}\PY{l+s+s1}{\PYZsq{}}\PY{p}{,} \PY{n}{color} \PY{o}{=} \PY{l+s+s1}{\PYZsq{}}\PY{l+s+s1}{r}\PY{l+s+s1}{\PYZsq{}} \PY{p}{)}
         \PY{n}{plt}\PY{o}{.}\PY{n}{plot}\PY{p}{(}\PY{l+m+mi}{4}\PY{p}{,}\PY{l+m+mi}{4}\PY{p}{,}\PY{n}{marker} \PY{o}{=} \PY{l+s+s1}{\PYZsq{}}\PY{l+s+s1}{o}\PY{l+s+s1}{\PYZsq{}}\PY{p}{,} \PY{n}{color} \PY{o}{=} \PY{l+s+s1}{\PYZsq{}}\PY{l+s+s1}{r}\PY{l+s+s1}{\PYZsq{}} \PY{p}{)}
         \PY{n}{plt}\PY{o}{.}\PY{n}{plot}\PY{p}{(}\PY{l+m+mf}{10.0}\PY{o}{/}\PY{l+m+mi}{3}\PY{p}{,}\PY{l+m+mi}{0}\PY{p}{,}\PY{n}{marker} \PY{o}{=} \PY{l+s+s1}{\PYZsq{}}\PY{l+s+s1}{o}\PY{l+s+s1}{\PYZsq{}}\PY{p}{,} \PY{n}{color} \PY{o}{=} \PY{l+s+s1}{\PYZsq{}}\PY{l+s+s1}{r}\PY{l+s+s1}{\PYZsq{}} \PY{p}{)}
         
         \PY{n}{plt}\PY{o}{.}\PY{n}{show}\PY{p}{(}\PY{p}{)}
\end{Verbatim}


    \begin{center}
    \adjustimage{max size={0.9\linewidth}{0.9\paperheight}}{Use_PY_in_Calculus_files/Use_PY_in_Calculus_58_0.png}
    \end{center}
    { \hspace*{\fill} \\}
    
    下面舉一個例子,\(f(x) = x^2 -2x -4 = 0\)的解,從\(x_0 = 4\)
的初始猜測值開始,找到\(x_0\)的切線:\(y=2x-8\),找到與\(x\)軸的交點\((4,0)\),將此點更新為新解:\(x_1 = 4\),如此循環。

    \begin{Verbatim}[commandchars=\\\{\}]
{\color{incolor}In [{\color{incolor}39}]:} \PY{k}{def} \PY{n+nf}{NewTon}\PY{p}{(}\PY{n}{f}\PY{p}{,} \PY{n}{s} \PY{o}{=} \PY{l+m+mi}{1}\PY{p}{,} \PY{n}{maxiter} \PY{o}{=} \PY{l+m+mi}{100}\PY{p}{,} \PY{n}{prt\PYZus{}step} \PY{o}{=} \PY{k+kc}{False}\PY{p}{)}\PY{p}{:}
             \PY{k}{for} \PY{n}{i} \PY{o+ow}{in} \PY{n+nb}{range}\PY{p}{(}\PY{n}{maxiter}\PY{p}{)}\PY{p}{:}
                 \PY{c+c1}{\PYZsh{} 相较于f.evalf(subs=\PYZob{}x:s\PYZcb{}),subs()是更好的将值带入并计算的方法。}
                 \PY{n}{s} \PY{o}{=} \PY{n}{s} \PY{o}{\PYZhy{}} \PY{n}{f}\PY{o}{.}\PY{n}{subs}\PY{p}{(}\PY{n}{x}\PY{p}{,}\PY{n}{s}\PY{p}{)}\PY{o}{/}\PY{n}{f}\PY{o}{.}\PY{n}{diff}\PY{p}{(}\PY{p}{)}\PY{o}{.}\PY{n}{subs}\PY{p}{(}\PY{n}{x}\PY{p}{,}\PY{n}{s}\PY{p}{)}
                 \PY{k}{if} \PY{n}{prt\PYZus{}step} \PY{o}{==} \PY{k+kc}{True}\PY{p}{:}
                     \PY{n+nb}{print}\PY{p}{(}\PY{l+s+s2}{\PYZdq{}}\PY{l+s+s2}{After }\PY{l+s+si}{\PYZob{}0\PYZcb{}}\PY{l+s+s2}{ iteration, the solution is updated to }\PY{l+s+si}{\PYZob{}1\PYZcb{}}\PY{l+s+s2}{\PYZdq{}}\PY{o}{.}\PY{n}{format}\PY{p}{(}\PY{n}{i}\PY{o}{+}\PY{l+m+mi}{1}\PY{p}{,}\PY{n}{s}\PY{p}{)}\PY{p}{)}
             \PY{k}{return} \PY{n}{s}
         
         \PY{k+kn}{from} \PY{n+nn}{sympy}\PY{n+nn}{.}\PY{n+nn}{abc} \PY{k}{import} \PY{n}{x}
         \PY{n}{f} \PY{o}{=} \PY{n}{x}\PY{o}{*}\PY{o}{*}\PY{l+m+mi}{2}\PY{o}{\PYZhy{}}\PY{l+m+mi}{2}\PY{o}{*}\PY{n}{x}\PY{o}{\PYZhy{}}\PY{l+m+mi}{4}
         \PY{n+nb}{print}\PY{p}{(}\PY{n}{NewTon}\PY{p}{(}\PY{n}{f}\PY{p}{,} \PY{n}{s} \PY{o}{=} \PY{l+m+mi}{2}\PY{p}{,} \PY{n}{maxiter} \PY{o}{=} \PY{l+m+mi}{4}\PY{p}{,} \PY{n}{prt\PYZus{}step} \PY{o}{=} \PY{k+kc}{True}\PY{p}{)}\PY{p}{)}
\end{Verbatim}


    \begin{Verbatim}[commandchars=\\\{\}]
After 1 iteration, the solution is updated to 4
After 2 iteration, the solution is updated to 10/3
After 3 iteration, the solution is updated to 68/21
After 4 iteration, the solution is updated to 3194/987
3194/987

    \end{Verbatim}

    另外,我們可以使用\texttt{sympy},它可以幫助我們運算

    \begin{Verbatim}[commandchars=\\\{\}]
{\color{incolor}In [{\color{incolor}42}]:} \PY{k+kn}{import} \PY{n+nn}{sympy}
         \PY{k+kn}{from} \PY{n+nn}{sympy}\PY{n+nn}{.}\PY{n+nn}{abc} \PY{k}{import} \PY{n}{x}
         \PY{n}{f} \PY{o}{=} \PY{n}{x}\PY{o}{*}\PY{o}{*}\PY{l+m+mi}{2}\PY{o}{\PYZhy{}}\PY{l+m+mi}{2}\PY{o}{*}\PY{n}{x}\PY{o}{\PYZhy{}}\PY{l+m+mi}{4}
         \PY{n+nb}{print}\PY{p}{(}\PY{n}{sympy}\PY{o}{.}\PY{n}{solve}\PY{p}{(}\PY{n}{f}\PY{p}{,}\PY{n}{x}\PY{p}{)}\PY{p}{)}
\end{Verbatim}


    \begin{Verbatim}[commandchars=\\\{\}]
[1 + sqrt(5), -sqrt(5) + 1]

    \end{Verbatim}

    \hypertarget{ux512aux5316}{%
\subsection{優化}\label{ux512aux5316}}

\hypertarget{ux9ad8ux968eux5c0eux6578higher-derivatives}{%
\subsubsection{高階導數(Higher
Derivatives)}\label{ux9ad8ux968eux5c0eux6578higher-derivatives}}

在之前,我們講過什麼是高階導數,這裡在此提及,高階導數的遞歸式的定義為:函數\(f(x)\)的\(n\)階導數\(f^{(n)}(x)\)(或記為\(\frac{d^n}{dx^n}(f)\)為:
\(f^{(n)}(x) = \frac{d}{dx}f^{(n-1}(x)\)
如果將求導\(\frac{d}{dx}\)看作一個運算符,則相當於反覆對運算的結果使用\(n\)次運算符:\((\frac{d}{dx})^n \ f=\frac{d^n}{dx^n}f\)

    \begin{Verbatim}[commandchars=\\\{\}]
{\color{incolor}In [{\color{incolor}22}]:} \PY{k+kn}{from} \PY{n+nn}{sympy}\PY{n+nn}{.}\PY{n+nn}{abc} \PY{k}{import} \PY{n}{x}
         \PY{k+kn}{from} \PY{n+nn}{sympy}\PY{n+nn}{.}\PY{n+nn}{abc} \PY{k}{import} \PY{n}{y}
         \PY{k+kn}{import} \PY{n+nn}{matplotlib}\PY{n+nn}{.}\PY{n+nn}{pyplot} \PY{k}{as} \PY{n+nn}{plt}
         
         \PY{n}{f} \PY{o}{=} \PY{n}{x}\PY{o}{*}\PY{o}{*}\PY{l+m+mi}{2}\PY{o}{*}\PY{n}{y}\PY{o}{\PYZhy{}}\PY{l+m+mi}{2}\PY{o}{*}\PY{n}{x}\PY{o}{*}\PY{n}{y} 
         \PY{n+nb}{print}\PY{p}{(}\PY{n}{f}\PY{o}{.}\PY{n}{diff}\PY{p}{(}\PY{n}{x}\PY{p}{,}\PY{l+m+mi}{2}\PY{p}{)}\PY{p}{)} \PY{c+c1}{\PYZsh{}the second derivatives of x}
         \PY{n+nb}{print}\PY{p}{(}\PY{n}{f}\PY{o}{.}\PY{n}{diff}\PY{p}{(}\PY{n}{x}\PY{p}{)}\PY{o}{.}\PY{n}{diff}\PY{p}{(}\PY{n}{x}\PY{p}{)}\PY{p}{)}\PY{c+c1}{\PYZsh{} the different writing of the second derivatives of x}
         \PY{n+nb}{print}\PY{p}{(}\PY{n}{f}\PY{o}{.}\PY{n}{diff}\PY{p}{(}\PY{n}{x}\PY{p}{,}\PY{n}{y}\PY{p}{)}\PY{p}{)} \PY{c+c1}{\PYZsh{} we first get the derivative of x , then get the derivative of y}
\end{Verbatim}


    \begin{Verbatim}[commandchars=\\\{\}]
2*y
2*y
2*(x - 1)

    \end{Verbatim}

    \hypertarget{ux4f18ux5316ux95eeux9898optimization-problem}{%
\subsubsection{优化问题(Optimization
Problem)}\label{ux4f18ux5316ux95eeux9898optimization-problem}}

在微積分中,優化問題常常指的是算最大面積,最大體積等,現在給出一個例子:

    \begin{Verbatim}[commandchars=\\\{\}]
{\color{incolor}In [{\color{incolor}51}]:} \PY{n}{plt}\PY{o}{.}\PY{n}{figure}\PY{p}{(}\PY{l+m+mi}{1}\PY{p}{,} \PY{n}{figsize}\PY{o}{=}\PY{p}{(}\PY{l+m+mi}{4}\PY{p}{,}\PY{l+m+mi}{4}\PY{p}{)}\PY{p}{)}
         \PY{n}{plt}\PY{o}{.}\PY{n}{axis}\PY{p}{(}\PY{l+s+s1}{\PYZsq{}}\PY{l+s+s1}{off}\PY{l+s+s1}{\PYZsq{}}\PY{p}{)}
         \PY{n}{plt}\PY{o}{.}\PY{n}{axhspan}\PY{p}{(}\PY{l+m+mi}{0}\PY{p}{,}\PY{l+m+mi}{1}\PY{p}{,}\PY{l+m+mf}{0.2}\PY{p}{,}\PY{l+m+mf}{0.8}\PY{p}{,}\PY{n}{ec}\PY{o}{=}\PY{l+s+s2}{\PYZdq{}}\PY{l+s+s2}{none}\PY{l+s+s2}{\PYZdq{}}\PY{p}{)}
         \PY{n}{plt}\PY{o}{.}\PY{n}{axhspan}\PY{p}{(}\PY{l+m+mf}{0.2}\PY{p}{,}\PY{l+m+mf}{0.8}\PY{p}{,}\PY{l+m+mi}{0}\PY{p}{,}\PY{l+m+mf}{0.2}\PY{p}{,}\PY{n}{ec}\PY{o}{=}\PY{l+s+s2}{\PYZdq{}}\PY{l+s+s2}{none}\PY{l+s+s2}{\PYZdq{}}\PY{p}{)}
         \PY{n}{plt}\PY{o}{.}\PY{n}{axhspan}\PY{p}{(}\PY{l+m+mf}{0.2}\PY{p}{,}\PY{l+m+mf}{0.8}\PY{p}{,}\PY{l+m+mf}{0.8}\PY{p}{,}\PY{l+m+mi}{1}\PY{p}{,}\PY{n}{ec}\PY{o}{=}\PY{l+s+s2}{\PYZdq{}}\PY{l+s+s2}{none}\PY{l+s+s2}{\PYZdq{}}\PY{p}{)}
         
         \PY{n}{plt}\PY{o}{.}\PY{n}{axhline}\PY{p}{(}\PY{l+m+mf}{0.2}\PY{p}{,}\PY{l+m+mf}{0.2}\PY{p}{,}\PY{l+m+mf}{0.8}\PY{p}{,}\PY{n}{linewidth} \PY{o}{=} \PY{l+m+mi}{2}\PY{p}{,} \PY{n}{color} \PY{o}{=} \PY{l+s+s1}{\PYZsq{}}\PY{l+s+s1}{black}\PY{l+s+s1}{\PYZsq{}}\PY{p}{)}
         \PY{n}{plt}\PY{o}{.}\PY{n}{axhline}\PY{p}{(}\PY{l+m+mf}{0.8}\PY{p}{,}\PY{l+m+mf}{0.17}\PY{p}{,}\PY{l+m+mf}{0.23}\PY{p}{,}\PY{n}{linewidth} \PY{o}{=} \PY{l+m+mi}{2}\PY{p}{,} \PY{n}{color} \PY{o}{=} \PY{l+s+s1}{\PYZsq{}}\PY{l+s+s1}{black}\PY{l+s+s1}{\PYZsq{}}\PY{p}{)}
         \PY{n}{plt}\PY{o}{.}\PY{n}{axhline}\PY{p}{(}\PY{l+m+mi}{1}\PY{p}{,}\PY{l+m+mf}{0.17}\PY{p}{,}\PY{l+m+mf}{0.23}\PY{p}{,}\PY{n}{linewidth} \PY{o}{=} \PY{l+m+mi}{2}\PY{p}{,} \PY{n}{color} \PY{o}{=} \PY{l+s+s1}{\PYZsq{}}\PY{l+s+s1}{black}\PY{l+s+s1}{\PYZsq{}}\PY{p}{)}
         
         \PY{n}{plt}\PY{o}{.}\PY{n}{axvline}\PY{p}{(}\PY{l+m+mf}{0.2}\PY{p}{,}\PY{l+m+mf}{0.8}\PY{p}{,}\PY{l+m+mi}{1}\PY{p}{,}\PY{n}{linewidth} \PY{o}{=} \PY{l+m+mi}{2}\PY{p}{,} \PY{n}{color} \PY{o}{=} \PY{l+s+s1}{\PYZsq{}}\PY{l+s+s1}{black}\PY{l+s+s1}{\PYZsq{}}\PY{p}{)}
         \PY{n}{plt}\PY{o}{.}\PY{n}{axhline}\PY{p}{(}\PY{l+m+mf}{0.8}\PY{p}{,}\PY{l+m+mf}{0.17}\PY{p}{,}\PY{l+m+mf}{0.23}\PY{p}{,}\PY{n}{linewidth} \PY{o}{=} \PY{l+m+mi}{2}\PY{p}{,} \PY{n}{color} \PY{o}{=} \PY{l+s+s1}{\PYZsq{}}\PY{l+s+s1}{black}\PY{l+s+s1}{\PYZsq{}}\PY{p}{)}
         \PY{n}{plt}\PY{o}{.}\PY{n}{axhline}\PY{p}{(}\PY{l+m+mi}{1}\PY{p}{,}\PY{l+m+mf}{0.17}\PY{p}{,}\PY{l+m+mf}{0.23}\PY{p}{,}\PY{n}{linewidth} \PY{o}{=} \PY{l+m+mi}{2}\PY{p}{,} \PY{n}{color} \PY{o}{=} \PY{l+s+s1}{\PYZsq{}}\PY{l+s+s1}{black}\PY{l+s+s1}{\PYZsq{}}\PY{p}{)}
         
         \PY{n}{plt}\PY{o}{.}\PY{n}{text}\PY{p}{(}\PY{l+m+mf}{0.495}\PY{p}{,}\PY{l+m+mf}{0.22}\PY{p}{,}\PY{l+s+sa}{r}\PY{l+s+s2}{\PYZdq{}}\PY{l+s+s2}{\PYZdl{}l\PYZdl{}}\PY{l+s+s2}{\PYZdq{}}\PY{p}{,}\PY{n}{fontsize} \PY{o}{=} \PY{l+m+mi}{18}\PY{p}{,}\PY{n}{color} \PY{o}{=} \PY{l+s+s2}{\PYZdq{}}\PY{l+s+s2}{black}\PY{l+s+s2}{\PYZdq{}}\PY{p}{)}
         \PY{n}{plt}\PY{o}{.}\PY{n}{text}\PY{p}{(}\PY{l+m+mf}{0.1}\PY{p}{,}\PY{l+m+mf}{0.9}\PY{p}{,}\PY{l+s+sa}{r}\PY{l+s+s2}{\PYZdq{}}\PY{l+s+s2}{\PYZdl{}}\PY{l+s+s2}{\PYZbs{}}\PY{l+s+s2}{frac}\PY{l+s+s2}{\PYZob{}}\PY{l+s+s2}{4\PYZhy{}1\PYZcb{}}\PY{l+s+si}{\PYZob{}2\PYZcb{}}\PY{l+s+s2}{\PYZdl{}}\PY{l+s+s2}{\PYZdq{}}\PY{p}{,}\PY{n}{fontsize} \PY{o}{=} \PY{l+m+mi}{18}\PY{p}{,}\PY{n}{color} \PY{o}{=} \PY{l+s+s2}{\PYZdq{}}\PY{l+s+s2}{black}\PY{l+s+s2}{\PYZdq{}}\PY{p}{)}
         
         \PY{n}{plt}\PY{o}{.}\PY{n}{show}\PY{p}{(}\PY{p}{)}
\end{Verbatim}


    \begin{center}
    \adjustimage{max size={0.9\linewidth}{0.9\paperheight}}{Use_PY_in_Calculus_files/Use_PY_in_Calculus_66_0.png}
    \end{center}
    { \hspace*{\fill} \\}
    
    用一張給定邊長\(4\)的正方形紙來一個沒有蓋的紙盒,設這個紙盒的底部邊長為\(l\),紙盒的高為\(\frac{4-l}{2}\),那麼紙盒的體積為:
\(V(l) = l^2\frac{4-l}{2}\) 我們會希望之道,怎麼樣得到\$ max\{V\_1,
V\_2, \dots V\_n\}\$ ;優化問題就是在滿足條件下,使得目標函數(objective
function)得到最大值(或最小)。

    \begin{Verbatim}[commandchars=\\\{\}]
{\color{incolor}In [{\color{incolor}77}]:} \PY{k+kn}{import} \PY{n+nn}{numpy} \PY{k}{as} \PY{n+nn}{np}
         \PY{k+kn}{import} \PY{n+nn}{matplotlib}\PY{n+nn}{.}\PY{n+nn}{pyplot} \PY{k}{as} \PY{n+nn}{plt}
         
         \PY{n}{l} \PY{o}{=} \PY{n}{np}\PY{o}{.}\PY{n}{linspace}\PY{p}{(}\PY{l+m+mi}{0}\PY{p}{,}\PY{l+m+mi}{4}\PY{p}{,}\PY{l+m+mi}{100}\PY{p}{)}
         \PY{n}{V} \PY{o}{=} \PY{k}{lambda} \PY{n}{l}\PY{p}{:} \PY{l+m+mf}{0.5}\PY{o}{*}\PY{n}{l}\PY{o}{*}\PY{o}{*}\PY{l+m+mi}{2}\PY{o}{*}\PY{p}{(}\PY{l+m+mi}{4}\PY{o}{\PYZhy{}}\PY{n}{l}\PY{p}{)} \PY{c+c1}{\PYZsh{} the \PYZsq{}l\PYZsq{} is the charcter \PYZsq{}l\PYZsq{}, not the number\PYZsq{}one\PYZsq{} as \PYZsq{}1\PYZsq{}}
         \PY{n}{plt}\PY{o}{.}\PY{n}{plot}\PY{p}{(}\PY{n}{l}\PY{p}{,}\PY{n}{V}\PY{p}{(}\PY{n}{l}\PY{p}{)}\PY{p}{)}
         \PY{n}{plt}\PY{o}{.}\PY{n}{vlines}\PY{p}{(}\PY{l+m+mf}{2.7}\PY{p}{,}\PY{l+m+mi}{0}\PY{p}{,}\PY{l+m+mi}{5}\PY{p}{,} \PY{n}{colors} \PY{o}{=} \PY{l+s+s2}{\PYZdq{}}\PY{l+s+s2}{c}\PY{l+s+s2}{\PYZdq{}}\PY{p}{,} \PY{n}{linestyles} \PY{o}{=} \PY{l+s+s2}{\PYZdq{}}\PY{l+s+s2}{dashed}\PY{l+s+s2}{\PYZdq{}}\PY{p}{)}
         \PY{n}{plt}\PY{o}{.}\PY{n}{show}\PY{p}{(}\PY{p}{)}
\end{Verbatim}


    \begin{center}
    \adjustimage{max size={0.9\linewidth}{0.9\paperheight}}{Use_PY_in_Calculus_files/Use_PY_in_Calculus_68_0.png}
    \end{center}
    { \hspace*{\fill} \\}
    
    通過觀察可得,在\(l\)的值略大於\(2.5\)的位置(虛線),獲得最大體積。

    \hypertarget{ux95dcux9375ux9edecritical-points}{%
\subsubsection{關鍵點(Critical
Points)}\label{ux95dcux9375ux9edecritical-points}}

通過導數一節,我們知道一個函數在某一處的導數是代表了在輸入後函數值所發生的相對應的變化。
因此,如果在給定一個函數\(f\),如果知道點\(x=a\)處函數的導數不為\(0\),則在該點處稍微改變函數的輸入值,函數值會發生變化,這表明函數在該點的函數值,既不是局部最大值(local
maximum),也不是局部最小值(local
minimum);相反,如果函數\(f\)在點\(x=a\)處函數的導數為\(0\),或者該點出的導數不存在則稱這個點為關鍵點(critical
Plints)
要想知道一個\(f'(a)=0\)的關鍵處,函數值\(f(a)\)是一個局部最大值還是局部最小值,可以使用二次導數測試:
1. 如果 \(f''(a) > 0\), 則函數\(f\)在\(a\)處的函數值是局部最小值; 2.
如果 \(f''(a) < 0\), 則函數\(f\)在\(a\)處的函數值是局部最大值; 3. 如果
\(f''(a) = 0\), 則無結論。
二次函數測試在中學課本中,大多是要求不求甚解地記憶的規則,其實理解起來非常容易。二次導數測試中涉及到函數在某一點處的函數值、一次導數和二次導數,於是我們可以利用泰勒級數:\(f(x)\)在\(x=a\)的泰勒級數:
\(f(x) = f(a) + f'(a)(x-a) + \frac{1}{2}f''(a)(x-a)^2 + \dots\)
因為\(a\)是關鍵點,\(f'(a)\) = 0,
因而:\(f(x) = f(a) + \frac{1}{2}f''(a)(x-a)^2 + O(x^3)\)
表明\(f''(a) \neq 0\)時,函數\(f(x)\)在\(x=a\)附近的表現近似於二次函數,二次項的係數\(\frac{1}{2}f''(a)\)決定了函數值在該點的表現。
回到剛才那題:求最大體積,現在,我們就可以求了:

    \begin{Verbatim}[commandchars=\\\{\}]
{\color{incolor}In [{\color{incolor}86}]:} \PY{k+kn}{import} \PY{n+nn}{sympy}
         \PY{k+kn}{from} \PY{n+nn}{sympy}\PY{n+nn}{.}\PY{n+nn}{abc} \PY{k}{import} \PY{n}{l}
         \PY{n}{V} \PY{o}{=} \PY{l+m+mf}{0.5}\PY{o}{*}\PY{n}{l}\PY{o}{*}\PY{o}{*}\PY{l+m+mi}{2}\PY{o}{*}\PY{p}{(}\PY{l+m+mi}{4}\PY{o}{\PYZhy{}}\PY{n}{l}\PY{p}{)}
         \PY{c+c1}{\PYZsh{} first derivative}
         \PY{n+nb}{print}\PY{p}{(}\PY{n}{V}\PY{o}{.}\PY{n}{diff}\PY{p}{(}\PY{n}{l}\PY{p}{)}\PY{p}{)}
         \PY{c+c1}{\PYZsh{} the domain of first derivative is (\PYZhy{}oo,oo),so, the critical point is the root of V\PYZsq{}(1) = 0}
         \PY{n}{cp} \PY{o}{=} \PY{n}{sympy}\PY{o}{.}\PY{n}{solve}\PY{p}{(}\PY{n}{V}\PY{o}{.}\PY{n}{diff}\PY{p}{(}\PY{n}{l}\PY{p}{)}\PY{p}{,}\PY{n}{l}\PY{p}{)}
         \PY{n+nb}{print}\PY{p}{(}\PY{n+nb}{str}\PY{p}{(}\PY{n}{cp}\PY{p}{)}\PY{p}{)}
         \PY{c+c1}{\PYZsh{}after finding out the critical point, we can calculate the second derivative}
         \PY{k}{for} \PY{n}{p} \PY{o+ow}{in} \PY{n}{cp}\PY{p}{:}
             \PY{n+nb}{print}\PY{p}{(}\PY{n+nb}{int}\PY{p}{(}\PY{n}{V}\PY{o}{.}\PY{n}{diff}\PY{p}{(}\PY{n}{l}\PY{p}{,}\PY{l+m+mi}{2}\PY{p}{)}\PY{o}{.}\PY{n}{subs}\PY{p}{(}\PY{n}{l}\PY{p}{,}\PY{n}{p}\PY{p}{)}\PY{p}{)}\PY{p}{)}
         \PY{c+c1}{\PYZsh{} known that whenl=2.666..., we get the maximum V}
\end{Verbatim}


    \begin{Verbatim}[commandchars=\\\{\}]
-0.5*l**2 + 1.0*l*(-l + 4)
[0.0, 2.66666666666667]
4
-4

    \end{Verbatim}

    \hypertarget{ux7ddaux6027ux8ff4ux6b78linear-regression}{%
\subsubsection{線性迴歸(Linear
Regression)}\label{ux7ddaux6027ux8ff4ux6b78linear-regression}}

二維平面上有\(n\)個數據點,\(p_i = (x_i,y_i)\),現在嘗試找到一條經過原點的直線\(y=ax\),使得所有數據點到該直線的殘差(數據點和回歸直線之間的水平距離)的平方和最小。

    \begin{Verbatim}[commandchars=\\\{\}]
{\color{incolor}In [{\color{incolor}87}]:} \PY{k+kn}{import} \PY{n+nn}{numpy} \PY{k}{as} \PY{n+nn}{np}
         \PY{k+kn}{import} \PY{n+nn}{matplotlib}\PY{n+nn}{.}\PY{n+nn}{pyplot} \PY{k}{as} \PY{n+nn}{plt}
         
         \PY{c+c1}{\PYZsh{} Set seed of random function to ensure reproducibility of simulation data}
         \PY{n}{np}\PY{o}{.}\PY{n}{random}\PY{o}{.}\PY{n}{seed}\PY{p}{(}\PY{l+m+mi}{123}\PY{p}{)}
         
         \PY{c+c1}{\PYZsh{} Randomly generate some data with errors}
         \PY{n}{x} \PY{o}{=} \PY{n}{np}\PY{o}{.}\PY{n}{linspace}\PY{p}{(}\PY{l+m+mi}{0}\PY{p}{,}\PY{l+m+mi}{10}\PY{p}{,}\PY{l+m+mi}{10}\PY{p}{)}
         \PY{n}{res} \PY{o}{=} \PY{n}{np}\PY{o}{.}\PY{n}{random}\PY{o}{.}\PY{n}{randint}\PY{p}{(}\PY{o}{\PYZhy{}}\PY{l+m+mi}{5}\PY{p}{,}\PY{l+m+mi}{5}\PY{p}{,}\PY{l+m+mi}{10}\PY{p}{)}
         \PY{n}{y} \PY{o}{=} \PY{l+m+mi}{3}\PY{o}{*}\PY{n}{x} \PY{o}{+} \PY{n}{res}
         
         \PY{c+c1}{\PYZsh{} Solve the coefficient of the regression line}
         \PY{n}{a} \PY{o}{=} \PY{n+nb}{sum}\PY{p}{(}\PY{n}{x}\PY{o}{*}\PY{n}{y}\PY{p}{)}\PY{o}{/}\PY{n+nb}{sum}\PY{p}{(}\PY{n}{x}\PY{o}{*}\PY{o}{*}\PY{l+m+mi}{2}\PY{p}{)}
         
         \PY{c+c1}{\PYZsh{} 绘图}
         \PY{n}{plt}\PY{o}{.}\PY{n}{plot}\PY{p}{(}\PY{n}{x}\PY{p}{,}\PY{n}{y}\PY{p}{,}\PY{l+s+s1}{\PYZsq{}}\PY{l+s+s1}{o}\PY{l+s+s1}{\PYZsq{}}\PY{p}{)}
         \PY{n}{plt}\PY{o}{.}\PY{n}{plot}\PY{p}{(}\PY{n}{x}\PY{p}{,}\PY{n}{a}\PY{o}{*}\PY{n}{x}\PY{p}{,}\PY{l+s+s1}{\PYZsq{}}\PY{l+s+s1}{red}\PY{l+s+s1}{\PYZsq{}}\PY{p}{)}
         \PY{k}{for} \PY{n}{i} \PY{o+ow}{in} \PY{n+nb}{range}\PY{p}{(}\PY{n+nb}{len}\PY{p}{(}\PY{n}{x}\PY{p}{)}\PY{p}{)}\PY{p}{:}
             \PY{n}{plt}\PY{o}{.}\PY{n}{axvline}\PY{p}{(}\PY{n}{x}\PY{p}{[}\PY{n}{i}\PY{p}{]}\PY{p}{,}\PY{n+nb}{min}\PY{p}{(}\PY{p}{(}\PY{n}{a}\PY{o}{*}\PY{n}{x}\PY{p}{[}\PY{n}{i}\PY{p}{]}\PY{o}{+}\PY{l+m+mi}{5}\PY{p}{)}\PY{o}{/}\PY{l+m+mf}{35.0}\PY{p}{,}\PY{p}{(}\PY{n}{y}\PY{p}{[}\PY{n}{i}\PY{p}{]}\PY{o}{+}\PY{l+m+mi}{5}\PY{p}{)}\PY{o}{/}\PY{l+m+mf}{35.0}\PY{p}{)}\PY{p}{,}\PYZbs{}
                  \PY{n+nb}{max}\PY{p}{(}\PY{p}{(}\PY{n}{a}\PY{o}{*}\PY{n}{x}\PY{p}{[}\PY{n}{i}\PY{p}{]}\PY{o}{+}\PY{l+m+mi}{5}\PY{p}{)}\PY{o}{/}\PY{l+m+mf}{35.0}\PY{p}{,}\PY{p}{(}\PY{n}{y}\PY{p}{[}\PY{n}{i}\PY{p}{]}\PY{o}{+}\PY{l+m+mi}{5}\PY{p}{)}\PY{o}{/}\PY{l+m+mf}{35.0}\PY{p}{)}\PY{p}{,}\PY{n}{linestyle} \PY{o}{=} \PY{l+s+s1}{\PYZsq{}}\PY{l+s+s1}{\PYZhy{}\PYZhy{}}\PY{l+s+s1}{\PYZsq{}}\PY{p}{,}\PYZbs{}
                  \PY{n}{color} \PY{o}{=} \PY{l+s+s1}{\PYZsq{}}\PY{l+s+s1}{black}\PY{l+s+s1}{\PYZsq{}}\PY{p}{)}
         
         \PY{n}{plt}\PY{o}{.}\PY{n}{show}\PY{p}{(}\PY{p}{)}
\end{Verbatim}


    \begin{center}
    \adjustimage{max size={0.9\linewidth}{0.9\paperheight}}{Use_PY_in_Calculus_files/Use_PY_in_Calculus_73_0.png}
    \end{center}
    { \hspace*{\fill} \\}
    
    要找到這樣一條直線,實際上是一個優化問題:
\(\min_a Err(a) = \sum_i(y_i - ax_i)^2\)
要找出函數\(Err(a)\)的最小值,首先計算一次導函數:\(\frac{dErr}{da} = \sum_i 2(y_i-ax_i)(-x_i)\),因此,\(a = \frac{\sum_i x_iy_i}{\sum_i x_i^2}\)
是能夠使得函數值最小的輸入。
這也是上面\texttt{python}代碼中,求解回歸線斜率所用的計算方式。
如果,我們不限定直線一定經過原點,即,\(y=ax+b\),則變量變成兩個:\(a\)和\(b\):
\(\min_a Err(a,b) = \sum_i(y_i - ax_i-b)^2\)
這個問題就是多元微積分中所要分析的問題了,這裡給出一種\texttt{python}中的解法:

    \begin{Verbatim}[commandchars=\\\{\}]
{\color{incolor}In [{\color{incolor}92}]:} \PY{k+kn}{import} \PY{n+nn}{numpy} \PY{k}{as} \PY{n+nn}{np}
         \PY{k+kn}{import} \PY{n+nn}{matplotlib}\PY{n+nn}{.}\PY{n+nn}{pyplot} \PY{k}{as} \PY{n+nn}{plt}
         
         \PY{c+c1}{\PYZsh{} 设定好随机函数种子,确保模拟数据的可重现性}
         \PY{n}{np}\PY{o}{.}\PY{n}{random}\PY{o}{.}\PY{n}{seed}\PY{p}{(}\PY{l+m+mi}{123}\PY{p}{)}
         
         \PY{c+c1}{\PYZsh{} 随机生成一些带误差的数据}
         \PY{n}{x} \PY{o}{=} \PY{n}{np}\PY{o}{.}\PY{n}{linspace}\PY{p}{(}\PY{l+m+mi}{0}\PY{p}{,}\PY{l+m+mi}{10}\PY{p}{,}\PY{l+m+mi}{10}\PY{p}{)}
         \PY{n}{res} \PY{o}{=} \PY{n}{np}\PY{o}{.}\PY{n}{random}\PY{o}{.}\PY{n}{randint}\PY{p}{(}\PY{o}{\PYZhy{}}\PY{l+m+mi}{5}\PY{p}{,}\PY{l+m+mi}{5}\PY{p}{,}\PY{l+m+mi}{10}\PY{p}{)}
         \PY{n}{y} \PY{o}{=} \PY{l+m+mi}{3}\PY{o}{*}\PY{n}{x} \PY{o}{+} \PY{n}{res}
         
         \PY{c+c1}{\PYZsh{} 求解回归线的系数}
         \PY{n}{a} \PY{o}{=} \PY{n+nb}{sum}\PY{p}{(}\PY{n}{x}\PY{o}{*}\PY{n}{y}\PY{p}{)}\PY{o}{/}\PY{n+nb}{sum}\PY{p}{(}\PY{n}{x}\PY{o}{*}\PY{o}{*}\PY{l+m+mi}{2}\PY{p}{)}
         
         \PY{n}{slope}\PY{p}{,} \PY{n}{intercept} \PY{o}{=} \PY{n}{np}\PY{o}{.}\PY{n}{polyfit}\PY{p}{(}\PY{n}{x}\PY{p}{,}\PY{n}{y}\PY{p}{,}\PY{l+m+mi}{1}\PY{p}{)}
         
         \PY{c+c1}{\PYZsh{} 绘图}
         \PY{n}{plt}\PY{o}{.}\PY{n}{plot}\PY{p}{(}\PY{n}{x}\PY{p}{,}\PY{n}{y}\PY{p}{,}\PY{l+s+s1}{\PYZsq{}}\PY{l+s+s1}{o}\PY{l+s+s1}{\PYZsq{}}\PY{p}{)}
         \PY{n}{plt}\PY{o}{.}\PY{n}{plot}\PY{p}{(}\PY{n}{x}\PY{p}{,}\PY{n}{a}\PY{o}{*}\PY{n}{x}\PY{p}{,}\PY{l+s+s1}{\PYZsq{}}\PY{l+s+s1}{red}\PY{l+s+s1}{\PYZsq{}}\PY{p}{,}\PY{n}{linestyle}\PY{o}{=}\PY{l+s+s1}{\PYZsq{}}\PY{l+s+s1}{\PYZhy{}\PYZhy{}}\PY{l+s+s1}{\PYZsq{}}\PY{p}{)}
         \PY{n}{plt}\PY{o}{.}\PY{n}{plot}\PY{p}{(}\PY{n}{x}\PY{p}{,}\PY{n}{slope}\PY{o}{*}\PY{n}{x}\PY{o}{+}\PY{n}{intercept}\PY{p}{,} \PY{l+s+s1}{\PYZsq{}}\PY{l+s+s1}{blue}\PY{l+s+s1}{\PYZsq{}}\PY{p}{)}
         \PY{k}{for} \PY{n}{i} \PY{o+ow}{in} \PY{n+nb}{range}\PY{p}{(}\PY{n+nb}{len}\PY{p}{(}\PY{n}{x}\PY{p}{)}\PY{p}{)}\PY{p}{:}
             \PY{n}{plt}\PY{o}{.}\PY{n}{axvline}\PY{p}{(}\PY{n}{x}\PY{p}{[}\PY{n}{i}\PY{p}{]}\PY{p}{,}\PY{n+nb}{min}\PY{p}{(}\PY{p}{(}\PY{n}{a}\PY{o}{*}\PY{n}{x}\PY{p}{[}\PY{n}{i}\PY{p}{]}\PY{o}{+}\PY{l+m+mi}{5}\PY{p}{)}\PY{o}{/}\PY{l+m+mf}{35.0}\PY{p}{,}\PY{p}{(}\PY{n}{y}\PY{p}{[}\PY{n}{i}\PY{p}{]}\PY{o}{+}\PY{l+m+mi}{5}\PY{p}{)}\PY{o}{/}\PY{l+m+mf}{35.0}\PY{p}{)}\PY{p}{,}\PYZbs{}
                  \PY{n+nb}{max}\PY{p}{(}\PY{p}{(}\PY{n}{a}\PY{o}{*}\PY{n}{x}\PY{p}{[}\PY{n}{i}\PY{p}{]}\PY{o}{+}\PY{l+m+mi}{5}\PY{p}{)}\PY{o}{/}\PY{l+m+mf}{35.0}\PY{p}{,}\PY{p}{(}\PY{n}{y}\PY{p}{[}\PY{n}{i}\PY{p}{]}\PY{o}{+}\PY{l+m+mi}{5}\PY{p}{)}\PY{o}{/}\PY{l+m+mf}{35.0}\PY{p}{)}\PY{p}{,}\PY{n}{linestyle} \PY{o}{=} \PY{l+s+s1}{\PYZsq{}}\PY{l+s+s1}{\PYZhy{}\PYZhy{}}\PY{l+s+s1}{\PYZsq{}}\PY{p}{,}\PYZbs{}
                  \PY{n}{color} \PY{o}{=} \PY{l+s+s1}{\PYZsq{}}\PY{l+s+s1}{black}\PY{l+s+s1}{\PYZsq{}}\PY{p}{)}
         
         \PY{n}{plt}\PY{o}{.}\PY{n}{show}\PY{p}{(}\PY{p}{)}
\end{Verbatim}


    \begin{center}
    \adjustimage{max size={0.9\linewidth}{0.9\paperheight}}{Use_PY_in_Calculus_files/Use_PY_in_Calculus_75_0.png}
    \end{center}
    { \hspace*{\fill} \\}
    
    \hypertarget{ux7a4dux5206ux8207ux5faeux5206integration-and-differentiation}{%
\subsection{積分與微分(Integration and
Differentiation)}\label{ux7a4dux5206ux8207ux5faeux5206integration-and-differentiation}}

\hypertarget{ux7a4dux5206}{%
\subsubsection{積分}\label{ux7a4dux5206}}

積分時微積分中一個一個核心概念,通常會分為\textbf{定積分和不定積分}兩種。

    \hypertarget{ux5b9aux7a4dux5206integral}{%
\paragraph{定積分(Integral)}\label{ux5b9aux7a4dux5206integral}}

也被稱為\textbf{黎曼積分(Riemann
integral)},直觀地說,對於一個給定的正實數值函數\(f(x)\),\(f(x)\)在一個實數區間\([a,b]\)上的定積分:\(\int_a^b f(x) dx\)
可以理解成在\(O-xy\)坐標平面上,由曲線\((x,f(x))\),直線\(x=a, x=b\)以及\(x\)軸圍成的面積。

    \begin{Verbatim}[commandchars=\\\{\}]
{\color{incolor}In [{\color{incolor}117}]:} \PY{n}{x} \PY{o}{=} \PY{n}{np}\PY{o}{.}\PY{n}{linspace}\PY{p}{(}\PY{l+m+mi}{0}\PY{p}{,} \PY{l+m+mi}{5}\PY{p}{,} \PY{l+m+mi}{100}\PY{p}{)}
          \PY{n}{y} \PY{o}{=}  \PY{n}{np}\PY{o}{.}\PY{n}{sqrt}\PY{p}{(}\PY{n}{x}\PY{p}{)}
          
          \PY{n}{plt}\PY{o}{.}\PY{n}{plot}\PY{p}{(}\PY{n}{x}\PY{p}{,} \PY{n}{y}\PY{p}{)}
          \PY{n}{plt}\PY{o}{.}\PY{n}{fill\PYZus{}between}\PY{p}{(}\PY{n}{x}\PY{p}{,} \PY{n}{y}\PY{p}{,} \PY{n}{interpolate}\PY{o}{=}\PY{k+kc}{True}\PY{p}{,} \PY{n}{color}\PY{o}{=}\PY{l+s+s1}{\PYZsq{}}\PY{l+s+s1}{b}\PY{l+s+s1}{\PYZsq{}}\PY{p}{,} \PY{n}{alpha}\PY{o}{=}\PY{l+m+mf}{0.5}\PY{p}{)}
          \PY{n}{plt}\PY{o}{.}\PY{n}{xlim}\PY{p}{(}\PY{l+m+mi}{0}\PY{p}{,}\PY{l+m+mi}{5}\PY{p}{)}
          \PY{n}{plt}\PY{o}{.}\PY{n}{ylim}\PY{p}{(}\PY{l+m+mi}{0}\PY{p}{,}\PY{l+m+mi}{5}\PY{p}{)}
          
          \PY{n}{plt}\PY{o}{.}\PY{n}{show}\PY{p}{(}\PY{p}{)}
\end{Verbatim}


    \begin{center}
    \adjustimage{max size={0.9\linewidth}{0.9\paperheight}}{Use_PY_in_Calculus_files/Use_PY_in_Calculus_78_0.png}
    \end{center}
    { \hspace*{\fill} \\}
    
    \textbf{黎曼積分}的核心思想就是試圖通過無限逼近來確定這個積分值。同時請注意,如果\(f(x)\)取負值,則相應的面積值\(S\)也取負值。這裡不給出詳細的證明和分析。不太嚴格的講,黎曼積分就是當分割的月來月``精細''的時候,黎曼河去想的極限。下面的圖就是展示,如何通過``矩形逼近''來證明。(這裡不提及勒貝格積分
Lebesgue integral)

    \begin{Verbatim}[commandchars=\\\{\}]
{\color{incolor}In [{\color{incolor}111}]:} \PY{k+kn}{import} \PY{n+nn}{numpy} \PY{k}{as} \PY{n+nn}{np}
          \PY{k+kn}{import} \PY{n+nn}{matplotlib}\PY{n+nn}{.}\PY{n+nn}{pyplot} \PY{k}{as} \PY{n+nn}{plt}
          \PY{k+kn}{import} \PY{n+nn}{matplotlib}\PY{n+nn}{.}\PY{n+nn}{ticker} \PY{k}{as} \PY{n+nn}{ticker}
          
          \PY{k}{def} \PY{n+nf}{func}\PY{p}{(}\PY{n}{x}\PY{p}{)}\PY{p}{:}
              \PY{k}{return} \PY{o}{\PYZhy{}}\PY{n}{x}\PY{o}{*}\PY{o}{*}\PY{l+m+mi}{3} \PY{o}{\PYZhy{}} \PY{n}{x}\PY{o}{*}\PY{o}{*}\PY{l+m+mi}{2} \PY{o}{+} \PY{l+m+mi}{5}
          
          \PY{n}{a}\PY{p}{,} \PY{n}{b} \PY{o}{=} \PY{l+m+mi}{2}\PY{p}{,} \PY{l+m+mi}{9}  \PY{c+c1}{\PYZsh{} integral limits}
          \PY{n}{x} \PY{o}{=} \PY{n}{np}\PY{o}{.}\PY{n}{linspace}\PY{p}{(}\PY{o}{\PYZhy{}}\PY{l+m+mi}{5}\PY{p}{,} \PY{l+m+mi}{5}\PY{p}{)}
          \PY{n}{y} \PY{o}{=} \PY{n}{func}\PY{p}{(}\PY{n}{x}\PY{p}{)}
          \PY{n}{ix} \PY{o}{=} \PY{n}{np}\PY{o}{.}\PY{n}{linspace}\PY{p}{(}\PY{o}{\PYZhy{}}\PY{l+m+mi}{5}\PY{p}{,} \PY{l+m+mi}{5}\PY{p}{,}\PY{l+m+mi}{10}\PY{p}{)}
          \PY{n}{iy} \PY{o}{=} \PY{n}{func}\PY{p}{(}\PY{n}{ix}\PY{p}{)}
          
          \PY{n}{fig}\PY{p}{,} \PY{n}{ax} \PY{o}{=} \PY{n}{plt}\PY{o}{.}\PY{n}{subplots}\PY{p}{(}\PY{p}{)}
          \PY{n}{plt}\PY{o}{.}\PY{n}{plot}\PY{p}{(}\PY{n}{x}\PY{p}{,} \PY{n}{y}\PY{p}{,} \PY{l+s+s1}{\PYZsq{}}\PY{l+s+s1}{r}\PY{l+s+s1}{\PYZsq{}}\PY{p}{,} \PY{n}{linewidth}\PY{o}{=}\PY{l+m+mi}{2}\PY{p}{,} \PY{n}{zorder}\PY{o}{=}\PY{l+m+mi}{5}\PY{p}{)}
          \PY{n}{plt}\PY{o}{.}\PY{n}{bar}\PY{p}{(}\PY{n}{ix}\PY{p}{,} \PY{n}{iy}\PY{p}{,} \PY{n}{width}\PY{o}{=}\PY{l+m+mf}{1.1}\PY{p}{,} \PY{n}{color}\PY{o}{=}\PY{l+s+s1}{\PYZsq{}}\PY{l+s+s1}{b}\PY{l+s+s1}{\PYZsq{}}\PY{p}{,} \PY{n}{align}\PY{o}{=}\PY{l+s+s1}{\PYZsq{}}\PY{l+s+s1}{edge}\PY{l+s+s1}{\PYZsq{}}\PY{p}{,} \PY{n}{ec}\PY{o}{=}\PY{l+s+s1}{\PYZsq{}}\PY{l+s+s1}{olive}\PY{l+s+s1}{\PYZsq{}}\PY{p}{,} \PY{n}{ls}\PY{o}{=}\PY{l+s+s1}{\PYZsq{}}\PY{l+s+s1}{\PYZhy{}}\PY{l+s+s1}{\PYZsq{}}\PY{p}{,} \PY{n}{lw}\PY{o}{=}\PY{l+m+mi}{2}\PY{p}{,}\PY{n}{zorder}\PY{o}{=}\PY{l+m+mi}{5}\PY{p}{)}
          
          \PY{n}{plt}\PY{o}{.}\PY{n}{figtext}\PY{p}{(}\PY{l+m+mf}{0.9}\PY{p}{,} \PY{l+m+mf}{0.05}\PY{p}{,} \PY{l+s+s1}{\PYZsq{}}\PY{l+s+s1}{\PYZdl{}x\PYZdl{}}\PY{l+s+s1}{\PYZsq{}}\PY{p}{)}
          \PY{n}{plt}\PY{o}{.}\PY{n}{figtext}\PY{p}{(}\PY{l+m+mf}{0.1}\PY{p}{,} \PY{l+m+mf}{0.9}\PY{p}{,} \PY{l+s+s1}{\PYZsq{}}\PY{l+s+s1}{\PYZdl{}y\PYZdl{}}\PY{l+s+s1}{\PYZsq{}}\PY{p}{)}
          
          \PY{n}{ax}\PY{o}{.}\PY{n}{spines}\PY{p}{[}\PY{l+s+s1}{\PYZsq{}}\PY{l+s+s1}{left}\PY{l+s+s1}{\PYZsq{}}\PY{p}{]}\PY{o}{.}\PY{n}{set\PYZus{}visible}\PY{p}{(}\PY{k+kc}{True}\PY{p}{)}
          \PY{n}{ax}\PY{o}{.}\PY{n}{spines}\PY{p}{[}\PY{l+s+s1}{\PYZsq{}}\PY{l+s+s1}{right}\PY{l+s+s1}{\PYZsq{}}\PY{p}{]}\PY{o}{.}\PY{n}{set\PYZus{}visible}\PY{p}{(}\PY{k+kc}{True}\PY{p}{)}
          \PY{n}{ax}\PY{o}{.}\PY{n}{xaxis}\PY{o}{.}\PY{n}{set\PYZus{}major\PYZus{}locator}\PY{p}{(}\PY{n}{ticker}\PY{o}{.}\PY{n}{IndexLocator}\PY{p}{(}\PY{n}{base}\PY{o}{=}\PY{l+m+mi}{1}\PY{p}{,} \PY{n}{offset}\PY{o}{=}\PY{l+m+mi}{0}\PY{p}{)}\PY{p}{)}
          \PY{n}{plt}\PY{o}{.}\PY{n}{xlim}\PY{p}{(}\PY{o}{\PYZhy{}}\PY{l+m+mi}{6}\PY{p}{,}\PY{l+m+mi}{6}\PY{p}{)}
          \PY{n}{plt}\PY{o}{.}\PY{n}{ylim}\PY{p}{(}\PY{o}{\PYZhy{}}\PY{l+m+mi}{100}\PY{p}{,}\PY{l+m+mi}{100}\PY{p}{)}
          
          \PY{n}{plt}\PY{o}{.}\PY{n}{show}\PY{p}{(}\PY{p}{)}
\end{Verbatim}


    \begin{center}
    \adjustimage{max size={0.9\linewidth}{0.9\paperheight}}{Use_PY_in_Calculus_files/Use_PY_in_Calculus_80_0.png}
    \end{center}
    { \hspace*{\fill} \\}
    
    \begin{Verbatim}[commandchars=\\\{\}]
{\color{incolor}In [{\color{incolor}112}]:} \PY{k+kn}{import} \PY{n+nn}{numpy} \PY{k}{as} \PY{n+nn}{np}
          \PY{k+kn}{import} \PY{n+nn}{matplotlib}\PY{n+nn}{.}\PY{n+nn}{pyplot} \PY{k}{as} \PY{n+nn}{plt}
          \PY{k+kn}{import} \PY{n+nn}{matplotlib}\PY{n+nn}{.}\PY{n+nn}{ticker} \PY{k}{as} \PY{n+nn}{ticker}
          
          \PY{k}{def} \PY{n+nf}{func}\PY{p}{(}\PY{n}{x}\PY{p}{)}\PY{p}{:}
              
              \PY{k}{return} \PY{o}{\PYZhy{}}\PY{n}{x}\PY{o}{*}\PY{o}{*}\PY{l+m+mi}{3} \PY{o}{\PYZhy{}} \PY{n}{x}\PY{o}{*}\PY{o}{*}\PY{l+m+mi}{2} \PY{o}{+} \PY{l+m+mi}{5}
          
          \PY{n}{a}\PY{p}{,} \PY{n}{b} \PY{o}{=} \PY{l+m+mi}{2}\PY{p}{,} \PY{l+m+mi}{9}  \PY{c+c1}{\PYZsh{} integral limits}
          \PY{n}{x} \PY{o}{=} \PY{n}{np}\PY{o}{.}\PY{n}{linspace}\PY{p}{(}\PY{o}{\PYZhy{}}\PY{l+m+mi}{5}\PY{p}{,} \PY{l+m+mi}{5}\PY{p}{)}
          \PY{n}{y} \PY{o}{=} \PY{n}{func}\PY{p}{(}\PY{n}{x}\PY{p}{)}
          \PY{n}{ix} \PY{o}{=} \PY{n}{np}\PY{o}{.}\PY{n}{linspace}\PY{p}{(}\PY{o}{\PYZhy{}}\PY{l+m+mi}{5}\PY{p}{,} \PY{l+m+mi}{5}\PY{p}{,}\PY{l+m+mi}{20}\PY{p}{)}
          \PY{n}{iy} \PY{o}{=} \PY{n}{func}\PY{p}{(}\PY{n}{ix}\PY{p}{)}
          
          \PY{n}{fig}\PY{p}{,} \PY{n}{ax} \PY{o}{=} \PY{n}{plt}\PY{o}{.}\PY{n}{subplots}\PY{p}{(}\PY{p}{)}
          \PY{n}{plt}\PY{o}{.}\PY{n}{plot}\PY{p}{(}\PY{n}{x}\PY{p}{,} \PY{n}{y}\PY{p}{,} \PY{l+s+s1}{\PYZsq{}}\PY{l+s+s1}{r}\PY{l+s+s1}{\PYZsq{}}\PY{p}{,} \PY{n}{linewidth}\PY{o}{=}\PY{l+m+mi}{2}\PY{p}{,} \PY{n}{zorder}\PY{o}{=}\PY{l+m+mi}{5}\PY{p}{)}
          
          \PY{n}{plt}\PY{o}{.}\PY{n}{bar}\PY{p}{(}\PY{n}{ix}\PY{p}{,} \PY{n}{iy}\PY{p}{,} \PY{n}{width}\PY{o}{=}\PY{l+m+mf}{1.1}\PY{p}{,} \PY{n}{color}\PY{o}{=}\PY{l+s+s1}{\PYZsq{}}\PY{l+s+s1}{b}\PY{l+s+s1}{\PYZsq{}}\PY{p}{,} \PY{n}{align}\PY{o}{=}\PY{l+s+s1}{\PYZsq{}}\PY{l+s+s1}{edge}\PY{l+s+s1}{\PYZsq{}}\PY{p}{,}\PY{n}{ec}\PY{o}{=}\PY{l+s+s1}{\PYZsq{}}\PY{l+s+s1}{olive}\PY{l+s+s1}{\PYZsq{}}\PY{p}{,} \PY{n}{ls}\PY{o}{=}\PY{l+s+s1}{\PYZsq{}}\PY{l+s+s1}{\PYZhy{}}\PY{l+s+s1}{\PYZsq{}}\PY{p}{,} \PY{n}{lw}\PY{o}{=}\PY{l+m+mi}{2}\PY{p}{,}\PY{n}{zorder}\PY{o}{=}\PY{l+m+mi}{5}\PY{p}{)}
          
          \PY{n}{plt}\PY{o}{.}\PY{n}{figtext}\PY{p}{(}\PY{l+m+mf}{0.9}\PY{p}{,} \PY{l+m+mf}{0.05}\PY{p}{,} \PY{l+s+s1}{\PYZsq{}}\PY{l+s+s1}{\PYZdl{}x\PYZdl{}}\PY{l+s+s1}{\PYZsq{}}\PY{p}{)}
          \PY{n}{plt}\PY{o}{.}\PY{n}{figtext}\PY{p}{(}\PY{l+m+mf}{0.1}\PY{p}{,} \PY{l+m+mf}{0.9}\PY{p}{,} \PY{l+s+s1}{\PYZsq{}}\PY{l+s+s1}{\PYZdl{}y\PYZdl{}}\PY{l+s+s1}{\PYZsq{}}\PY{p}{)}
          
          \PY{n}{ax}\PY{o}{.}\PY{n}{spines}\PY{p}{[}\PY{l+s+s1}{\PYZsq{}}\PY{l+s+s1}{left}\PY{l+s+s1}{\PYZsq{}}\PY{p}{]}\PY{o}{.}\PY{n}{set\PYZus{}visible}\PY{p}{(}\PY{k+kc}{True}\PY{p}{)}
          \PY{n}{ax}\PY{o}{.}\PY{n}{spines}\PY{p}{[}\PY{l+s+s1}{\PYZsq{}}\PY{l+s+s1}{right}\PY{l+s+s1}{\PYZsq{}}\PY{p}{]}\PY{o}{.}\PY{n}{set\PYZus{}visible}\PY{p}{(}\PY{k+kc}{True}\PY{p}{)}
          \PY{n}{ax}\PY{o}{.}\PY{n}{xaxis}\PY{o}{.}\PY{n}{set\PYZus{}major\PYZus{}locator}\PY{p}{(}\PY{n}{ticker}\PY{o}{.}\PY{n}{IndexLocator}\PY{p}{(}\PY{n}{base}\PY{o}{=}\PY{l+m+mi}{1}\PY{p}{,} \PY{n}{offset}\PY{o}{=}\PY{l+m+mi}{0}\PY{p}{)}\PY{p}{)}
          \PY{n}{plt}\PY{o}{.}\PY{n}{xlim}\PY{p}{(}\PY{o}{\PYZhy{}}\PY{l+m+mi}{6}\PY{p}{,}\PY{l+m+mi}{6}\PY{p}{)}
          \PY{n}{plt}\PY{o}{.}\PY{n}{ylim}\PY{p}{(}\PY{o}{\PYZhy{}}\PY{l+m+mi}{100}\PY{p}{,}\PY{l+m+mi}{100}\PY{p}{)}
          
          \PY{n}{plt}\PY{o}{.}\PY{n}{show}\PY{p}{(}\PY{p}{)}
\end{Verbatim}


    \begin{center}
    \adjustimage{max size={0.9\linewidth}{0.9\paperheight}}{Use_PY_in_Calculus_files/Use_PY_in_Calculus_81_0.png}
    \end{center}
    { \hspace*{\fill} \\}
    
    \begin{Verbatim}[commandchars=\\\{\}]
{\color{incolor}In [{\color{incolor}113}]:} \PY{k+kn}{import} \PY{n+nn}{numpy} \PY{k}{as} \PY{n+nn}{np}
          \PY{k+kn}{import} \PY{n+nn}{matplotlib}\PY{n+nn}{.}\PY{n+nn}{pyplot} \PY{k}{as} \PY{n+nn}{plt}
          \PY{k+kn}{import} \PY{n+nn}{matplotlib}\PY{n+nn}{.}\PY{n+nn}{ticker} \PY{k}{as} \PY{n+nn}{ticker}
          
          \PY{k}{def} \PY{n+nf}{func}\PY{p}{(}\PY{n}{x}\PY{p}{)}\PY{p}{:}
              \PY{n}{n} \PY{o}{=} \PY{l+m+mi}{10}
              \PY{k}{return} \PY{n}{n} \PY{o}{/} \PY{p}{(}\PY{n}{n} \PY{o}{*}\PY{o}{*} \PY{l+m+mi}{2} \PY{o}{+} \PY{n}{x} \PY{o}{*}\PY{o}{*} \PY{l+m+mi}{3}\PY{p}{)}
          
          \PY{n}{a}\PY{p}{,} \PY{n}{b} \PY{o}{=} \PY{l+m+mi}{2}\PY{p}{,} \PY{l+m+mi}{9}  \PY{c+c1}{\PYZsh{} integral limits}
          \PY{n}{x} \PY{o}{=} \PY{n}{np}\PY{o}{.}\PY{n}{linspace}\PY{p}{(}\PY{l+m+mi}{0}\PY{p}{,} \PY{l+m+mi}{11}\PY{p}{)}
          \PY{n}{y} \PY{o}{=} \PY{n}{func}\PY{p}{(}\PY{n}{x}\PY{p}{)}
          \PY{n}{x2} \PY{o}{=} \PY{n}{np}\PY{o}{.}\PY{n}{linspace}\PY{p}{(}\PY{l+m+mi}{1}\PY{p}{,} \PY{l+m+mi}{12}\PY{p}{)}
          \PY{n}{y2} \PY{o}{=} \PY{n}{func}\PY{p}{(}\PY{n}{x2}\PY{o}{\PYZhy{}}\PY{l+m+mi}{1}\PY{p}{)}
          \PY{n}{ix} \PY{o}{=} \PY{n}{np}\PY{o}{.}\PY{n}{linspace}\PY{p}{(}\PY{l+m+mi}{1}\PY{p}{,} \PY{l+m+mi}{10}\PY{p}{,} \PY{l+m+mi}{10}\PY{p}{)}
          \PY{n}{iy} \PY{o}{=} \PY{n}{func}\PY{p}{(}\PY{n}{ix}\PY{p}{)}
          
          \PY{n}{fig}\PY{p}{,} \PY{n}{ax} \PY{o}{=} \PY{n}{plt}\PY{o}{.}\PY{n}{subplots}\PY{p}{(}\PY{p}{)}
          \PY{n}{plt}\PY{o}{.}\PY{n}{plot}\PY{p}{(}\PY{n}{x}\PY{p}{,} \PY{n}{y}\PY{p}{,} \PY{l+s+s1}{\PYZsq{}}\PY{l+s+s1}{r}\PY{l+s+s1}{\PYZsq{}}\PY{p}{,} \PY{n}{linewidth}\PY{o}{=}\PY{l+m+mi}{2}\PY{p}{,} \PY{n}{zorder}\PY{o}{=}\PY{l+m+mi}{15}\PY{p}{)}
          \PY{n}{plt}\PY{o}{.}\PY{n}{plot}\PY{p}{(}\PY{n}{x2}\PY{p}{,} \PY{n}{y2}\PY{p}{,} \PY{l+s+s1}{\PYZsq{}}\PY{l+s+s1}{g}\PY{l+s+s1}{\PYZsq{}}\PY{p}{,} \PY{n}{linewidth}\PY{o}{=}\PY{l+m+mi}{2}\PY{p}{,} \PY{n}{zorder}\PY{o}{=}\PY{l+m+mi}{15}\PY{p}{)}
          \PY{n}{plt}\PY{o}{.}\PY{n}{bar}\PY{p}{(}\PY{n}{ix}\PY{p}{,} \PY{n}{iy}\PY{p}{,} \PY{n}{width}\PY{o}{=}\PY{l+m+mi}{1}\PY{p}{,} \PY{n}{color}\PY{o}{=}\PY{l+s+s1}{\PYZsq{}}\PY{l+s+s1}{r}\PY{l+s+s1}{\PYZsq{}}\PY{p}{,} \PY{n}{align}\PY{o}{=}\PY{l+s+s1}{\PYZsq{}}\PY{l+s+s1}{edge}\PY{l+s+s1}{\PYZsq{}}\PY{p}{,} \PY{n}{ec}\PY{o}{=}\PY{l+s+s1}{\PYZsq{}}\PY{l+s+s1}{olive}\PY{l+s+s1}{\PYZsq{}}\PY{p}{,} \PY{n}{ls}\PY{o}{=}\PY{l+s+s1}{\PYZsq{}}\PY{l+s+s1}{\PYZhy{}\PYZhy{}}\PY{l+s+s1}{\PYZsq{}}\PY{p}{,} \PY{n}{lw}\PY{o}{=}\PY{l+m+mi}{2}\PY{p}{,}\PY{n}{zorder}\PY{o}{=}\PY{l+m+mi}{10}\PY{p}{)}
          \PY{n}{plt}\PY{o}{.}\PY{n}{ylim}\PY{p}{(}\PY{n}{ymin}\PY{o}{=}\PY{l+m+mi}{0}\PY{p}{)}
          
          \PY{n}{plt}\PY{o}{.}\PY{n}{figtext}\PY{p}{(}\PY{l+m+mf}{0.9}\PY{p}{,} \PY{l+m+mf}{0.05}\PY{p}{,} \PY{l+s+s1}{\PYZsq{}}\PY{l+s+s1}{\PYZdl{}x\PYZdl{}}\PY{l+s+s1}{\PYZsq{}}\PY{p}{)}
          \PY{n}{plt}\PY{o}{.}\PY{n}{figtext}\PY{p}{(}\PY{l+m+mf}{0.1}\PY{p}{,} \PY{l+m+mf}{0.9}\PY{p}{,} \PY{l+s+s1}{\PYZsq{}}\PY{l+s+s1}{\PYZdl{}y\PYZdl{}}\PY{l+s+s1}{\PYZsq{}}\PY{p}{)}
          
          \PY{n}{ax}\PY{o}{.}\PY{n}{spines}\PY{p}{[}\PY{l+s+s1}{\PYZsq{}}\PY{l+s+s1}{right}\PY{l+s+s1}{\PYZsq{}}\PY{p}{]}\PY{o}{.}\PY{n}{set\PYZus{}visible}\PY{p}{(}\PY{k+kc}{False}\PY{p}{)}
          \PY{n}{ax}\PY{o}{.}\PY{n}{spines}\PY{p}{[}\PY{l+s+s1}{\PYZsq{}}\PY{l+s+s1}{top}\PY{l+s+s1}{\PYZsq{}}\PY{p}{]}\PY{o}{.}\PY{n}{set\PYZus{}visible}\PY{p}{(}\PY{k+kc}{False}\PY{p}{)}
          \PY{n}{ax}\PY{o}{.}\PY{n}{xaxis}\PY{o}{.}\PY{n}{set\PYZus{}major\PYZus{}locator}\PY{p}{(}\PY{n}{ticker}\PY{o}{.}\PY{n}{IndexLocator}\PY{p}{(}\PY{n}{base}\PY{o}{=}\PY{l+m+mi}{1}\PY{p}{,} \PY{n}{offset}\PY{o}{=}\PY{l+m+mi}{1}\PY{p}{)}\PY{p}{)}
          \PY{n}{plt}\PY{o}{.}\PY{n}{show}\PY{p}{(}\PY{p}{)}
\end{Verbatim}


    \begin{center}
    \adjustimage{max size={0.9\linewidth}{0.9\paperheight}}{Use_PY_in_Calculus_files/Use_PY_in_Calculus_82_0.png}
    \end{center}
    { \hspace*{\fill} \\}
    
    \hypertarget{ux4e0dux5b9aux7a4dux5206indefinite-integral}{%
\paragraph{不定積分(indefinite
integral)}\label{ux4e0dux5b9aux7a4dux5206indefinite-integral}}

如果,我們將求導看作一個高階函數,輸入進去的一個函數,求導後成為一個新的函數。那麼不定積分可以視作求導的「反函數」,\(F'(x) = f(x)\)
,則\(\int f(x)dx = F(x) + C\),
寫成類似於反函數之間的複合的形式有:\(\int((\frac{d}{dx}F(x))dx) = F(x) + C, \ \ C \in R\)
即,在微積分中,一個函數\(f = f\)的不定積分,也稱為\textbf{原函數}或\textbf{反函數},是一個導數等於\$
f=f \(的函數\) f = F \(,即,\)f = F' =
f\$。不定積分和定積分之間的關係,由 微積分基本定理 確定。
\(\int f(x) dx = F(x) + C\) 其中\(f = F\) 是
\(f = f\)的不定積分。這樣,許多函數的定積分的計算就可以簡便的通過求不定積分來進行了。
這裡介紹\texttt{python}中的實現方法

    \begin{Verbatim}[commandchars=\\\{\}]
{\color{incolor}In [{\color{incolor}108}]:} \PY{n+nb}{print}\PY{p}{(}\PY{n}{a}\PY{o}{.}\PY{n}{integrate}\PY{p}{(}\PY{p}{)}\PY{p}{)}
          \PY{n+nb}{print}\PY{p}{(}\PY{n}{sympy}\PY{o}{.}\PY{n}{integrate}\PY{p}{(}\PY{n}{sympy}\PY{o}{.}\PY{n}{E}\PY{o}{*}\PY{o}{*}\PY{n}{t}\PY{o}{+}\PY{l+m+mi}{3}\PY{o}{*}\PY{n}{t}\PY{o}{*}\PY{o}{*}\PY{l+m+mi}{2}\PY{p}{)}\PY{p}{)}
\end{Verbatim}


    \begin{Verbatim}[commandchars=\\\{\}]
t**3 - 3*t
t**3 + exp(t)

    \end{Verbatim}

    \hypertarget{ux5e38ux5faeux5206ux65b9ux7a0bordinary-differential-equationsode}{%
\subsection{常微分方程(Ordinary Differential
Equations,ODE)}\label{ux5e38ux5faeux5206ux65b9ux7a0bordinary-differential-equationsode}}


我們觀察一輛行駛的汽車,假設我們發現函數\(a(t)\)能夠很好地描述這輛汽車在各個時刻的加速度,因為對速度的時間函數(v-t)求導可以得到加速度的時間函數(a-t),如果我們希望根據\(a(t)\)求出\(v(t)\),很自然就會得出下面的方程:
\(\frac{dv}{dt}=a(t)\);如果我們能夠找到一個函數滿足:\(\frac{dv}{dt} = a(t)\),那麼\(v(t)\)就是上面房車的其中一個解,因為常數項求導的結果是\(0\),那麼\(\forall C \in R\),\(v(t)+C\)也都是這個方程的解,因此,常微分方程的解就是\(set \ = \{v(t) + C\}\)
在得到這一系列的函數後,我們只需要知道任意一個時刻裡汽車行駛的速度,就可以解出常數項\(C\),從而得到最終想要的一個速度時間函數。
如果我們沿用「導數是函數在某一個位置的切線斜率」這一種解讀去看上面的方正,就像是我們知道了一個函數在各個位置的切線斜率,反過來曲球這個函數一樣。

    \begin{Verbatim}[commandchars=\\\{\}]
{\color{incolor}In [{\color{incolor}115}]:} \PY{k+kn}{import} \PY{n+nn}{sympy}
          \PY{n}{t} \PY{o}{=} \PY{n}{sympy}\PY{o}{.}\PY{n}{Symbol}\PY{p}{(}\PY{l+s+s1}{\PYZsq{}}\PY{l+s+s1}{t}\PY{l+s+s1}{\PYZsq{}}\PY{p}{)}
          \PY{n}{c} \PY{o}{=} \PY{n}{sympy}\PY{o}{.}\PY{n}{Symbol}\PY{p}{(}\PY{l+s+s1}{\PYZsq{}}\PY{l+s+s1}{c}\PY{l+s+s1}{\PYZsq{}}\PY{p}{)}
          \PY{n}{domain} \PY{o}{=} \PY{n}{np}\PY{o}{.}\PY{n}{linspace}\PY{p}{(}\PY{o}{\PYZhy{}}\PY{l+m+mi}{3}\PY{p}{,}\PY{l+m+mi}{3}\PY{p}{,}\PY{l+m+mi}{100}\PY{p}{)}
          \PY{n}{v} \PY{o}{=} \PY{n}{t}\PY{o}{*}\PY{o}{*}\PY{l+m+mi}{3}\PY{o}{\PYZhy{}}\PY{l+m+mi}{3}\PY{o}{*}\PY{n}{t}\PY{o}{\PYZhy{}}\PY{l+m+mi}{6}
          \PY{n}{a} \PY{o}{=} \PY{n}{v}\PY{o}{.}\PY{n}{diff}\PY{p}{(}\PY{p}{)}
              
          \PY{k}{for} \PY{n}{p} \PY{o+ow}{in} \PY{n}{np}\PY{o}{.}\PY{n}{linspace}\PY{p}{(}\PY{o}{\PYZhy{}}\PY{l+m+mi}{2}\PY{p}{,}\PY{l+m+mi}{2}\PY{p}{,}\PY{l+m+mi}{20}\PY{p}{)}\PY{p}{:}
              \PY{n}{slope} \PY{o}{=} \PY{n}{a}\PY{o}{.}\PY{n}{subs}\PY{p}{(}\PY{n}{t}\PY{p}{,}\PY{n}{p}\PY{p}{)}
              \PY{n}{intercept} \PY{o}{=} \PY{n}{sympy}\PY{o}{.}\PY{n}{solve}\PY{p}{(}\PY{n}{slope}\PY{o}{*}\PY{n}{p}\PY{o}{+}\PY{n}{c}\PY{o}{\PYZhy{}}\PY{n}{v}\PY{o}{.}\PY{n}{subs}\PY{p}{(}\PY{n}{t}\PY{p}{,}\PY{n}{p}\PY{p}{)}\PY{p}{,}\PY{n}{c}\PY{p}{)}\PY{p}{[}\PY{l+m+mi}{0}\PY{p}{]}
              \PY{n}{lindomain} \PY{o}{=} \PY{n}{np}\PY{o}{.}\PY{n}{linspace}\PY{p}{(}\PY{n}{p}\PY{o}{\PYZhy{}}\PY{l+m+mi}{1}\PY{p}{,}\PY{n}{p}\PY{o}{+}\PY{l+m+mi}{1}\PY{p}{,}\PY{l+m+mi}{20}\PY{p}{)}
              \PY{n}{plt}\PY{o}{.}\PY{n}{plot}\PY{p}{(}\PY{n}{lindomain}\PY{p}{,}\PY{n}{slope}\PY{o}{*}\PY{n}{lindomain}\PY{o}{+}\PY{n}{intercept}\PY{p}{,}\PY{l+s+s1}{\PYZsq{}}\PY{l+s+s1}{red}\PY{l+s+s1}{\PYZsq{}}\PY{p}{,}\PY{n}{linewidth} \PY{o}{=} \PY{l+m+mi}{1}\PY{p}{)}
                  
          \PY{n}{plt}\PY{o}{.}\PY{n}{plot}\PY{p}{(}\PY{n}{domain}\PY{p}{,}\PY{p}{[}\PY{n}{v}\PY{o}{.}\PY{n}{subs}\PY{p}{(}\PY{n}{t}\PY{p}{,}\PY{n}{i}\PY{p}{)} \PY{k}{for} \PY{n}{i} \PY{o+ow}{in} \PY{n}{domain}\PY{p}{]}\PY{p}{,}\PY{n}{linewidth} \PY{o}{=} \PY{l+m+mi}{2}\PY{p}{)}
\end{Verbatim}


\begin{Verbatim}[commandchars=\\\{\}]
{\color{outcolor}Out[{\color{outcolor}115}]:} [<matplotlib.lines.Line2D at 0x6ec982d0>]
\end{Verbatim}
            
    \begin{center}
    \adjustimage{max size={0.9\linewidth}{0.9\paperheight}}{Use_PY_in_Calculus_files/Use_PY_in_Calculus_86_1.png}
    \end{center}
    { \hspace*{\fill} \\}
    
    \hypertarget{ux65cbux8f49ux9ad4rotator}{%
\subsection{旋轉體(Rotator)}\label{ux65cbux8f49ux9ad4rotator}}

分割法是微積分中的第一步,簡單的講,就是講研究對象的一小部分座位單元,放大了仔細研究,找出特徵,然後在總結整體規律。普遍連說,有兩種分割方式:直角坐標系分割和極座標分割。

\hypertarget{ux76f4ux89d2ux5750ux6a19ux7cfbux5206ux5272}{%
\subsubsection{直角坐標系分割}\label{ux76f4ux89d2ux5750ux6a19ux7cfbux5206ux5272}}

對於直角坐標系分割,我們已經很熟悉了,上面講到的``矩陣逼近''其實就是沿著\(x\)軸分割成\(n\)段\(\{\Delta x_i\}\),即。在直角坐標系下分割,是按照自變量進行分割。
\emph{當然,也可以沿著\(y\)軸進行分割。(勒貝格積分)}

    \hypertarget{ux6975ux5750ux6a19ux5206ux5272}{%
\subsubsection{極坐標分割}\label{ux6975ux5750ux6a19ux5206ux5272}}

同樣的,極座標也是按照自變量進行分割。這是由函數的影射關係決定的,一直自變量,通過函數運算,就可以得到函數值。從圖形上看,這樣分割可以是的每個分割單元``不規則的邊''的數量最小,最好是只有一條。所以,在實際問題建模時,重要的是選取合適的坐標系。
\href{https://i.loli.net/2018/06/13/5b1ff2e2bbee6.png}{\includegraphics{https://i.loli.net/2018/06/13/5b1ff2e2bbee6.png}}

    \hypertarget{ux8fd1ux4f3c}{%
\subsubsection{近似}\label{ux8fd1ux4f3c}}

近似,是微積分中重要的一部,通過近似將分割出來的不規則的``單元''近似成一個規則的''單元``。跟上面一樣,我們無法直接計算曲線圍成的面積,但是可以用一個\textbf{相似}的矩形去替代。
1. Riemann
的定義的例子:在待求解的是區間\([a, b]\)上曲線與\(x\)軸圍成的面積,因此套用的是平面的面積公式:\(S_i = h_i \times w_i = f(\xi) \times \Delta x_i\)
2. 極坐標系曲線積分
待求解的是在區間\([\theta_1, \theta_2]\)上曲線與原點圍成的面積,因此套用的圓弧面積公式:\(S_i = \frac{1}{2}\times r_i^2 \times \Delta \theta_i = \frac{1}{2} \times [f(\xi_i)^2 \times \Delta \theta_i\)
3. 平面曲線長度
平面曲線在微觀上近似為一段``斜線'',那麼,它遵循的是``勾股定理''了,即``Pythagoras
定理'':\(\Delta l_i = \sqrt{(\Delta x_i)^2 + (\Delta y_i)^2} = \sqrt{1 + (\frac{\Delta y_i}{\Delta x_i}^2 \Delta x_i}\)
4. 極坐標曲線長度
\(dl = \sqrt{(dx)^2 + (dy)^2 } = \sqrt{ \frac{d^2[r(\theta)\times cos(\theta)]}{d\theta^2} + \frac{d^2[r(\theta)\times sin(\theta)]}{d\theta^2} d\theta } = \sqrt{ r^2(\theta) + r'^2(\theta)}d\theta\)
我們不能直接用弧長公式,弧長公式的推導用了\(\pi\),而\(\pi\)本身就是一個近似值

    \hypertarget{ux6c42ux548c}{%
\subsubsection{求和}\label{ux6c42ux548c}}

前面幾步都是在微觀層面進行的,只有通過``求和''(Remann
和)才能回到宏觀層面:\(\lim_{\lambda \rightarrow 0^+}\sum_{i = 0}^n F_i\)
其中,\(F_i\) 表示各種圍觀單元的公式。

    例題:求(lemniscate)\(\rho^2 = 2a^2 cos(2\theta)\)
圍成的平民啊區域的面積。

    \begin{Verbatim}[commandchars=\\\{\}]
{\color{incolor}In [{\color{incolor}1}]:} \PY{o}{\PYZpc{}}\PY{k}{matplotlib} inline
        \PY{k+kn}{import} \PY{n+nn}{numpy} \PY{k}{as} \PY{n+nn}{np}
        \PY{k+kn}{import} \PY{n+nn}{matplotlib}\PY{n+nn}{.}\PY{n+nn}{pyplot} \PY{k}{as} \PY{n+nn}{plt}
        \PY{n}{alpha} \PY{o}{=} \PY{l+m+mi}{1}
        \PY{n}{theta} \PY{o}{=} \PY{n}{np}\PY{o}{.}\PY{n}{linspace}\PY{p}{(}\PY{l+m+mi}{0}\PY{p}{,} \PY{l+m+mi}{2}\PY{o}{*}\PY{n}{np}\PY{o}{.}\PY{n}{pi}\PY{p}{,} \PY{n}{num}\PY{o}{=}\PY{l+m+mi}{1000}\PY{p}{)}
        \PY{n}{x} \PY{o}{=} \PY{n}{alpha} \PY{o}{*} \PY{n}{np}\PY{o}{.}\PY{n}{sqrt}\PY{p}{(}\PY{l+m+mi}{2}\PY{p}{)} \PY{o}{*} \PY{n}{np}\PY{o}{.}\PY{n}{cos}\PY{p}{(}\PY{n}{theta}\PY{p}{)} \PY{o}{/} \PY{p}{(}\PY{n}{np}\PY{o}{.}\PY{n}{sin}\PY{p}{(}\PY{n}{theta}\PY{p}{)}\PY{o}{*}\PY{o}{*}\PY{l+m+mi}{2} \PY{o}{+} \PY{l+m+mi}{1}\PY{p}{)}
        \PY{n}{y} \PY{o}{=} \PY{n}{alpha} \PY{o}{*} \PY{n}{np}\PY{o}{.}\PY{n}{sqrt}\PY{p}{(}\PY{l+m+mi}{2}\PY{p}{)} \PY{o}{*} \PY{n}{np}\PY{o}{.}\PY{n}{cos}\PY{p}{(}\PY{n}{theta}\PY{p}{)} \PY{o}{*} \PY{n}{np}\PY{o}{.}\PY{n}{sin}\PY{p}{(}\PY{n}{theta}\PY{p}{)} \PY{o}{/} \PY{p}{(}\PY{n}{np}\PY{o}{.}\PY{n}{sin}\PY{p}{(}\PY{n}{theta}\PY{p}{)}\PY{o}{*}\PY{o}{*}\PY{l+m+mi}{2} \PY{o}{+} \PY{l+m+mi}{1}\PY{p}{)}
        \PY{n}{plt}\PY{o}{.}\PY{n}{plot}\PY{p}{(}\PY{n}{x}\PY{p}{,} \PY{n}{y}\PY{p}{)}
        \PY{n}{plt}\PY{o}{.}\PY{n}{grid}\PY{p}{(}\PY{p}{)}
        \PY{n}{plt}\PY{o}{.}\PY{n}{show}\PY{p}{(}\PY{p}{)}
\end{Verbatim}


    \begin{center}
    \adjustimage{max size={0.9\linewidth}{0.9\paperheight}}{Use_PY_in_Calculus_files/Use_PY_in_Calculus_92_0.png}
    \end{center}
    { \hspace*{\fill} \\}
    
    這是一個對稱圖形,只需要計算其中的四分之一區域面積即可

    \begin{Verbatim}[commandchars=\\\{\}]
{\color{incolor}In [{\color{incolor}9}]:} \PY{k+kn}{from} \PY{n+nn}{sympy} \PY{k}{import} \PY{o}{*}
        
        \PY{n}{t}\PY{p}{,} \PY{n}{a} \PY{o}{=} \PY{n}{symbols}\PY{p}{(}\PY{l+s+s1}{\PYZsq{}}\PY{l+s+s1}{t a}\PY{l+s+s1}{\PYZsq{}}\PY{p}{)}
        \PY{n}{f} \PY{o}{=} \PY{n}{a} \PY{o}{*}\PY{o}{*} \PY{l+m+mi}{2} \PY{o}{*} \PY{n}{cos}\PY{p}{(}\PY{l+m+mi}{2} \PY{o}{*} \PY{n}{t}\PY{p}{)}
        \PY{l+m+mi}{4} \PY{o}{*} \PY{n}{integrate}\PY{p}{(}\PY{n}{f}\PY{p}{,} \PY{p}{(}\PY{n}{t}\PY{p}{,} \PY{l+m+mi}{0}\PY{p}{,} \PY{n}{pi} \PY{o}{/} \PY{l+m+mi}{4}\PY{p}{)}\PY{p}{)}
\end{Verbatim}


\begin{Verbatim}[commandchars=\\\{\}]
{\color{outcolor}Out[{\color{outcolor}9}]:} 2*a**2
\end{Verbatim}
            

    % Add a bibliography block to the postdoc
    
    
    
    \end{document}
